% Created 5 June 1984

\documentstyle[twoside]{report}

\oddsidemargin -2pt
\evensidemargin 96pt

\newskip\foobarskip
\foobarskip=10pt plus 5pt minus 5pt
\newlength{\pagewidth}
\setlength{\pagewidth}{\textwidth}
\addtolength\pagewidth\marginparwidth
\addtolength\pagewidth\marginparsep

\makeatletter
\def\ps@headings{\def\@oddfoot{}%
\def\@oddhead{\makebox[\textwidth][l]{\underline{\hbox to \pagewidth{\bf
\firstmark\hfill\thepage}}}}%
\def\@evenfoot{}%
\def\@evenhead{\makebox[\textwidth][r]{\underline{\hbox to \pagewidth{\bf
\thepage\hfill\@lhead}}}}%
\def\chaptermark##1{\mark{}\def\@lhead{##1}}%
\def\sectionmark##1{{\let\protect\noexpand\mark{\thesection 
    \hskip 1em##1}}}}

\pagestyle{headings}




\newlength\examboxwidth
\setlength\examboxwidth{.5\pagewidth}
\addtolength\examboxwidth{-13pt}
\def\exambox{\par\addvspace\foobarskip\setbox0=\hbox 
  to \pagewidth\bgroup \hspace*{3pt}\small \minipage[t]{\examboxwidth}}
\def\midbox{\endminipage\hfill\minipage[t]{\examboxwidth}}
\def\endexambox{\endminipage\hspace{3pt}\egroup 
\setbox0=\hbox{\frame{\vbox{\vskip 3pt \box0 \vskip 3pt}}}
\noindent \rule{0pt}{\ht0}
\marginpar[{\makebox[0pt][l]{\copy0}}]{\makebox[\marginparwidth][r]{\copy0}}
    \par\vskip\foobarskip}

\long\def\exampage#1{\mbox{}\hfill\raisebox{-176pt}{\framebox(145,188){\vbox 
   to 153pt{\hsize=115pt #1}}}\hfill\mbox{}}

\long\def\slidepage#1{\exampage{\xipt \baselineskip=13.5pt \sf #1}}

\parskip 0pt plus 1pt 
\nofiles


\newcommand\bs{\char '134 }  % A backslash character for \tt font
\newcommand\lb{\char '173 }  % A left brace character for \tt font
\newcommand\rb{\char '175 }  % A right brace character for \tt font
\newcommand{\xsp}{\hspace{.3em}}
\newcommand{\Xsp}{\nolinebreak\hspace{.3em}}
\newcommand{\yspace}{\vspace{1.1ex}}
\newcommand{\remindbox}[1]{\begin{center}
\fbox{\parbox{300pt}{\it #1}}
\end{center}}
\newcommand{\warningbox}[1]{\par\vskip 6pt
{\it \noindent\underline{Warning:}
#1}\par\vskip 6pt}

\setlength{\unitlength}{1pt}


\def\LATEX{L\kern-.3em\raise.8ex\hbox{a}\TeX}
\let\TEX = \TeX
\def\BIBTEX{BIB\kern-.1em\TeX}
\def\SLITEX{S\kern-.065em L\kern-.18em\raise.32ex\hbox{i}\kern-.03em\TEX}


% TEMPORARY DEFINITIONS
\def\knuth{The \TEX book}

%&t&\mbox{\tt #}&
%&h&\hbox#& 
%&v&\hbox{\verb"#"}&
%&m&\mbox{#}& 
%&i&\index{#}&
%&x&\xsp{}#& 
%&.&\Xsp.#& 


\begin{document}

\begin{titlepage}
\vspace*{2in}
\begin{center}
\Large Addendum \\to the
Second Preliminary Edition\\
(dated December 13, 1983)\\
of the\\
\LARGE \LATEX{} Manual\\[.3in]
\large Leslie Lamport\\
\today
\end{center}
\end{titlepage}


\appendix

\setcounter{chapter}{2}
\setcounter{page}{156}

\chapter{Making Color Slides
  With \protect\SLITEX}

\section{How \protect\SLITEX{} Makes Colors}

\SLITEX{} \index{!slitex} is a version of \LATEX{} designed for making
color \index{color slides}\index{slides,color}slides, \index{slides}
though you can use it for black-and-white \index{slides,black and
white} slides as well.  You don't need a special printer to make color
slides; \SLITEX{} uses the same black-and-white printer as \LATEX.
You get color slides by copying \SLITEX's output onto colored
transparencies.  To see how this works, suppose you want to make
the following slide:\footnote{The illustrations of slides used here are not
accurate representations of the output actually produced by
\SLITEX.}
\begin{center}
\slidepage{\hbox to \hsize{\small +\hfill +}
    \vfill
    \noindent\hspace*{20pt}\rule{0pt}{25pt}RED\\
    \hspace*{20pt}\rule{0pt}{25pt}BLACK\\
    \hspace*{20pt}\rule{0pt}{25pt}BLUE
    \vfill
    \hbox to \hsize{\small +\hfill \makebox[2em][l]{9}}}
\end{center}
where ``{\sf RED}'' is colored red, ``{\sf BLACK}'' is colored black,
and ``{\sf BLUE}'' is colored blue.  \SLITEX{} would generate
the following three separate pages of output for this slide:

{\setbox0=\hbox to \pagewidth
{\slidepage{\hbox to \hsize{\small +\hfill +}
    \vfill
    \noindent\hspace*{20pt}\rule{0pt}{25pt}RED\\
    \hspace*{20pt}\rule{0pt}{25pt}\\
    \hspace*{20pt}\rule{0pt}{25pt}
    \vfill
    \hbox to \hsize{\small +\hfill \makebox[2em][l]{9}}}\hfill
\slidepage{\hbox to \hsize{\small +\hfill +}
    \vfill
    \noindent\hspace*{20pt}\rule{0pt}{25pt}\\
    \hspace*{20pt}\rule{0pt}{25pt}BLACK\\
    \rule{0pt}{25pt}
    \vfill
    \hbox to \hsize{\small +\hfill \makebox[2em][l]{9}}}\hfill
\slidepage{\hbox to \hsize{\small +\hfill +}
    \vfill
    \rule{0pt}{25pt}\\
    \rule{0pt}{25pt}\\
    \hspace*{20pt}\rule{0pt}{25pt}BLUE
    \vfill
    \hbox to \hsize{\small +\hfill \makebox[2em][l]{9}}}}
\par
\vskip\foobarskip
\noindent \vrule width 0pt height \ht0 depth \dp0
\marginpar[{\makebox[0pt][l]{\copy0}}]{\makebox[\marginparwidth][r]{\copy0}}
    \par\vskip\foobarskip}

These pages are called {\it color \index{color layer} layers}.  Each
color layer is then copied onto a special sheet that produces a
transparency of the appropriate color.  (Such sheets are commercially
available for an assortment of colors.)  The slide is produced by
laying the three transparencies on top of one another.

I will refer to the text that is meant to be colored red on the slide,
and is therefore printed by \SLITEX{} on the red color layer, as ``red
text''.  So, remember that when I write about the color of
a piece of text, I'm referring only to the color layer on which it
appears; \SLITEX{} doesn't print anything in red ink.

It's hard to tell what a slide will look like from the
separate color layers.  Therefore, \SLITEX{} also produces a
black-and-white version of the slide, containing all the color layers
properly superimposed.  When first making a set of slides, you should
generate only the black-and-white versions, making the color layers
after you've fixed all the problems that are visible in the
black-and-white versions.  If you don't want color slides, you
don't have to make any color layers, and can copy the black-and-white
versions onto transparencies.


\section{The Root File}

\SLITEX{} is a separate program that you run the same way you run
\LATEX{}, giving it the first name of an input file whose second name
is \xsp{}\mbox{\tt .tex}\Xsp.  This file is called the {\it root
\index{root file} file}.  As usual, I'll suppose that your root file
is named \index{myfile.tex}\xsp{}\mbox{\tt myfile.tex}\Xsp.  This file
starts out with the customary
\xsp\hbox{\verb"\pagelayout"}\index{!pagelayou}\xsp{} 
and \linebreak %%%%%%
\xsp{}\hbox{\verb"\documentstyle"}\index{!documentstyle}\xsp{}
commands.  The standard page layout and document styles for making
slides are both named \xsp{}\mbox{\tt slides}\Xsp,\index{slides
document style} so your file is likely to begin
\begin{verbatim}
       \pagelayout{slides}
       \documentstyle{slides}
\end{verbatim}
The commands are followed by any declarations that you may want to
make, followed in turn by the \xsp{}\hbox{\verb"\begin{document}"}\Xsp.

Any text that comes after the \xsp{}\hbox{\verb"\begin{document}"}\xsp{}
is treated as
``front \index{front matter} matter''
and not as slide material.  You can use it for notes to
identify the slides.   
\begin{exambox}
\slidepage{\vfill
This is an example of front matter.
Note the different type style, and how the
text is centered on the page. 
\par\vfill}
\midbox
\begin{verbatim}
\begin{document}

This is an example of front matter.
Note the different type style, and how 
the text ...
\end{verbatim}
\end{exambox}

For \SLITEX{} to produce color slides, you have to tell it what colors
you will be using.  This is done with the
\xsp{}\hbox{\verb"\colors"\index{!colors}}\xsp{} command.  The command
\begin{verbatim}
       \colors{red,black,blue}
\end{verbatim}
states that you will be using three colors, which you have named
\xsp{}\mbox{\tt red}\Xsp, \xsp{}\mbox{\tt black}\Xsp, and
\xsp{}\mbox{\tt blue}\Xsp.  \SLITEX{} knows nothing about real colors,
so you could just as well have called your three colors \mbox{\tt
puce}, \mbox{\tt mauve}, and \mbox{\tt fred}.  If you're making only
black-and-white slides, then you don't need a
\xsp{}\hbox{\verb"\colors"}\xsp{} command.

The text of your slides is contained not in \xsp{}\mbox{\tt
myfile.tex}\Xsp, but in a separate {\it slide \index{slide file}
file}.  This file can have any name that ends in \mbox{\tt .tex}; I
will assume that it is called \xsp{}\mbox{\tt myslid.tex}\Xsp.  What
goes into the file \xsp{}\mbox{\tt myslid.tex}\xsp{} is explained
below.  Black-and white-slides are generated by placing the following
command in the root file:\index{!blackandwhite}
\begin{verbatim}
       \blackandwhite{myslid}
\end{verbatim}
Color slides are generated by the command\index{!colorslides}
\begin{verbatim}
       \colorslides{myslid}
\end{verbatim}
The \xsp{}\hbox{\verb"\colorslides"}\xsp{} command generates a set of
color layer pages for each color specified by the
\hbox{\verb"\colors"} command.  For example, the command
\begin{verbatim}
       \colors{red,black,blue}
\end{verbatim}
causes a subsequent \xsp{}\hbox{\verb"\colorslides"}\xsp{} command to
generate first all the red color-layer pages, then the black ones, and
then the blue ones.

As usual, your root file ends with an
\xsp{}\hbox{\verb"\end{document}"}\xsp{} command.

\section{The Slide File}

The main purpose of the root file is to tell \SLITEX{} what colors to
use and where to find the slide file, so the root file tends to be pretty
short.  It's the slide file that actually makes the individual
slides.


\subsection{Slides}

Each slide is produced by a \xsp{}\mbox{\tt \index{slide
environment}slide}\xsp{} environment.  This environment has a single
argument, which is a list of all the colors contained on the slide.
For example, a slide that has the colors \xsp{}\mbox{\tt red}\xsp{}
and \xsp{}\mbox{\tt blue}\xsp{} is created by an environment
\begin{verbatim}
       \begin{slide}{red,blue}
       ...
       \end{slide}
\end{verbatim}
The colors in the argument must have been declared by a
\xsp{}\hbox{\verb"\colors"\index{!colors}}\xsp{} command in the root
file.  They tell \SLITEX{} which color layers to produce for this
particular slide.  If there is green text in the slide, that text will
appear in the black-and-white version, but no green color layer will
be generated unless \xsp{}\mbox{\tt green}\xsp{} is included in the
\xsp{}\mbox{\tt slide}\xsp{} environment's argument.  If you want only
black-and-white slides, then you can use a null argument:
\begin{verbatim}
       \begin{slide}{}
       ...
\end{verbatim}

The text that appears on a slide is produced using ordinary \LATEX{}
commands.  You can use any commands that make sense for slides.
Commands that {\it don't\/} make sense include sectioning commands,
\xsp{}\mbox{\tt figure}\xsp{} and \xsp{}\mbox{\tt table}\xsp{}
environments, indexing commands, commands for generating a
bibliography, and page-breaking commands.  The latter make no sense in
a slide because each slide must fit on a single page.  You can use an
\xsp{}\hbox{\verb"\input"\index{!input}}\xsp{} command, but not an
\xsp{}\hbox{\verb"\include"}\xsp{} command.  Commands for producing
only some of the slides in your slide file are described below.

There are two major differences between the text generated by
\SLITEX{} and that generated by \LATEX{}.  First of all, text is
automatically centered vertically on the slide.  Secondly, and most
noticable, \SLITEX{} uses a set of type \index{type faces} faces
especially chosen for slides.  The characters in these type faces are
much larger than the ones in the corresponding \LATEX{} type faces.
\SLITEX's \xsp{}\hbox{\verb"\normalsize"}\xsp{} produces roughly the
same size characters as \LATEX's \xsp{}\hbox{\verb"\LARGE"}\Xsp.
Also, \SLITEX's ordinary Roman type style is similar to \LATEX's sans
serif style.  Besides Roman, the only other type styles generally
available are italic (\hbox{\verb"\it"}), bold (\hbox{\verb"\bf"}),
and typewriter (\hbox{\verb"\tt"}).

The only commands you need inside a slide that aren't present in
ordinary \LATEX{} input are ones to tell \SLITEX{} what color the text
is.  The \xsp{}\hbox{\verb"\colors"}\xsp{} command in your root file
defines the declarations \index{color declarations} for doing this.
For example, if the root file contains the command
\xsp{}\hbox{\verb"\colors{red,black,blue}"}\Xsp,
then \xsp{}\hbox{\verb"\red"}\Xsp, \xsp{}\hbox{\verb"\black"}\Xsp, and
\xsp{}\hbox{\verb"\blue"}\xsp{} are declarations that specify the
color.  They work just like any other declaration, such as
\xsp{}\hbox{\verb"\bf"}\Xsp, having the same scoping rules.  This is
illustrated below, where only the red color layer is shown.
\begin{exambox}
\slidepage{\hbox to \hsize{\small +\hfill +}
\vfill
This is red text with two
\makebox[50pt]{}.

This is more red text.
\vfill 
\hbox to \hsize{\small +\hfill \makebox[2em][l]{6}}}
\midbox
\begin{verbatim}
\red
\begin{slide}{red,blue}
   This is red text with two 
   {\blue blue words}.

   This is more red text.
\end{slide}
\end{verbatim}
\end{exambox}
A color declaration does not affect the type style, as illustrated
by the following example, where again only the red color layer is
shown.

\begin{exambox}
\slidepage{\hbox to \hsize{\small +\hfill +}
\vfill
\it This is red italic.
\vfill
\hbox to \hsize{\small +\hfill\makebox[2em][l]{7}}}
\midbox
\begin{verbatim}
\begin{slide}{red,blue}
  \begin{blue} This is blue Roman text.
      {\it This is blue italic.

           {\red This is red italic.}

           This is more blue italic.    }
      This is blue Roman.
  \end{blue}
\end{slide}
\end{verbatim}
\end{exambox}

The command \xsp{}\hbox{\verb"\invisible"\index{!invisible}}\xsp{} is
a special color declaration for invisible text.  Invisible text is not
only colorless, appearing in no color layer, but does not appear in
the black-and-white version either.  The use of invisible text is
explained below.

\warningbox{Don't use a color declaration or an 
\xsp\hbox{\tt \bs invisible}\xsp{} command in math mode.}

\warningbox{Certain horizontal lines drawn by\/ {\rm \TEX} will appear
in color layers where they shouldn't.  The offending lines are the
ones produced by \xsp{}\mbox{\tt \index{!underline}\bs underline}\Xsp,
\xsp{}\mbox{\tt \index{!overline}\bs overline}\Xsp, and
\xsp{}\mbox{\tt \index{!frac}\bs frac}\Xsp.}

\subsection{Overlays}

The \xsp{}\mbox{\tt overlay\index{overlay environment}}\xsp{}
environment is exactly the same as the \xsp{}\mbox{\tt slide}\xsp{}
environment except for how the page is numbered.  The first
\xsp{}\mbox{\tt overlay}\xsp{} following slide number~9 is numbered
``9a'', the second one is numbered ``9b'', and so forth.  To make an
overlay that perfectly overlays a slide, the slide and the overlay
should be absolutely identical except that text visible in one should
be invisible in the other.  This is illustrated by the following
example.

\settowidth{\dimen2}{\xipt\sf overlay}
\settowidth{\dimen0}{\xipt \sf An}
\begin{exambox}
\slidepage{\hbox to \hsize{\small +\hfill +}
\vfill
An \makebox[\dimen2]{} goes here.\par
\vfill 
\hbox to \hsize{\small +\hfill \makebox[2em][l]{9}}}
\midbox
\begin{verbatim}
\begin{slide}{red}

\red
An {\invisible overlay}
goes here.

\end{slide}
\end{verbatim}
\end{exambox}
\begin{exambox}
\slidepage{\hbox to \hsize{\small +\hfill +}
\vfill
\makebox[\dimen0]{} overlay \par
\vfill 
\hbox to \hsize{\small +\hfill \makebox[2em][l]{9-a}}}
\midbox
\begin{verbatim}
\begin{overlay}{red}

\invisible
An {\red overlay}
goes here.

\end{overlay}
\end{verbatim}
\end{exambox}
 
The slide and the overlay will match up to read
\[ \mbox{\xipt \sf An overlay goes here.} \]
when placed atop one another.

\subsection{Notes}

It is sometimes convenient to put notes to yourself in with the
slides.  The \mbox{\tt note}\index{note environment} environment
produces a one-page note that appears only in the black-and-white
versions of the slides.  For example,
\begin{exambox}
\slidepage{\vfill This is a note to myself,
perhaps reminding me of
what I wanted to say here.
\vfill 
\hbox to \hsize{\small \hfill 9-1}}
\midbox
\begin{verbatim}
\begin{note}

  This is a note to myself,
  perhaps ...

\end{note}
\end{verbatim}
\end{exambox}

\noindent 
Notes that follow slide number~9 are numbered
``9-1'', ``9-2'', etc.  


\section{Making Some of the Slides}

For making corrections, it's handy to be able to produce a subset of
the slides in your file.  The command\index{!onlyslides}
\begin{verbatim}
       \onlyslides{4,7-13,23}
\end{verbatim}
in the root file will cause the following
\xsp{}\hbox{\verb"\blackandwhite"}\xsp{} and
\xsp{}\hbox{\verb"\colorslides"}\xsp{} commands to generate only
slides numbered 4, 7-13 (inclusive) and 23, plus all of their
overlays.  The slide numbers in the argument must be in ascending
order, and can include nonexistent slides---for example, you can type
\begin{verbatim}
       \onlyslides{10-9999}
\end{verbatim}
to produce all but the first nine slides.  The argument of the
\xsp{}\hbox{\verb"\onlyslides"}\xsp{} command must be nonempty.

There is also an analogous
\xsp{}\hbox{\verb"\onlynotes"\index{!onlynotes}}\xsp{} command to
generate a subset of the notes.  Notes numbered 11-1, 11-2, etc. will
all be generated by specifying page 11 in the argument of the
\xsp{}\hbox{\verb"\onlynotes"}\xsp{} command.

If your input has an \xsp{}\hbox{\verb"\onlyslides"}\xsp{} command and
no \hbox{\verb"\onlynotes"} command, then notes will be produced for
the specified slides.  If there is an
\xsp{}\hbox{\verb"\onlynotes"}\xsp{} command but no
\xsp{}\hbox{\verb"\onlyslide"}\xsp{} command, then no slides will be
produced.  Including both an \xsp{}\hbox{\verb"\onlyslides"}\xsp{} and
an \xsp{}\hbox{\verb"\onlynotes"}\xsp{} command has the expected
effect of producing only the specified slides and notes.

\end{document}

