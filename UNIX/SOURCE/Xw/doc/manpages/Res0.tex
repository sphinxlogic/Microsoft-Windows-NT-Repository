\input local
\def\resource#1#2
  {
  \vskip5mm
  \twelvebf #1
  \vskip5mm
  \tenrm #2
  }
\resource{XtNSelectedElements}{This is a list of the widgets currently marked as selected. An application program can issue a call to XtGetValues on this resource at any time to query the selected elements.}
\resource{XtNalignment}{This specifies the alignment to be applied when drawing the text.  The alignment resource is interpreted without regard to case.\nl\nl Alignment never causes leading or trailing spaces to be stripped.\nl\nl Alignment may have the following values and effects:\nl\nl XwALIGN\_LEFT will cause the left sides of the lines will be vertically aligned.  Specified in resource default file as "Left".\nl\nl XwALIGN\_CENTER will cause the centers of the lines will be vertically aligned.  Specified in resource default file as "Center".\nl\nl XwALIGN\_RIGHT will cause the right sides of the lines will be vertically aligned.  Specified in resource default file as "Right".}
\resource{XtNallowResize}{Allows an application to specify whether the vertical paned manager should allow a pane to request to be resized.  This flag only has an effect after the paned manager and its children have been realized.  If this flag is set to TRUE, the manager will try to honor requests to alter the height of the pane. If false, it will always deny pane requests to resize.}
\resource{XtNancestorSensitive}{This argument specifies whether the immediate parent of the widget will receive input events.  Use the function XtSetSensitive if you are changing the argument to preserve data integrity (see XtNsensitive below).}
\resource{XtNarrowDirection}{This resource is the means by which the arrow direction is set.  It can be defined in either of two ways: Through the .Xdefaults file by the strings "up", "down", "left" and "right".  Within an arg list for use in XtSetValues() by the defines XwARROW\_UP, XwARROW\_DOWN, XwARROW\_LEFT and XwARROW\_RIGHT.}
\resource{XtNassociateChildren}{This resource indicates whether the menu hierarchy controlled by the menu manager is accessible only from within the associated widget, or from within the widget and any of the widget's children.}
\resource{XtNattachTo}{When used in conjunction with a menu manager, this resource provides the means by which the menupane may be attached as a cascade to a menubutton.  The string which is specified represents the name of the menubutton to which the menupane is to be attached; this provides the means by which the menu manager is able to construct the menu tree.  To specify that this menupane should be treated as the top level menupane within the menu tree, this string should contain the name of the menu manager widget, instead of a menubutton widget.  Specifying a NULL string indicates that the menupane will not be presently attached to anything.  If the menupane does not have a menu manager associated with it, then this resource is unused.}
\resource{XtNbackgroundPixmap}{The application can specify a pixmap to be used for tiling the background. The first tile is place at the upper left hand corner of the widget's window.}
\resource{XtNbackgroundTile}{This resource defines the tile to be used for the background of the widget.  It defines a particular tile to be combined with the foreground and background pixel colors.  The \#defines for setting the tile value through an arg list and the strings to be used in the .Xdefaults files are described in XwCreateTile(3X).}
\resource{XtNbackgroundTile}{This resource defines the tile to be used for the background of the widget.  It defines a particular tile to be combined with the foreground and background pixel colors.  The \#defines for setting the tile value through an arg list and the strings to be used in the .Xdefaults files are described in XwCreateTile(3X).}
\resource{XtNbackground}{ImageEdit redefines the core class background resource to default it to the color black.  This is then used as the background color for the widget's window which will be reflected in the grid color.}
\resource{XtNbackground}{This argument specifies the background color for the widget.}
\resource{XtNborderColor}{This argument specifies the color of the border.}
\resource{XtNborderFrame}{The application can specify the thickness of the borderwidth of all panes in the paned manager.  The value must be greater than or equal to 0.}
\resource{XtNborderPixmap}{The application can specify a pixmap to be used for tiling the border. The first tile is place at the upper left hand corner of the border.}
\resource{XtNborderWidth}{This argument sets the width of the border that surrounds the widget's window on all four sides.  The width is specified in pixels.  A width of zero means no border will show.}
\resource{XtNborderWidth}{This redefines the core class default border width from 1 pixel to 0 pixels.}
\resource{XtNbottomMargin}{The number of pixels used for the bottom margin.}
\resource{XtNcallback}{This is used by the paned window widget to be informed of button presses and mouse movement associated with the sash.}
\resource{XtNcascadeImage}{This resource points to an XImage structure which describes the cascade image data.  The cascade area is a fixed size (16x16).  If this resource is set to NULL, then the default cascade image, an arrow, is used. The cascade indicator is not displayed if the XtNcascadeOn resource is set to NULL.  If the image is defined with XYBitmap data, then the image is nicely inverted when the menubutton is highlighted.}
\resource{XtNcascadeOn}{This resource determines if the cascade indicator is displayed.  It is typically set only by the menu manager and contains the widget ID of the menupane which cascades as a submenu from this menubutton.  This resource is set to NULL to disable the display of the cascade indicator.}
\resource{XtNcascadeSelect}{This resource provides the means for registering callback routines which are invoked if a cascade indicator is displayed and the pointer moves into the cascade area.  In some cases, the menu manager suppresses the calling of these callback routines.  The menubutton does not pass any data in the call\_data field of the callback.}
\resource{XtNcascadeUnselect}{This resource provides the means for registering callback routines which are invoked if a cascade indicator is displayed and the pointer moves out of the cascade area.  These callbacks are only invoked if the XtNcascadeSelect callbacks have been previously invoked.  The menubutton passes data in the call\_data field of the callback.  It is a pointer to the XwunselectParams data structure shown below:\nl typedef struct { Dimension         rootX; Dimension         rootY; Boolean           remainHighlighted; } XwunselectParams;\nl The rootX and rootY parameters have the position of the pointer relative to the root window when the event occurred which caused the XtNcascadeUnselect call backs to be called. The remainHighlighted parameter is used by cascading submenus.  It is set by the menu manager's call back routine to indicate that the pointer traversed from a cascade into the submenu.  If the boolean is set TRUE, then the menubutton does not unhighlight on exit.  It also sets up an event handler on its parent menupane so that it is notified if the pointer enters another menubutton, in which case the menubutton should then unhighlight.}
\resource{XtNcausesResize}{Controls whether changes in the child geometry can cause the Panel to make a geometry request of its parent.  If TRUE for only one child, Panel will request changes whenever that child requests changes.  If TRUE for multiple children, Panel will request changes whenever any of that set of children grow, and when all of that set of children have shrunk.}
\resource{XtNcolumnWidth}{The width of each column. If the value is 0, the width defaults to the width of the largest element.}
\resource{XtNcolumns}{The application can specify the number of columns to be used when laying out the widgets children.}
\resource{XtNdepth}{Determines how many bits should be used for each pixel in the widget's window.  Programs should not change or set this, it will be set by the Xt Intrinsics when the widget is created.}
\resource{XtNdestroyCallback}{This is a pointer to a callback list containing routines to be called when the widget is destroyed.}
\resource{XtNdestroyMode}{Controls the visual appearance of the list when an element is deleted. One of XwSHRINK\_COLUMN, XwSHRINK\_ALL or XwNO\_SHRINK.}
\resource{XtNdisplayPosition}{The position in the text source that will be displayed at the top of the screen.  The default is 0, or the start of the text source.}
\resource{XtNdisplayTitle}{Ignored if XtNtopLevel is FALSE.\nl\nl               Otherwise, if XtNdisplayTitle is TRUE, the titlebar child will be displayed.  If XtNdisplayTitle is FALSE, the titlebar child will not be displayed.\nl\nl               This resource should be set by the user in the resource defaults file.  If the user runs the application without a window manager or with a non-titling window manager, this resource should be set to TRUE.  If the user runs with a titling window manager this resource should be set to FALSE.}
\resource{XtNdrawColor}{This resource define the color to be used for drawing in the widget.}
\resource{XtNeditType}{This resource controls the edit state of the source. It can be XttextRead, a read only source, XttextAppend, a source than can only be appended to, and XttextEdit, a fully editable source.}
\resource{XtNeditType}{This resource controls the edit state of the source. It can be XttextRead, a read only source, and XttextAppend, a source than can only be appended to.}
\resource{XtNelementHeight}{The height of each element. Zero implies that each element is resized to the height of the tallest element.}
\resource{XtNelementHighlight}{This controls the highlight mode on selection - either border highlighting (XwBORDER) or inversion (XwINVERT).}
\resource{XtNenter, XtNleave, XtNselect, and XtNrelease}{Callbacks provided for control of TitleBar.  The data parameter is unused.}
\resource{XtNeraseColor}{This resource defines the color used for erasing in the widget.  Erase is enabled by the eraseOn resource. When selections occur on the widget, the widget determines the color of the pixel selected.  If the selected pixel is not the same as the draw color, the draw color will be used to draw until the button release occurs.  If the selected pixel is the draw color, the erase color will be used for drawing until the button release occurs.}
\resource{XtNeraseOn}{This resource is a boolean variable that indicates whether erasing is enabled or not.  If set to TRUE, drawing will occur as described above.  If set to FALSE, only the draw color will be used for drawing.}
\resource{XtNexecute}{This callback list is similar to a selection function on a button.  When the user invokes an event that calls the "execute" function (see the translation table below), this callback list will be executed. In the default translation table, this is bound to the "enter" key.}
\resource{XtNexpose}{This resource defines a callback list which is invoked when an exposure event occurs on the widget.  The call\_data parameter for the callback will contain a Region structure containing the exposed region.}
\resource{XtNfile}{The absolute pathname of a disk file to be viewed and/or edited.  If no file is given, a temporary file will be created.}
\resource{XtNfont}{If the title type resource indicates that a title string should be displayed, then this resource will describe the font used to draw the title string.}
\resource{XtNfont}{The application may define the font to be used when displaying the button string.  Any valid X11 font may be used.}
\resource{XtNfont}{The font used to display the text.  The default is fixed.  There are currently several display bugs associated with proportional fonts.}
\resource{XtNfont}{This resource controls which font the text will drawn in.}
\resource{XtNforceSize}{The application has the option of forcing the widths of each widget in a column and the heights of each widget in a row to be the same.  This can be used, for example to enforce an orderly layout for a group of buttons. For the layout type of maximum unaligned, only the heights of the widgets in a row are forced to the same size.}
\resource{XtNforeground}{The color for drawing the text.  The default is black.}
\resource{XtNforeground}{This resource defines the foreground color for the widget.  Widgets built upon this class can use the foreground for their drawing.}
\resource{XtNforeground}{This resource defines the foreground color for the widget.  Widgets built upon this class can use the foreground for their drawing.}
\resource{XtNframed}{The application can specify whether the panes should be displayed with some padding surrounding each pane (TRUE) or whether the panes should be set flush with the paned manager (FALSE).}
\resource{XtNgranularity}{This resource defines the increment in the valuator's coordinate system to move the slider while continuous scrolling.}
\resource{XtNgravity}{This resource controls the use of extra space within the widget.\nl\nl CenterGravity will cause the string to be centered in the extra space.  Specified in the resource defaults file as "CenterGravity".\nl\nl NorthGravity will cause the string to always to be at the top of the window centered in any extra width.  Specified in the resource defaults file as "NorthGravity".\nl\nl SouthGravity will cause the string to always to be at the bottom of the window centered in any extra width.  Specified in the resource defaults file as "SouthGravity".\nl\nl EastGravity will cause the string to always be at the right of the window centered in any extra height.  Specified in the resource defaults file as "EastGravity".\nl\nl WestGravity will cause the string to always be at the left of the window centered in any extra height.  Specified in the resource defaults file as "WestGravity".\nl\nl NorthWestGravity will cause the string to always be in the upper left corner of the window. Specified in the resource defaults file as "NorthWestGravity".\nl\nl NorthEastGravity will cause the string to always be in the upper right corner of the window. Specified in the resource defaults file as "NorthEastGravity".\nl\nl SouthWestGravity will cause the string to always be in the lower left corner of the window. Specified in the resource defaults file as "SouthWestGravity".\nl\nl SouthEastGravity will cause the string to always be in the lower right corner of the window. Specified in the resource defaults file as "SouthEastGravity".}
\resource{XtNgridThickness}{This resource defines the separation between the magnified pixels.}
\resource{XtNgrow}{This resource controls if the widget will try to resize its window when it needs more height or width to display the text.  When set to XwGrowOff it will not resize itself.  When set to XwGrowHorizontal it will attempt to change its width when lines are too long for the current screen width.  When set to XwGrowVertical it will attempt to resize its height when the number of text lines is greater than can be displayed with the current screen height.  When set to XwGrowBoth, the widget will attempt resizes in both dimensions.  Growth attempts have higher priority than either wrapping or scrolling.  If enabled, the widget will always try to grow to display text before trying to wrap or scroll. The default is XwGrowOff.  The success of a resize request is determined by the widget's parent.}
\resource{XtNhSpace}{Padding between the sides of the Panel and the sides of the displayed children.}
\resource{XtNhSpace}{The amount of space to maintain between the right and left of the titlebar and the interior widgets.}
\resource{XtNhSpace}{The application may determine the number of pixels of space left between each element within a given row. This defines a minimum spacing.}
\resource{XtNhSpace}{The application may determine the number of pixels of space left between the left side of the button and the leftmost part of the button label, and between the rightmost part of the button label and the right side of the button.}
\resource{XtNhSpace}{This specifies the number of pixels to maintain between the text and the highlight area to the right and left of the text.}
\resource{XtNheight}{This argument contains the height of the widget's window in pixels, not including the border area.  Programs should not change this argument directly, but use geometry manager requests instead in order to ensure proper relationships with other widgets are maintained.}
\resource{XtNhighlightColor}{This resource defines the color to be used in the highlighting drawn by Primitive around the exterior of the widget.}
\resource{XtNhighlightStyle}{Two styles of border highlighting are supported by Primitive.  They include drawing the highlighting with a pattern and widget specific border drawing.  To set the highlight style through an arg list, use the \#define XwPATTERN\_BORDER.  To set the highlight style through the .Xdefaults file, use the string pattern\_border.\nl For Widget Writers:  The highlighting style of}{XwWIDGET\_DEFINED is used exclusively by widgets with special highlighting requirements that need to override the normal highlighting types.  To use, the widget inserts a highlight and unhighlight procedure into its primitive class and forces the highlightStyle field in the primitive instance to the define XwWIDGET\_DEFINED. The primitive class will then make the appropriate calls to the highlight and dehighlight functions.}
\resource{XtNhighlightThickness}{The width of the highlight can be set using this resource.  It is specified as an integer value representing the width, in pixels, of the highlight to be drawn.  This value must be greater than or equal to 0.  Note that highlighting takes place within the window created for a widget and is separate from the window border.}
\resource{XtNhighlightThickness}{This resource specifies an amount of border spacing around the border of the widget.  It is typically used by managers to have padding space around their children and to draw special borders.  This highlight thickness is and an integer value representing the width, in pixels, of the border area.  This value must be greater than or equal to 0.}
\resource{XtNhighlightTile}{When the highlight style is XwPATTERN\_BORDER, one of several tiles can be used for the drawing.  The \#defines for setting the values through an arg list and the strings to be used in the .Xdefaults files are described in XwCreateTile(3X).}
\resource{XtNimage}{This is a pointer to the image that is displayed in the grid.  It points to an XImage structure.}
\resource{XtNinitialDelay}{The ScrollBar supports smooth time sequenced movement of the slider when a selection occurs on the arrows. This resource defines the amount of delay to wait between the initial selection and the slider starting its repetitive movement.  The value is defined in milliseconds.}
\resource{XtNinsertPosition}{The position in the text source of the insert cursor. The default is 0.}
\resource{XtNinvertOnSelect}{If this resource is TRUE, the raster image will invert its foreground and background colors when selected, and return to normal when unselected.}
\resource{XtNkbdAccelerator}{This resource is a string which describe a set of modifiers and the key which may be used to select this menubutton widget.  The format for this string is identical to that used by the translations manager, with the exception that only a single event may be specified, and only KeyPress events are allowed.  If the menubutton does not have a menu manager associated with it, then this resource is ignored.  The menu manager determines when, and if, this accelerator is available.}
\resource{XtNkbdSelect}{This string resource describes the key event and any required modifiers needed to select the currently highlighted menu button. This provides the user with the means for selecting a menu item from the keyboard, without being required to use the mouse.  The string is specified using the syntax supported by the Xt Intrinsic's translation manager, with three exceptions. First, only a single event may be specified.  Secondly, the event must be a key event. Thirdly, all modifiers specified are interpreted as being exclusive; this means that only the specified modifiers can be present when the button event occurs.}
\resource{XtNkeyDown}{This resource defines a callback list which is invoked when keyboard input occurs in the widget.  The call\_data parameter for the callback will contain the key pressed event.}
\resource{XtNlabelImage}{If XtNlabelType indicates that a label image should be displayed, then this resource contains the image used. This is a pointer to an XImage structure which describes the label image data.  If the image is defined with XYBitmap data, then the image is nicely inverted when the menubutton is highlighted.}
\resource{XtNlabelLocation}{For those buttons that have a separate graphic, this field specifies whether the label should appear to the left or to the right of that graphic.  The acceptable values are the defines XwRIGHT (the default) and XwLEFT.}
\resource{XtNlabelType}{Two styles of labels are supported by the MenuButton widget: text string labels and image labels.  The text string label is defined by the Button resource XtNlabel and the image label is defined by the XtNlabelImage resource.  To programmatically set this resource, use either the XwSTRING define or the XwIMAGE define.  To set this resource using the .Xdefaults files, use one of the strings string or image.}
\resource{XtNlabel}{The application may define the button label by providing a pointer to a null terminated character string.  If no label is provided the class name of the widget will be used.}
\resource{XtNlayoutType}{The application can specify the type of layout the row column manager is to perform.  Allowable argument list settings are XwREQUESTED\_COLUMNS, XwMAXIMUM\_COLUMNS and XwMAXIMUM\_UNALIGNED.  To set this value in .Xdefaults or another resource file use the strings requested\_columns, maximum\_columns and maximum\_unaligned.}
\resource{XtNlayout}{This flag controls how the manager widget's geometry deals with too little or too much space.  The valid settings for this field are XwMINIMIZE, XwMAXIMIZE and XwIGNORE.  (The counterpart for these settings to be used in resource files, like .Xdefaults, are: minimize, maximize and ignore.)  Typically, the XwMINIMIZE means to request the minimum amount of space necessary to display all children.  The XwMAXIMIZE means that if additional space is given to the widget (i.e., at create time or set values time) then use the additional space as padding between children widgets. The XwIGNORE settings means, maintain the size set at create time or at set value time and never change size in response to a child widget's request (i.e., added/deleted/modified a child widget).  Look at the description of the individual manager widgets to see if this feature is supported.}
\resource{XtNleaveVerification}{This verification callback list is called before the widget loses input focus.  The default is NULL.  See the verification section below.}
\resource{XtNleftMargin}{The number of pixels used for the left margin.\nl\nl NOTE: if TextEdit is embedded in a manager with keyboard traversal enabled, it will silently enforce the constraint that all margins must be at least 3 pixels wider than the highlight border width.}
\resource{XtNlineSpace}{This resource controls the amount of space between lines.  It is specified as a percentage of the font height.  This space is added between each line of text. XtNlineSpace may be negative to a maximum of -100 (which causes all lines to overwrite each other).}
\resource{XtNmappedWhenManaged}{If set to TRUE, the widget will be mapped (made visible) as soon as it is both realized and managed. If set to FALSE, the client is responsible for mapping and unmapping the widget.  If the value is changed from TRUE to FALSE after the widget has been realized and managed, the widget is unmapped.}
\resource{XtNmarkImage}{This resource points to an XImage structure which describes the mark image data.  The mark area is a fixed size (16x16).  If this resource is set to NULL, then the default mark image is used.  The mark is not displayed if the XtNsetMark resource is set to FALSE. If the image is defined with XYBitmap data, then the image is nicely inverted when the menubutton is highlighted.}
\resource{XtNmaximumSize}{The maximum number of characters that can be entered into the internal buffer.  If this value is not set then the internal buffer will increase its size as needed limited only by the space limitations of the process.}
\resource{XtNmax}{Allows an application to specify the maximum size to which a pane may be resized.  This value must be greater than the specified minimum.}
\resource{XtNmenuMgrId}{This resource is used only by menu managers to indicate to the menubutton widget its menu manager.  If this is set to NULL, then the menubutton checks if it has a menu manager at the appropriate level in its parentage. This resource should not be set by users.}
\resource{XtNmenuPost}{This string resource describes the button event and any required modifiers needed to post one of the top level menupanes controlled by the menu manager.  The string is specified using the syntax supported by the Xt Intrinsic's translation manager, with three exceptions.\nl\nl First, only a single event may be specified.  Secondly, the event must be a ButtonPress or ButtonRelease event. Thirdly, all modifiers specified are interpreted as being exclusive; this means that only the specified modifiers can be present when the button event occurs.}
\resource{XtNmenuSelect}{This string resource describes the button event and any required modifiers needed to select a menu button within any of the menupanes controlled by the menu manager.  The string is specified using the syntax supported by the Xt Intrinsic's translation manager, with three exceptions. First, only a single event may be specified.  Secondly, the event must be a ButtonPress or ButtonRelease event.  Thirdly, all modifiers specified are interpreted as being exclusive; this means that only the specified modifiers can be present when the button event occurs.}
\resource{XtNmenuUnpost}{This string resource describes the key event and any required modifiers needed to unpost the currently viewable set of menupanes controlled by the menu manager.  This provides the user with the means for unposting a menu hierarchy from the keyboard, without selecting a menu button.  The string is specified using the syntax supported by the Xt Intrinsic's translation manager, with three exceptions. First, only a single event may be specified.  Secondly, the event must be a key event. Thirdly, all modifiers specified are interpreted as being exclusive; this means that only the specified modifiers can be present when the button event occurs.}
\resource{XtNmgrOverrideMnemonic}{This boolean resource determines if the mnemonic character is underlined in the label string.  If it is set to TRUE, then the mnemonic character is not underlined.  This resource is typically set only by menu managers.}
\resource{XtNmgrTitleOverride}{This resource is not intended to be used by applications; it should only be used by a menu manager widget, for overriding the application, and forcing off the menupane title.  This is useful for those menu managers whose style dictates that certain menupane should not have a title displayed.}
\resource{XtNmin}{Allows an application to specify the mimimum size to which a pane may be resized.  This value must be greater than 0.}
\resource{XtNmnemonic}{Certain menu managers allow some of their menupanes to have a mnemonic.  Mnemonics may be used to post a menupane using the keyboard, instead of using the pointer device.  This resource is a NULL terminated string, containing a single character.  Typically, the character is the same as one present in the menupane title.}
\resource{XtNmnemonic}{Certain menu managers allow the menubuttons to have a mnemonic.  Mnemonics provide the user with another means for selecting a menu button.  This resource is a NULL terminated string, containing a single character. The menu manager determines if this mnemonic is available.  If the XtNmgrOverrideMnemonic resource is false and the mnemonic is found in the label string, then that character is underlined when the menubutton is displayed.  Refer to XwPullDown(3X) man page for further discussion of traversal.}
\resource{XtNmode}{The application can specify whether the selection policy is n\_of\_many or one\_of\_many. Allowable argument list settings are XwONE\_OF\_MANY and XwN\_OF\_MANY.  To set this value in .Xdefaults or another resource file use the strings one\_of\_many and n\_of\_many.}
\resource{XtNmodifyVerification}{This verification callback list is called before text is deleted from or inserted to the text source.  The default is NULL.  See the verification section below.}
\resource{XtNmotionVerification}{This verification callback list is called before the insertion cursor is moved to a new position.  The default is NULL.  See the verification section below.}
\resource{XtNnextTop}{This callback procedure is used by the applications programmer to move the focus from one toplevel widget to another toplevel widget.}
\resource{XtNnumColumns}{The number of columns in the list.}
\resource{XtNnumSelectedElements}{The number of widgets currently selected (in the list pointed to by XtNselectedElements).}
\resource{XtNpadding}{The application can specify how many pixels of padding should surround each pane when it is being displayed in framed mode.  This value must be greater than or equal to 0.}
\resource{XtNpixelScale}{This resource defines the magnification factor to use when displaying the expanded image.}
\resource{XtNpostAccelerator}{This resource indicates the keyboard event that can be used to post the top level menupane.  The string is specified using the syntax supported by the translation manager, with three exceptions.  First, only a single event may be specified.  Second, the event must be a KeyPress or KeyRelease event.  Third, all modifiers specified are interpreted as being exclusive; this means that only the specified modifiers can be present when the key event occurs.}
\resource{XtNprecedence}{When TitleBar is too narrow to display all of its children, this resource is used to determine which children should be hidden.  Widgets with high values of XtNprecedence are hidden first.  Precedence values are relative to all other widgets within an instantiation of TitleBar.  This means that all widgets, regardless of their region, with high values of XtNprecedence will be hidden before any widgets with the next lower values are hidden.\nl Values of XtNprecedence need not be unique.  If values are unique, there is no question about which widget is first to lose its padding, nor about which widget is first to be hidden.\nl Ifvalues are not unique for all children of TitleBar, there need be no question about which widget is acted on first, but it is dependent on both insertion order and precedence.  The last widget inserted in TitleBar of a given precedence is the first to lose its requested padding  (of widgets with that priority). Widgets lose padding from last inserted to first inserted, within a given level of precedence.  When hiding widgets, widgets within a given precedence level are hidden from last inserted to first inserted.\nl XtNsquare If True, forces the button to draw a square box, otherwise it will draw a diamond shape box.  One possible usage for this resource is to make the convention that row column managers containing diamond shaped toggles have their XtNmode resource set to one\_of\_many which will only allow one of the buttons to be set at any one time, while row column managers containing square buttons use the default mode setting of n\_of\_many which allows any or all of the buttons to be set.}
\resource{XtNrecomputeSize}{This boolean resource indicates to a primitive widget whether it should recalculate its size when an application makes a XtSetValues call to it.  If set to TRUE, the widget will perform its normal size calculations will may cause its geometry to change.  If set to FALSE, the widget will not recalculate its size.}
\resource{XtNrefigureMode}{This setting is useful if a large number of programmatic manipulations are taking place.  It will prevent the manager from recomputing and displaying new positions for the child panes (FALSE).  Once the changes have been executed this flag should be set to TRUE to allow the vertical paned manager to show the correct positions of the current children.}
\resource{XtNregion}{Associates a child with a region of the titlebar.  The regions may be specified in the resource default file as "left" for XwALIGN\_LEFT, "center" for XwALIGN\_CENTER, and "right" for XwALIGN\_RIGHT.\nl During layout widgets with XtNregion values of XwALIGN\_LEFT grouped to the left end of TitleBar. Widgets with XtNregion values of XwALIGN\_LEFT are grouped to the right of TitleBar.  Widgets with XtNregion values of XwALIGN\_CENTER will be grouped between the left and right groups.  Additionally, TitleBar tries to center the center group within the TitleBar.\nl Widgets for which XtNregion is unspecified or XwALIGN\_NONE when XtNstring is non-null, will be children's padding requests without hiding some children, some, possibly all, padding requests will be collapsed.}
\resource{XtNrelease}{This is a reserved callback list used by widget subclasses built upon Primitive to implement there callback lists.}
\resource{XtNrepeatRate}{This resource defines the continuous repeat rate to use to move the slider while the button is being held down on an arrow.  The value is also defined in milliseconds.}
\resource{XtNresize}{This resource defines a callback list which is invoked when the widget is resized.  The widget parameter can be accessed to obtain the new size of the widget.}
\resource{XtNrightMargin}{The number of pixels used for the right margin.}
\resource{XtNrowPosition,XtNcolumnPosition}{This is the row,column location of the element in the list.  If these values are greater than or equal to zero, the widget is inserted into the list at that position. If the values are left at -1, the List widget will create a list with XtNnumColumns number of columns, assigning row and column positions as needed.}
\resource{XtNsRimage}{This is a pointer to an XImage data structure.}
\resource{XtNsashIndent}{This controls where along the bottom of the pane the control widget (the pane's sash) will be placed.  A positive number will cause the sash to be offset from the left side of the pane, a negative number will cause the sash to be offset from the right side of the pane. If the offset specified is greater than the width of the vertical paned manager, minus the width of the sash, the sash will be placed flush against the left hand side of the paned manager.}
\resource{XtNscroll}{This resource controls the horizontal and vertical scrolling of lines longer than the screen width.  When set to XwAutoScrollOff the widget will not scroll. When set to XwAutoScrollVertical, the widget will scroll lines vertically.  When set to XwAutoScrollHorizontal, the widget will scroll a single-line display horizontally.  Horizontal scrolling is not currently supported for multi-line displays. Both horizontal and vertical scrolling can be set with XwAutoScrollBoth (again, subject to the single-line horizontal restriction).  The default is XwAutoScrollOff.  XtNscroll has lower priority than XtNwrap, meaning if wrapping is enabled, the widget will attempt to wrap to the next line before it will attempt to scroll horizontally.}
\resource{XtNselectColor}{Allows the application to specify what color should be used to fill in the center of the square (or the diamond) when it is set.}
\resource{XtNselectionBias}{Bias mode - either XwNO\_BIAS, XwROW\_BIAS or XwCOL\_BIAS.}
\resource{XtNselectionLeft}{The starting position of the initial selection.  The default is 0.}
\resource{XtNselectionMethod}{Controls the selection mode - either one element at a time (XwSINGLE) or multiple (XwMULTIPLE).}
\resource{XtNselectionRight}{The ending position of the initial selection.  The default is 0.}
\resource{XtNselectionStyle}{Controls the type of selection - either XwINSTANT or XwSTICKY.}
\resource{XtNselect}{This is a reserved callback list used by widget subclasses built upon Primitive to implement there callback lists.}
\resource{XtNselect}{This resource provides the means for registering callback routines which will be invoked when the menupane receives a select action.}
\resource{XtNsensitiveTile}{The application can determine the mix of foreground and background that will be used to draw text to show insensitivity. The \#defines for setting the values through an arg list and the strings to be used in the .Xdefault file are described in XwCreateTile(3X).  The default is Xw75\_FOREGROUND which is a 75/25 mix of foreground and background colors.}
\resource{XtNsensitive}{This argument determines whether a widget will receive input events.  If a widget is sensitive, the Xt Intrinsic's Event Manager will dispatch to the widget all keyboard, mouse button, motion, window enter/leave, and focus events.  Insensitive widgets do not receive these events.  Use the function XtSetSensitive if you are changing the sensitivity argument.  That way you ensure that if a parent widget has XtNsensitive set to FALSE, the ancestor-sensitive flag of all its children will be appropriately set.}
\resource{XtNseparatorType}{This resource defines the type of line drawing to be done in the menu separator widget.  Five different line drawing styles are provided.  They are single, double, single dashed, double dashed and no line.  The separator type can be set through an argument list by using one of the defines: XwSINGLE\_LINE, XwDOUBLE\_LINE, XwSINGLE\_DASHED\_LINE, XwDOUBLE\_DASHED\_LINE, and XwNO\_LINE.  The separator type can be set through the .Xdefaults file by using one of the following strings: single\_line, double\_line single\_dashed\_line, double\_dashed\_line and no\_line.}
\resource{XtNsetMark}{This boolean resource determines whether the mark is displayed.}
\resource{XtNset}{If set to true the button will display itself in its selected state.  This is useful for showing some conditions as active when a set of buttons appear.}
\resource{XtNset}{This is a Boolean resource which indicates whether the raster is currently selected (TRUE) or not (FALSE).}
\resource{XtNshowSelected}{If TRUE, this will cause the image to appear to be indented when selected, and raised when unselected.}
\resource{XtNsingleRow}{For layout types of maximum columns and maximum unaligned, the application has the option of having the row column manager to try to lay itself out in a single row whenever one of its children makes a geometry request.}
\resource{XtNskipAdjust}{Allows an application to specify that the vertical paned manager should not automatically resize this pane (flag set to TRUE).}
\resource{XtNslideOrientation}{The Valuator widget supports both horizontal and vertical scrolling.  This resource type is the means by which this is set.  It can be defined through the .Xdefaults file by the strings "horizontal", and "vertical" or within an arg list for use in XtSetValues() by the defines XwHORIZONTAL and XwVERTICAL.}
\resource{XtNsliderExtent}{The size of the slider can be set by the application. The acceptable values are 0 < sliderExtent < (sliderMax - sliderMin).}
\resource{XtNsliderMin, XtNsliderMax}{The Valuator widget lets the application define its own coordinate system for the valuator.  Any integer values with sliderMin less than sliderMax can be specified.}
\resource{XtNsliderMoved, XtNsliderReleased, XtNareaSelected}{The Valuator widget defines three types of callback lists which get invoked upon different event conditions when interacting with a valuator. All types use the data parameter to send the location of the slider to the callback functions.\nl The first callback type, sliderMoved, defines the}{callback list containing the callback functions called when the slider is interactively moved.\nl The second callback type, sliderReleased, defines a callback list containing callback functions called when the mouse button is released after moving the slider.\nl The third callback type, areaSelected, defines a callback list containing the callback functions called when an area in a valuator not containing the slider is selected.  The slider is not moved to this position but if the application wants the slider moved, it can use the position value contained in the parameter call\_data and perform a XtSetValues() to its valuator.\nl For the callback types, the call\_data parameter of the callback function will be an integer containing the slider or selection position.}
\resource{XtNsliderOrigin}{The location of the slider can be set by the application.  The acceptable values are between sliderMin and (sliderMax - sliderExtent).}
\resource{XtNsliderTile}{This resource is used to set the tile used to create the pixmap to use when drawing the slider.  The \#defines for setting the values through an arg list and the strings to be used in the .Xdefaults files are described in XwCreateTile(3X).}
\resource{XtNsourceType}{This defines the type of the text source. It is one of "stringsrc," "disksrc" or "progdefinedsource."}
\resource{XtNsource}{This specifies a new Source.  The default is StringSrc.}
\resource{XtNstickyMenus}{This resource controls whether the menu manager operates in sticky menu mode.}
\resource{XtNstring}{The initial string to be viewed and/or edited.  The default is the empty string.  An XtGetValues call on this resource will return a copy of the internal buffer. The application program is responsible for freeing the space allocated by the copy. An XtSetValues call will copy the given string into the internal buffer.}
\resource{XtNstring}{This resource is the string which will be drawn.  The string must be null terminated.  If the string is given in a resource defaults file, newlines may be specified by "$\backslash$n" within the string.}
\resource{XtNstrip}{This resource controls the stripping of leading an trailing spaces during the layout of the text string. If XtNstrip is FALSE, spaces are not stripped.  If XtNstrip is TRUE and XtNalignment is XwALIGN\_LEFT, leading spaces are stripped from each line.  If XtNstrip is TRUE and XtNalignment is XwALIGN\_CENTER, both leading and trailing spaces are stripped from each line.  If XtNstrip is TRUE and XtNalignment is XwALIGN\_RIGHT, trailing spaces are stripped from each line.}
\resource{XtNtitleBackground}{The value to be loaded into the XtNbackground resource of the optional StaticText widget's core part.}
\resource{XtNtitleBorderWidth}{The value to loaded into the XtNborderWidth resource of the optional StaticText widget.}
\resource{XtNtitleForeground}{The value to be loaded into the XtNforeground resource of the optional StaticText widget's core part.}
\resource{XtNtitleHSpace}{The value to be loaded into the XtNhSpace resource of the optional StaticText widget.}
\resource{XtNtitleImage}{If the title type resource indicates that a title image should be displayed, then this resource will contain a pointer to an XImage structure; this structure describes the title image data.}
\resource{XtNtitleLPadding}{The value to be loaded into the XtNtitleLPadding constraint resource of the optional StaticText widget.}
\resource{XtNtitlePosition}{The value to be loaded into the XtNtitlePosition constraint resource of the optional StaticText widget.}
\resource{XtNtitlePosition}{This resource is used to control where the title is displayed within the cascading menupane.  To programmatically set this resource, use either the XwTOP, XwBOTTOM or XwBOTH define.  To set this resource using the .Xdefaults file, use one of the strings top, bottom or both.}
\resource{XtNtitlePrecedence}{The value to be loaded into the constraint record of the optional StaticText widget.}
\resource{XtNtitleRPadding}{The value to be loaded into the XtNtitleRPadding constraint resource of the optional StaticText widget.}
\resource{XtNtitleRegion}{The value to be loaded into the XtNtitleRegion constraint resource of the optional StaticText widget.}
\resource{XtNtitleRelease}{The value loaded into the XtNrelease resource of the optional StaticText widget.}
\resource{XtNtitleSelect}{The value loaded into the XtNselect resource of the optional StaticText widget.}
\resource{XtNtitleShowing}{This resource may be used by the application to control the displaying of a title within the menupane.  This may be overridden, however, by a menu manager using the XtNmgrTitleOverride resource.}
\resource{XtNtitleString}{If the title type resource indicates that a title string should be displayed, then this resource will contain the title string which is to be used.  In the case where the application does not specify a title string, the name of the menupane widget will be used. The title is displayed using the foreground color.}
\resource{XtNtitleToMenuPad}{If both a title and a menu child are being displayed, the padding between them in pixels.}
\resource{XtNtitleType}{Two styles of titles are supported by the MenuPane widget.  They include text string titles and image titles.  To programmatically set this resource, use either the XwSTRING define or the XwIMAGE define.  To set this resource using the .Xdefaults file, use one of the strings string or image.}
\resource{XtNtitleVSpace}{The value to be loaded into the XtNvSpace resource of the optional StaticText widget.}
\resource{XtNtoggle}{If set to TRUE makes the pushbutton act like a toggle button.}
\resource{XtNtopLevel}{Indicates whether not the panel is a candidate for management by a window manager.  This should always be set by the application.}
\resource{XtNtopMargin}{The number of pixels used for the top margin.}
\resource{XtNtranslations}{The set of default translations are described below.}
\resource{XtNtranslations}{This is a pointer to a translations list.}
\resource{XtNtraversalOn}{The application can define whether keyboard traversal is active or not.  The default for this resource is typically FALSE.}
\resource{XtNtraversalType}{Three modes of border highlighting activation are supported by Primitive.  They are, no highlighting, highlight on the cursor entering the widgets window and highlight for keyboard traversal.  The last mode is used by the keyboard traversal mechanism to indicate the widget that is to receive all input occurring within the widget hierarchy.  To set the traversal type through an arg list, one of three defines can be used. They are XwHIGHLIGHT\_OFF, XwHIGHLIGHT\_ENTER and XwHIGHLIGHT\_TRAVERSAL.  The strings that can be used to set this resource through the .Xdefaults file are highlight\_off, highlight\_enter, and highlight\_traversal.}
\resource{XtNvSpace}{Padding between the top of the Panel and the top child in pixels, and between the bottom of the Panel and the bottom child in pixels.}
\resource{XtNvSpace}{The amount of space to maintain between the top and bottom of the titlebar and the interior widgets.}
\resource{XtNvSpace}{The application may determine the number of pixels of space left between each column.  This defines a minimum spacing.}
\resource{XtNvSpace}{The application may determine the number of pixels of space left between the top of the button and the top of the button label, and between the bottom of the label and the bottom of the button.}
\resource{XtNvSpace}{This specifies the number of pixels to maintain between the text and the highlight area to the top and bottom of the text.}
\resource{XtNwidgetType}{Indicates to Panel what type of child it is.  The possible values are, XwWORK\_SPACE, specified in a resource defaults file as "work space", XwTITLE, specified in a resource defaults file as "title", and XwPULLDOWN, specified in a resource defaults file as "pulldown".}
\resource{XtNwidth}{This argument contains the width of the widget's window in pixels, not including the border area. Programs should not change this argument directly, but use geometry manager requests instead in order to ensure proper relationships with other widgets are maintained.}
\resource{XtNworkSpaceToSiblingPad}{The padding between the work space child and the sibling above it.  If there is no title nor menu being displayed this resource is ignored.}
\resource{XtNwrapBreak}{This resource specifies how the wrap position is determined.  When set to XwWrapAny, the wrap will happen at the character position closest to the right margin.  When set to XwWrapWhiteSpace, the wrap will happen at the  last whitespace before the right margin. If the line does not have whitespace, it will be wrapped as XwWrapAny.}
\resource{XtNwrap}{This resource controls the wrapping of lines within the v               widget. If XtNwrap is TRUE, lines which are too long are broken on spaces.  The spaces are converted to new-lines to break the line. Imbedded new-lines are honored.  If there is too much text for the specified window size, it will be clipped at the bottom.\nl\nl If XtNwrap is FALSE, lines which are too long will be clipped according to the alignment.  An XtNalignment value of XwALIGN\_LEFT will cause lines which are too long to be clipped to the right.  An XtNalignment value of XwALIGN\_RIGHT will cause lines which are too long to be clipped to the left.  An XtNalignment value of XwALIGN\_CENTER will cause lines to be clipped equally on both the right and the left.}
\resource{XtNwrap}{This resource specifies how the widget displays lines longer than the screen width.  When set to XwWrapOff, the lines may extend off screen to the right.  When set to XwSoftWrap, the lines will be wrapped at the right margin with the actual position determined by the resource XtNwrapBreak.}
\resource{XtNxAddWidth XtNyAddHeight}{This resource indicates whether or not to add the width or height of the reference widget to a widget's location when determining the widget's position.}
\resource{XtNxAttachOffset XtNyAttachOffset}{When a widget is attached to the right or bottom edge of the form (through the above resources), the separation between the widget and the form is defaulted to 0 pixels.  This resource allows that separation to be set to some other value.  Also, for widgets that are not attached to the right or bottom edge of the form, this constraint specifies the minimum spacing between the widget and the form.}
\resource{XtNxAttachRight XtNyAttachBottom}{Widgets are normally referenced from "form left" to "form right" or from "form top" to "form bottom."  The attach resources allow this reference to occur on the opposite edge of the form.  These resources, when used in conjunction with the varyOffset resources, allow a widget to float along the right or bottom edge of the form.  This is done by setting both the Attach and VaryOffset resources to TRUE.  A widget can also span the width and height of the form by setting the Attach resource to TRUE and the VaryOffset resource to FALSE.}
\resource{XtNxOffset XtNyOffset}{The location of a widget is determined by the widget it references.  As the default, a widget's position on the form exactly matches its reference widget's location. There are two additional pieces of data used to determine the location.  This resource defines an integer value representing the number of pixels to add to the reference widget's location when calculating the widget's location.}
\resource{XtNxRefName XtNyRefName}{When a widget is added as a child of the form its position is determined by the widget it references. The reference widget must be created before the widget which references it is created.  These resources allow the name of the reference widget to be given.  The form converts this name to a widget to use for the referencing.  Any widget that is a direct child of the form or the form widget itself can be used as a reference widget.}
\resource{XtNxRefWidget XtNyRefWidget}{The application can specify the reference widget as either a string representing the name of the widget (as described above) or as the Widget ID value returned from XtCreateWidget.  This resource is the means by which a widget ID is specified.}
\resource{XtNxResizable XtNyResizable}{This resource specifies whether the form can resize (shrink) a widget.  When a form's size becomes smaller the form will resize its children only after all of the inter-widget spacing of widget's with their VaryOffset resource set to TRUE.  The form keeps track of a widgets initial size or size generated through XtSetValues so that when the form then becomes larger the widget will grow to it original size and no larger.}
\resource{XtNxVaryOffset XtNyVaryOffset}{When a form is resized, it processes the constraints contained within its children.  This resource allows the spacing between a widget and the widget it references to vary (either increase or decrease) when a form's size changes.  For widgets that directly reference the form widget this resource is ignored. The spacing between a widget and its reference widget can decrease to 0 pixels if the XtNAddWidth resource is FALSE or to 1 pixel if XtNAddWidth is TRUE.}
\resource{XtNx}{This argument contains the x-coordinate of the widget's upper left hand corner (excluding the border) in relation to its parent widget.  Programs should not change this argument directly, but use geometry manager requests instead in order to ensure proper relationships with other widgets are maintained.}
\resource{XtNy}{This argument contains the y-coordinate of the widget's upper left hand corner (excluding the border) in relation to its parent widget.  Programs should not change this argument directly, but use geometry manager requests instead in order to ensure proper relationships with other widgets are maintained.}
\bye