\documentstyle[12pt,german]{article}
% neue Version April 1990 (German 2.3, TeX 2, LaTeX 2.09)

\parindent=0pt
\parskip=6pt plus 3pt
\sloppy

\begin{document}

\section*{Einheitliche deutsche \TeX-Befehle}

Beim 6.~Treffen der deutschen \TeX-""Interessenten in M"unster (Oktober 1987)
wurde Einigung "uber ein
"`Minimal Subset von einheitlichen deutschen \TeX-Befehlen"'
erzielt, das seitdem an allen Installationen von \TeX\ und \LaTeX\
zur Verf"ugung stehen und
f"ur deutschsprachige Texte verwendet werden soll.
Damit wird erreicht, da"s alle \TeX- und \LaTeX-Dokumente,
die diese Befehle enthalten,
problemlos von einem Rechner zum anderen "ubertragen werden k"onnen.

Der in M"unster festgelegte Befehlssatz umfa"st die folgenden Befehle:
\begin{itemize}

\item
\verb|"a| als Abk"urzung f"ur \verb|\"a| (Umlaute wie~"a)
-- auch f"ur alle anderen Vokale,

\item
\verb|"s| als Abk"urzung f"ur \verb|\ss| (scharfes~s: "s),

\item
\verb|"ck|  f"ur "`ck"', das als "`k-k"' abgeteilt wird,

\item
\verb|"ff| f"ur "`"ff"', das als "`ff-f"' abgeteilt wird
-- auch f"ur die anderen relevanten Konsonanten,

\item
\verb|"`| oder \verb|\glqq| f"ur untere und
\verb|"'| oder \verb|\grqq| f"ur obere "`deutsche Anf"uhrungszeichen"'
(\glqq G"ansef"u"schen\grqq),

\item
\verb|\glq| f"ur untere und \verb|\grq| f"ur obere
\glq einfache Anf"uhrungszeichen\grq ,

\item
\verb|"<| oder \verb|\flqq| f"ur linke und
\verb|">| oder \verb|\frqq| f"ur rechte "<franz"osische
Anf"uh\-rungs\-zei\-chen">
(\flqq guillemets\frqq),

\item
\verb|\flq| f"ur linke und
\verb|\frq| f"ur rechte \flq einfache franz"osische
Anf"uh\-rungs\-zei\-chen\frq,

\item
\verb."|. f"ur die Trennung von Ligaturen,

\item
\verb|"-| f"ur eine Silbentrennstelle "ahnlich wie bei \verb|\-|,
bei der aber die automatische Silbentrennung im Rest des Wortes
erhalten bleibt,

\item
\verb|""| f"ur eine analoge Trennstelle, bei der aber im Fall der Trennung
kein Trennstrich hinzugef"ugt wird,     %#em statt Minuszeichen

\item
\verb|\dq| f"ur das Ausdrucken des Quote-Zeichens,

\goodbreak
\item
\verb|\selectlanguage{|{\it n\/}\verb|}|
f"ur das Umschalten zwischen
deutschen, "osterreichischen, englischen, amerikanischen und franz"o"-sischen
Datums"-angaben und "Uberschriften.
F"ur~{\it n\/} ist dabei einer der folgenden Befehls"-namen zu verwenden:
\begin{quote}
\verb|\german|   \\
\verb|\austrian| \\
\verb|\english|  \\
\verb|\USenglish| \\
\verb|\french|
\end{quote}

\item
\verb|\originalTeX|
f"ur das Zur"uckschalten auf Original-\TeX\ bzw.~-\LaTeX.

\item
\verb|\germanTeX|
f"ur das Wiedereinschalten der deutschen \TeX-Befehle.

\end{itemize}

\iffalse % --- jetzt nicht mehr erwaehnen, obwohl es noch richtig ist! ---
F"ur bereits existierende Anwendungen, die \verb|\3| f"ur scharfes~s
verwenden, wird auch dieser Befehl noch weiterhin unterst"utzt.
F"ur neue Anwendungen sollte aber nur mehr \verb|"s| verwendet werden,
da der Befehl \verb|\3| in manchen Macro-Paketen (z.B.~WEBMAC) bereits
f"ur andere Zwecke verwendet wird.
\fi

Die Befehle f"ur Umlaute und scharfes~s
und/oder die Hyphenation Patterns sind so definiert,
da"s auch in Silben {\em nach\/} einem Umlaut oder scharfen~s
die automatische Silbentrennung funktioniert.

\pagebreak[2]

Tab.~1 und 2 enthalten Beispiele f"ur die Verwendung dieser Befehle.

\vspace{0pt plus 1cm} \pagebreak[2]

Diese Definitionen werden mit dem \TeX-Befehl
\begin{quote}
\verb|\input german.sty|                %#em: .sty
\end{quote}
bzw.~bei \LaTeX\  mit einem \verb|\documentstyle|-Befehl wie z.B.
\begin{quote}
\verb|\documentstyle[11pt,german]{article}|
\end{quote}
in das Eingabefile eingef"ugt.

\pagebreak[2]

%--- {\em Hier geh"ort ein Hinweis, wie die Files
%\hbox{\tt GERMAN.TEX} und/oder \hbox{\tt GERMAN.STY}
%in Ihrem Rechen\-zentrum installiert sind und wo das dokumentiert ist
%(Local Guide, \LaTeX-""Kurzbeschreibung etc.)}  ---
%Wir empfehlen auch allen Benutzern, die \TeX\ an einem
%Personal Computer oder Institutsrechner installiert haben,
%diese Files an ihren Rechner zu "ubertragen
%und dort ebenfalls diese Konventionen zu verwenden.

%#em
Die Datei \hbox{\tt GERMAN.STY}
befindet sich im Verzeichnis \verb|\emtex\texinput|.
F"ur \LaTeX werden modifizierte STY-Dateien ben"otigt. Diese befinden
sich im Verzeichnis \verb|\emtex\texinput\german|. Da sich
die Originalversionen im Verzeichnis \verb|\emtex\texinput|
befinden, sollte die Environment-Variable \mbox{TEXINPUT} auf
\begin{quote}
\verb|\emtex\texinput\german;\emtex\texinput|
\end{quote}
gesetzt werden (eventuell sind noch Laufwerksbezeichnungen
einzuf"ugen), damit die modifizierten Versionen zuerst gefunden
werden.


Die Realisierung der deutschen \TeX-Befehle mit diesem
von Dr.~Partl an der Technischen Universit"at Wien zusammengestellten
und vom DANTE-Verein am Listserver in Heidelberg zur Verf"ugung
gestellten
File \hbox{\tt GERMAN.STY} ist vor allem als "`rasche           %#em .TEX
L"osung"' zu betrachten, die den Vorteil hat, da"s sie keine
"Anderungen an der \TeX-Software, den Font-Files und den Hyphenation
Patterns erfordert sondern direkt auf die Originalversion von \TeX\
aufgesetzt werden kann.

Die hier beschriebene Version von \hbox{\tt GERMAN.STY}         %#em .TEX
(Version~2.3) ist f"ur \TeX\ Version 2.{\it x\/}
und \LaTeX\ Version 2.09 konzipiert.

F"ur die Zukunft ist eine neue Version geplant, die die neuen
Features von \TeX\ Version~3 und die neuen \LaTeX-Versionen
(2.10 und 3) unter"-st"utzen wird.

\begin{table}[hp]
\caption{Beispiele}

\begin{center}
\begin{tabbing}
{\tt xxxMerci bienxxx}\qquad \= \kill

\verb|sch"on| \> ergibt:\qquad
        sch"on \\[3pt]
\verb|Stra"se| \> ergibt:\qquad
        Stra"se \\[3pt]
\verb|"`Ja, bitte!"'| \> ergibt:\qquad
        "`Ja, bitte!"' \\[3pt]
\verb|"<Merci bien!">| \> ergibt:\qquad
        "<Merci bien!"> \\[3pt]
\verb|Dru"cker | \> ergibt:\qquad
        Drucker bzw. Druk-ker \\[3pt]
\verb|Ro"lladen| \> ergibt:\qquad
        Rolladen bzw. Roll-laden \\[3pt]
\verb.Auf"|lage. \> ergibt:\qquad
        Auf"|lage
\end{tabbing}
\end{center}
\end{table}

\begin{table}[hp]
\day=31 \month=1 \year=1990
\caption{Datumsangaben}

\begin{center}
\begin{tabbing}
{\tt xselectlanguagexxxx}\qquad \= \kill

\verb|\selectlanguage| \> \verb|\today | \\[9pt]
\verb|\german| \>         \selectlanguage\german    \today  \\[3pt]
\verb|\austrian| \>       \selectlanguage\austrian  \today  \\[3pt]
\verb|\english| \>        \selectlanguage\english   \today  \\[3pt]
\verb|\USenglish| \>      \selectlanguage\USenglish \today  \\[3pt]
\verb|\french| \>         \selectlanguage\french    \today

\end{tabbing}
\end{center}
\end{table}

\end{document}
