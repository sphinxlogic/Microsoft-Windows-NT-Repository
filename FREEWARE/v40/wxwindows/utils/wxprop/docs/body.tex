\chapter{Introduction}\label{introduction}
\pagenumbering{arabic}%
\setheader{{\it CHAPTER \thechapter}}{}{}{}{}{{\it CHAPTER \thechapter}}%
\setfooter{\thepage}{}{}{}{}{\thepage}%

The Property Sheet Classes help the programmer to specify complex dialogs and
their relationship with their associated data. By specifying data as a
wxPropertySheet containing wxProperty objects, the programmer can use
a range of available or custom wxPropertyView classes to allow the user to
edit this data. Classes derived from wxPropertyView act as mediators between the
wxPropertySheet and the actual window (and associated panel items).

For example, the wxPropertyListView is a kind of wxPropertyView which displays
data in a Visual Basic-style property list (see \helpref{the next section}{appearance} for
screen shots). This is a listbox containing names and values, with
an edit control and other optional controls via which the user edits the selected
data item.

wxPropertyFormView is another kind of wxPropertyView which mediates between
the data and a panel or dialog box which has already been created. This makes it a contender for
the replacement of wxForm, since programmer-controlled layout is going to be much more
satisfactory. If automatic layout is desired, then wxPropertyListView could be used instead.

The main intention of this class library was to provide property {\it list} behaviour, but
it has been generalised as much as possible so that the concept of a property sheet and its viewers
can reduce programming effort in a range of user interface tasks.

For further details on the classes and how they are used, please see \helpref{Property classes overview}{propertyoverview}.

\section{The appearance and behaviour of a property list view}\label{appearance}

The property list, as seen in an increasing number of development tools
such as Visual Basic and Delphi, is a convenient and compact method for
displaying and editing a number of items without the need for one
control per item, and without the need for designing a special form. The
controls are as follows:

\begin{itemize}\itemsep=0pt
\item A listbox showing the properties and their current values, which has double-click
properties dependent on the nature of the current property;
\item a text editing area at the top of the display, allowing the user to edit
the currently selected property if appropriate;
\item `confirm' and `cancel' buttons to confirm or cancel an edit (for the property, not the
whole sheet);
\item an optional list that appears when the user can make a choice from several known possible values;
\item a small Edit button to invoke `detailed editing' (perhaps showing or hiding the above value list, or
maybe invoking a common dialog);
\item optional OK/Close, Cancel and Help buttons for the whole dialog.
\end{itemize}

The concept of `detailed editing' versus quick editing gives the user a choice
of editing mode, so novice and expert behaviour can be catered for, or the user can just
use what he feels comfortable with.

Behaviour alters depending on the kind of property being edited. For example, a boolean value has
the following behaviour:

\begin{itemize}\itemsep=0pt
\item Double-clicking on the item toggles between TRUE and FALSE.
\item Showing the value list enables the user to select TRUE or FALSE.
\item The user may be able to type in the word TRUE or FALSE, or the edit control
may be read-only to disallow this since it is error-prone.
\end{itemize}

A list of strings may pop up a dialog for editing them, a simple string just allows text editing,
double-clicking a colour property may show a colour selector, double-clicking on a filename property may
show a file selector (in addition to being able to type in the name in the edit control), etc.

Note that the `type' of property, such as string or integer, does not
necessarily determine the behaviour of the property. The programmer has
to be able to specify different behaviours for the same type, depending
on the meaning of the property. For example, a colour and a filename may
both be strings, but their editing behaviour should be different. This
is why objects of type wxPropertyValidator need to be used, to define
behaviour for a given class of properties or even specific property
name.  Objects of class wxPropertyView contain a list of property
registries, which enable reuse of bunches of these validators in
different circumstances. Or a wxProperty can be explicitly set to use a
particular validator object. 

The following screen shot of the property classes test program shows the
user editing a string, which is constrained to be one of three possible
values.

$$\image{8cm;0cm}{prop1.eps}$$\\

The second picture shows the user having entered a integer that
was outside the range specified to the validator. Note that in this picture,
the value list is hidden because it is not used when editing an integer.

$$\image{8cm;0cm}{prop2.eps}$$

\chapter{Files}\label{files}
\setheader{{\it CHAPTER \thechapter}}{}{}{}{}{{\it CHAPTER \thechapter}}%
\setfooter{\thepage}{}{}{}{}{\thepage}%

The property class library comprises the following files:

\begin{itemize}\itemsep=0pt
\item wx\_prop.h: base property class header
\item wx\_plist.h: wxPropertyListView and associated classes
\item wx\_pform.h: wxPropertyListView and associated classes
\item wx\_prop.cc: base property class implementation
\item wx\_plist.cc: wxPropertyListView and associated class implementions
\item wx\_pform.cc: wxPropertyFormView and associated class implementions
\end{itemize}
