\documentstyle[a4,makeidx,verbatim,texhelp,fancyhea,mysober,mytitle]{report}%
\input psbox.tex
\parskip=10pt%
\title{Manual for wxTreeLayout 1.0: a tree layout library for wxWindows}
\author{Julian Smart\\Artificial Intelligence Applications Institute\\
University of Edinburgh\\EH1 1HN}
\date{November 1993}%
\makeindex%
\begin{document}%
\maketitle

\pagestyle{fancyplain}
\bibliographystyle{plain}
\pagenumbering{roman}
\setheader{{\it CONTENTS}}{}{}{}{}{{\it CONTENTS}}
\setfooter{\thepage}{}{}{}{}{\thepage}
\tableofcontents%

\chapter*{Copyright notice}
\setheader{{\it COPYRIGHT}}{}{}{}{}{{\it COPYRIGHT}}%
\setfooter{\thepage}{}{}{}{}{\thepage}

Copyright (c) 1993 Artificial Intelligence Applications Institute,
The University of Edinburgh.

Permission to use, copy, modify, and distribute this software and its
documentation for any purpose is hereby granted without fee, provided that the
above copyright notice, author statement and this permission notice appear in
all copies of this software and related documentation.

THE SOFTWARE IS PROVIDED ``AS-IS'' AND WITHOUT WARRANTY OF ANY KIND, EXPRESS,
IMPLIED OR OTHERWISE, INCLUDING WITHOUT LIMITATION, ANY WARRANTY OF
MERCHANTABILITY OR FITNESS FOR A PARTICULAR PURPOSE.

IN NO EVENT SHALL THE ARTIFICIAL INTELLIGENCE APPLICATIONS INSTITUTE OR THE
UNIVERSITY OF EDINBURGH BE LIABLE FOR ANY SPECIAL, INCIDENTAL, INDIRECT OR
CONSEQUENTIAL DAMAGES OF ANY KIND, OR ANY DAMAGES WHATSOEVER RESULTING FROM
LOSS OF USE, DATA OR PROFITS, WHETHER OR NOT ADVISED OF THE POSSIBILITY OF
DAMAGE, AND ON ANY THEORY OF LIABILITY, ARISING OUT OF OR IN CONNECTION WITH
THE USE OR PERFORMANCE OF THIS SOFTWARE.

\chapter{Introduction}
\pagenumbering{arabic}%
\setheader{{\it CHAPTER \thechapter}}{}{}{}{}{{\it CHAPTER \thechapter}}%
\setfooter{\thepage}{}{}{}{}{\thepage}

This manual describes a tree-drawing class library for wxWindows. It
provides layout of simple trees with one root node, drawn left-to-right,
with user-defined spacing between nodes.

wxTreeLayout is an abstract class that must be subclassed. The programmer
defines various member functions which will access whatever data structures
are appropriate for the application, and wxTreeLayout uses these when laying
out the tree.

wxStoredTree is a class derived from wxTreeLayout that may be used directly to
draw trees on a canvas. It supplies storage for the nodes, and draws
to a device context.

\helponly{Below is the example tree generated by the program test.cc.

\begin{figure}
$$\image{11cm;0cm}{treetst.ps}$$
\caption{Example tree}\label{exampletree}
\end{figure}
}

\chapter{Implementation}
\setheader{{\it CHAPTER \thechapter}}{}{}{}{}{{\it CHAPTER \thechapter}}%
\setfooter{\thepage}{}{}{}{}{\thepage}

The algorithm is due to Gabriel Robins \cite{robins87}, a linear-time
algorithm originally implemented in LISP for AI applications.

The original algorithm has been modified so that both X and Y planes
are calculated simultaneously, increasing efficiency slightly. The basic
code is only a page or so long.

\input classes.tex
%
\bibliography{tree}

\helpignore{\addcontentsline{toc}{chapter}{Index}
\printindex}
\end{document}
