\chapter{Introduction}\label{introduction}
\pagenumbering{arabic}%
\setheader{{\it CHAPTER \thechapter}}{}{}{}{}{{\it CHAPTER \thechapter}}%
\setfooter{\thepage}{}{}{}{}{\thepage}%

The wxWindows Dialog Editor is a tool for creating dialog resource files, in {\tt .wxr} format.
It differs from wxBuilder in the following respectes:

\begin{enumerate}\itemsep=0pt
\item Scope. It is written for dialog editing only, and is therefore more convenient than wxBuilder for this purpose.
\item File format. Dialog editor reads and writes wxWindows resource files (extension {\tt .wxr}) and has
no independent file format.
\item Robustness. It is written in a more principled way than wxBuilder, and is less ambitious.
\item Ease of use. Windows are edited using the mouse or via consistent {\it property editors}, which
provide immediate visual feedback of changed properties.
\end{enumerate}

Dialog Editor should be compiled and used with wxWindows 1.65 or later.

\section{Current status}

Dialog Editor currently runs under MS Windows. It has yet to
be tested under Motif and XView: see \helpref{Technical notes}{technotes} for a discussion
of what needs to be done.

\section{Future developments}

\begin{itemize}\itemsep=0pt
\item XView and Motif versions.
\item It would be nice to have a dialog browser, showing thumbnails of
all dialogs in a particular directory.
\item Maybe add a menubar editor (from wxBuilder).
\item Maybe convert Windows .rc files.
\end{itemize}

\chapter{Commands}\label{commands}
\setheader{{\it CHAPTER \thechapter}}{}{}{}{}{{\it CHAPTER \thechapter}}%
\setfooter{\thepage}{}{}{}{}{\thepage}%

\section{Dialog editor menu bar}

\subsection{File menu}

\begin{twocollist}\itemsep=0pt
\twocolitem{New project}{Creates a new project (clears index and resets project name).}
\twocolitem{New dialog}{Creates a new dialog resource.}
\twocolitem{Open}{Opens an existing resource file.}
\twocolitem{Save}{Saves the current resources.}
\twocolitem{Save as}{Saves the current resources in a named file.}
\twocolitem{Clear}{Clears the current resources.}
\twocolitem{Exit}{Exit the program.}
\end{twocollist}

\subsection{Edit menu}

\begin{twocollist}\itemsep=0pt
\twocolitem{Recreate}{Recreates the currently selected panel item from the underlying resource. This may be necessary
to regenerate items that cannot be changed dynamically, and which have got out of sync with the displayed
item.}
\twocolitem{Delete}{Deletes the currently selected resource.}
\twocolitem{Toggle edit/test mode}{Toggles from edit to test mode, and vice versa. Edit mode is used for editing
dialogs, test mode allows panel items to be manipulated as they will appear to the user.}
\end{twocollist}

\subsection{Help menu}

\begin{twocollist}\itemsep=0pt
\twocolitem{Help topics}{Displays on-line help at the contents page.}
\twocolitem{About}{Displays an dialog showing the Dialog Editor version and author.}
\end{twocollist}

\latexonly{\newpage}
\section{Command toolbar}

The command toolbar consists of the following tools:

\begin{twocollist}%\itemsep=0pt
\twocolitem{\icon{new.eps}{New}}{Clears the project.}
\twocolitem{\icon{open.eps}{Open}}{Opens an existing resource file.}
\twocolitem{\icon{save.eps}{Save}}{Saves the current resources.}
\twocolitem{\icon{vert.eps}{Horizontal align}}{Aligns the centre of the selected items horizontally.}
\twocolitem{\icon{alignt.eps}{Horizontal top-align}}{Aligns the top sides of the selected items horizontally.}
\twocolitem{\icon{alignb.eps}{Horizontal bottom-align}}{Aligns the bottom sides of the selected items horizontally.}
\twocolitem{\icon{horiz.eps}{Vertical align}}{Aligns the centre of the selected items vertically.}
\twocolitem{\icon{alignl.eps}{Vertical left-align}}{Aligns the left sides of the selected items vertically.}
\twocolitem{\icon{alignr.eps}{Vertical right-align}}{Aligns the right sides of the selected items vertically.}
\twocolitem{\icon{copysize.eps}{Copy size}}{Copies the size of the first selected item to the subsequently selected item(s).}
\twocolitem{\icon{tofront.eps}{To front}}{Puts the selected item(s) to the front of the display list.}
\twocolitem{\icon{toback.eps}{To back}}{Puts the selected item(s) to the back of the display list.}
\twocolitem{\icon{help.eps}{Help}}{Invokes Dialog Editor help.}
\end{twocollist}

\latexonly{\newpage}
\section{Tool palette}

The tool palette is used to select a type of panel item to create on the dialog.
To create a new panel item, select a tool with left-click, then left-click on the dialog to 
window. Select the pointer tool to use left-click for selecting and deselecting
items.

The tool palette consists of the following tools.

\begin{twocollist}%\itemsep=0pt
\twocolitem{\icon{arrow.eps}{Pointer}}{Click this to select/deselect items on a dialog.}
\twocolitem{\icon{message.eps}{Message}}{Text message item.}
\twocolitem{\icon{picture.eps}{Bitmap message}}{Bitmap message item.}
\twocolitem{\icon{button.eps}{Button}}{Text button item.}
\twocolitem{\icon{bmpbuttn.eps}{Bitmap button}}{Bitmap button item.}
\twocolitem{\icon{check.eps}{Checkbox}}{Checkbox item.}
\twocolitem{\icon{radio.eps}{Radiobox}}{Radiobox item.}
\twocolitem{\icon{listbox.eps}{Listbox}}{Listbox item.}
\twocolitem{\icon{choice.eps}{Choice}}{Choice item.}
\twocolitem{\icon{text.eps}{Text}}{Single-line text item.}
\twocolitem{\icon{mtext.eps}{Multitext}}{Multi-line text item.}
\twocolitem{\icon{slider.eps}{Slider}}{Slider item.}
\twocolitem{\icon{group.eps}{Groupbpx}}{Groupbox item.}
\twocolitem{\icon{gauge.eps}{Gauge}}{Gauge item.}
\twocolitem{\icon{scroll.eps}{Scrollbar}}{Scrollbar item.}
\end{twocollist}

\section{Resource list}

The resource list shows a list of the dialogs, panel items and bitmaps currently loaded
in Dialog Editor. Double-clicking on a dialog item shows the associated dialog box.

\chapter{Procedures}\label{procedures}
\setheader{{\it CHAPTER \thechapter}}{}{}{}{}{{\it CHAPTER \thechapter}}%
\setfooter{\thepage}{}{}{}{}{\thepage}%

\section{Running Dialog Editor}

To run Dialog Editor under Windows, click on the Program Manager or Explorer icon.
Under UNIX, run from the command line.

The main window shows a menu bar, command toolbar, tool palette, resource list, and
status line.

\section{Creating a dialog}

To create a new dialog, click on the {\bf File: New} menu item, or equivalent
toolbar button. A dialog will appear. To put a panel item on the dialog, left-click
on the appropriate palette icon and then left-click on the dialog. A new item
will appear at the place you clicked.

You can edit any panel item or dialog by control-left clicking. A property editor
will appear, allowing any property to be selected and edited (see \helpref{Using property editors}{propeditors}).
You can also edit items by right-clicking to show a menu, and then selecting {\it Edit properties}.

To move a panel item, drag the item with the left mouse button, or edit
the position values in the property editor. To resize a panel item, you
can either select it by left-clicking and then dragging on a selection
handle, or edit the size values in the property editor. 

You can delete items from the right-click menu, or by selecting the item and
choosing {\bf Edit: Delete} from the menu bar.

\section{Using property editors}\label{propeditors}

Property editors consist of a list of properties and current values, plus controls at the top of
the editor. If the property is of an appropriate type, you can edit the value directly in the
text field, and confirm or cancel the value using the two buttons to the left of it.
If the property has a predefined range of values, such as labelFontFamily, you can
see a list of permissable values by clicking on the button labelled with an ellipsis symbol ({\bf ...}).
This will show a listbox with possible values and current selection. You may also be able
to cycle through values by double-clicking the value in the listbox.

Properties may have special editors appropriate to the type. Filename properties invoke
the file selector, and properties containing list of user-definable strings use a
string editor.

When you change a property value, this value is immediately reflected in
the dialog or panel item.  If the item allows this value to be changed
dynamically, the relevant wxWindows function will be called internally
to effect the change.  If the value cannot be changed dynamically, the
item will be destroyed and re-created, which means that there will be
more flickering associated with some kinds of property changes than
others.

\section{Saving and loading files}

Use {\it File: Save} and {\it File: Save as} or the equivalent toolbar button
to save the current dialog(s) in a wxWindows resource file (extension {\tt .wxr}).

The {\tt .wxr} file can be used directly in a wxWindows program, if
wxWindows resources have been enabled when building the wxWindows library.
These files can be loaded dynamically, or included directly into program source
with a \verb$#include$ directive. See the wxWindows user manual for further details.

\section{Multi-platform development}

{\tt .wxr} files generated on one environment (e.g. Windows) can be used in another (e.g. Motif).
However, because the same panel item can have different sizes on different GUIs,
the user should be cautious in assuming that one resource file will work for all
platforms. It may be better to plan to conditionally include or load different
resource files for different platforms, with spacing modified to suit each
environment.

