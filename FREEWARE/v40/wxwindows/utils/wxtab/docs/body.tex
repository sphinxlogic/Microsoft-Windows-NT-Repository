\chapter{Introduction}\label{introduction}
\pagenumbering{arabic}%
\setheader{{\it CHAPTER \thechapter}}{}{}{}{}{{\it CHAPTER \thechapter}}%
\setfooter{\thepage}{}{}{}{}{\thepage}%

The tab classes provides a way to display rows of tabs (like file divider tabs), which can be
used to switch between panels or other information. Tabs are most
commonly used in dialog boxes where the number of options is too great
to fit on one dialog.

This code compiles and runs on MS Windows and Motif, but limitations of
XView (panel refresh, no nested subwindows) prevent it from being used
on that platform.

\section{The appearance and behaviour of a wxTabbedDialogBox}\label{appearance}

The following screenshot shows the appearance of the sample tabbed dialog application.

$$\image{8cm;0cm}{wxtab1.eps}$$

By clicking on the tabs, the user can display a different set of controls. In the example,
the Close and Help buttons remain constant. These two buttons are children of the main dialog box,
whereas the other controls are children of panels which are shown and hidden according to
which tab is active.

A tabbed dialog may have several layers (rows) of tabs, each being
offset vertically and horizontally from the previous. Tabs work in
columns, in that when a tab is pressed, it swaps place with the tab on
the first row of the same column, in order to give the effect of
displaying that tab. All tabs must be of the same width.
This is a constraint of the implementation, but it also
means that the user will find it easier to find tabs since there are
distinct tab columns. On some tabbed dialog implementations, tabs jump around
seemingly randomly because tabs have different widths.
In this implementation, a tab can always be found on the same column.

Tabs are always drawn along the top of the view area; the implementation does
not allow for vertical tabs or any other configuration.

\chapter{Files}\label{files}
\setheader{{\it CHAPTER \thechapter}}{}{}{}{}{{\it CHAPTER \thechapter}}%
\setfooter{\thepage}{}{}{}{}{\thepage}%

The class library comprises the following files, plus makefiles:

\begin{itemize}\itemsep=0pt
\item wxtab.cpp: implementation
\item wxtab.h: header
\item test.h: test application header
\item test.cpp: test application implementation
\end{itemize}

