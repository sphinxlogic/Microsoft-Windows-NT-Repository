\documentstyle[a4,makeidx,verbatim,texhelp,fancyhea,mysober,mytitle]{report}
\parindent 0pt
\parskip 11pt
\title{Manual for PROLOGIO: a Prolog-compatible I/O and RPC library for C++}
\author{Julian Smart\\Decision Support Group\\Artificial Intelligence Applications Institute\\80 South Bridge\\University of Edinburgh\\EH1 1HN}
\date{April 1995}
%
\makeindex
\begin{document}
\maketitle
%
\pagestyle{fancyplain}
\bibliographystyle{plain}
\pagenumbering{roman}
\setheader{{\it CONTENTS}}{}{}{}{}{{\it CONTENTS}}
\setfooter{\thepage}{}{}{}{}{\thepage}
\tableofcontents%
\newpage%
\pagenumbering{arabic}%

\chapter*{Summary}
\setheader{{\it SUMMARY}}{}{}{}{}{{\it SUMMARY}}%
\setfooter{\thepage}{}{}{}{}{\thepage}

PROLOGIO is a C++ library for performing two main functions:

\begin{enumerate}
\item reading and writing a subset of Prolog syntax
\item implementing a remote procedure call facility (RPC) for
client-server communication
\end{enumerate}

This document describes the PROLOGIO API (Application Programming
Interface). It can be used to develop programs with readable, robust and
Prolog-compatible data files, as well as making RPC communication simple
to program on UNIX and PCs. The library requires the wxWindows library
for XView and Windows 3.

The Prolog I/O and RPC facilities are combined in the one package since
they both relate to reading and writing complex structures to and from
ASCII strings or files, and use the same YACC and LEX-based parser to do
so.

\chapter{Introduction}
\pagenumbering{arabic}%
\setheader{{\it CHAPTER \thechapter}}{}{}{}{}{{\it CHAPTER \thechapter}}%
\setfooter{\thepage}{}{}{}{}{\thepage}

During the development of the tool HARDY within the AIAI, a need arose
for a data file format for C++ that was easy for both humans and
programs to read, was robust in the face of fast-moving software
development, and that provided some compatibility with AI languages
such as Prolog and LISP.

The result was the PROLOGIO library, which is able to read and write a
Prolog-like attribute-value syntax, and is additionally capable of
writing LISP syntax for no extra programming effort.  The advantages of
such a library are as follows:

\begin{enumerate}
\item The data files are readable by humans
\item I/O routines are easier to write and debug compared with using binary files
\item The files are robust: unrecognised data will just be ignored by the application
\item Inbuilt hashing gives a random access capability, useful for when linking
up C++ objects as data is read in
\item Prolog and LISP programs can load the files using a single command
\end{enumerate}

The library was extended to use the ability to read and write
Prolog-like structures for remote procedure call (RPC) communication.
The next two sections outline the two main ways the library can be used.

\section{PROLOGIO for data file manipulation}

The fact that the output is in Prolog syntax may be irrelevant for most
programmers, who just need a reasonable I/O facility.  Typical output
looks like this:

\begin{verbatim}
diagram_definition(type = "Spirit Belief Network").

node_definition(type = "Model",
  image_type = "Diamond",
  attribute_for_label = "name",
  attribute_for_status_line = "label",
  colour = "CYAN",
  default_width = 120,
  default_height = 80,
  text_size = 10,
  can_resize = 1,
  has_hypertext_item = 1,
  attributes = ["name", "combining_function", "level_of_belief"]).

arc_definition(type = "Potentially Confirming",
  image_type = "Spline",
  arrow_type = "End",
  line_style = "Solid",
  width = 1,
  segmentable = 0,
  attribute_for_label = "label",
  attribute_for_status_line = "label",
  colour = "BLACK",
  text_size = 10,
  has_hypertext_item = 1,
  can_connect_to = ["Evidence", "Cluster", "Model", "Evidence", "Evidence", "Cluster"],
  can_connect_from = ["Data", "Evidence", "Cluster", "Evidence", "Data", "Cluster"]).
\end{verbatim}

This is substantially easier to read and debug than a series of numbers and
strings.

Note the object-oriented style: a file comprises a series of {\it clauses}.
Each clause is an object with a {\it functor}\/ or object name, followed
by a list of attribute-value pairs enclosed in parentheses, and finished
with a full stop.  Each attribute value may be a string, a word (no quotes),
an integer, a real number, or a list with potentially recursive elements.

The way that the facility is used by an application to read in a file is
as follows:

\begin{enumerate}
\item The application creates a PrologDatabase instance.
\item The application tells the database to read in the entire file.
\item The application searches the database for objects it requires,
decomposing the objects using the PROLOGIO API. The database may be hashed,
allowing rapid linking-up of application data.
\item The application deletes or clears the PrologDatabase.
\end{enumerate}

Writing a file is just as easy:

\begin{enumerate}
\item The application creates a PrologDatabase instance.
\item The application adds objects to the database using the API.
\item The application tells the database to write out the entire database,
in Prolog, LISP or CLIPS notation.
\item The application deletes or clears the PrologDatabase.
\end{enumerate}

To use the library, include "wx.h", "read.h" and link with libproio.a
(UNIX) or prologio.lib (PC) and the wxWindows library.

\section{PROLOGIO for RPC}

An RPC package is required because one application cannot simply call
into another's program area. The arguments and return value(s) need to
be wrapped up at one end and parsed at the other. This library provides
a convenient way of doing so, allowing a range of types of data to be passed
between applications, namely integers, reals, strings, words (unquoted
strings), and lists of these types (which can include further lists).
Therefore two main capabilities are required: a way of encoding
information in a string, and a way of communicating this string between
applications.  The basic PROLOGIO functions described above are used for
the first, and the wxWindows implementation of DDE is used for the
second.

The RPC facility is used in much the same way as the DDE facility in
wxWindows, only the base classes to use are rpcServer, rpcClient and
rpcConnection.  From these derive your own client, server and connection classes,
overriding {\bf OnAcceptConnection} (server) and {\bf OnMakeConnection}
(client) as usual. You now have a choice of how the server connection
responds to calls: either you can install your own {\bf OnCall} handler
which accepts and returns Prolog structures, or you may install an {\bf
rpcCallTable} so calls get automatically routed to appropriate C++
functions.  If no call table has been installed, the library will call
{\bf OnCall} instead.

Note that the topic name when using the RPC package must always be
"RPC".

RPC is implemented as follows. When the client calls {\bf
rpcConnection::Call} with a {\bf PrologExpr} structure as argument, this
structure is written to a string and {\bf Execute} is called. At the
server end, the data is parsed back into a {\bf PrologExpr} structure,
and either {\bf OnCall} is called, or, if there is a call table, a
functor (procedure name) matching the supplied name is searched for. If
one is found, the argument types are checked. If they are incorrect, an
error structure is sent back to the client. If they are correct, the
appropriate user-supplied C++ function implementing this command is
called, which returns another {\bf PrologExpr}. This structure is
written to a string, and when the client side {\bf Requests} the result, is
returned. The client side parses the result and returns a {\bf
PrologExpr} as the result of the original {\bf Call}.

To the client application, the {\bf Call} is like a normal procedure
call except the name and arguments need to be wrapped up in a {\bf
PrologExpr} structure, and the result is also a {\bf PrologExpr}.
Similarly, the server application implements a series of functions, one
per command, receiving and returning {\bf PrologExpr} structures.

To use the library, include "wx.h", "read.h", "prorpc.h" and link with
libproio.a (UNIX) or prologio.lib (PC) and the wxWindows library.

\section{Availability and location of PROLOGIO}

PROLOGIO is distributed as part of the wxWindows GUI library. Anyone
interested in using the library can contact Julian Smart, email
J.Smart@ed.ac.uk, tel. 031-650-2746.

\chapter{PROLOGIO compilation}
\setheader{{\it CHAPTER \thechapter}}{}{}{}{}{{\it CHAPTER \thechapter}}%
\setfooter{\thepage}{}{}{}{}{\thepage}

To make use of PROLOGIO, you currently need the following:

\begin{enumerate}
\item GNU C++ version 2.1 or later, or Microsoft C/C++ version 7 or Visual C++
\item The PROLOGIO library (libproio.a or prologio.lib) and header files
\item The wxWindows library (available from Julian Smart at AIAI)
\end{enumerate}

For UNIX compilation, ensure that YACC and LEX or FLEX are on your system. Check that
the makefile uses the correct programs: a common error is to compile
y\_tab.c with a C++ compiler. Edit the CCLEX variable in make.env
to specify a C compiler. Also, do not attempt to compile lex\_yy.c
since it is included by y\_tab.c.

For DOS compilation, the simplest thing is to copy dosyacc.c to y\_tab.c, and doslex.c to
lex\_yy.c. It is y\_tab.c that must be compiled (lex\_yy.c is included by
y\_tab.c) so if adding source files to a project file, ONLY add y\_tab.c
plus the .cc files. If you wish to alter the parser, you will need YACC
and FLEX on DOS.

The DOS tools are available at the AIAI ftp site, in the tools directory. Note that
for FLEX installation, you need to copy flex.skl into the directory
c:/lib.

If you are using Borland C++ and wish to regenerate lex\_yy.c and y\_tab.c
you need to generate lex\_yy.c with FLEX and then comment out the `malloc' and `free'
prototypes in lex\_yy.c. It will compile with lots of warnings. If you
get an undefined \_PROIO\_YYWRAP symbol when you link, you need to remove
USE\_DEFINE from the makefile and recompile. This is because the parser.y
file has a choice of defining this symbol as a function or as a define,
depending on what the version of FLEX expects. See the bottom of
parser.y, and if necessary edit it to make it compile in the opposite
way to the current compilation.

To test out PROLOGIO, compile the test program (test.exe), find
badcase.txt in the docs directory, and try loading it into the test
program. Then save it to another file. If the second is identical to the
first, PROLOGIO is in a working state.

\chapter{Bugs and future developments}
\setheader{{\it CHAPTER \thechapter}}{}{}{}{}{{\it CHAPTER \thechapter}}%
\setfooter{\thepage}{}{}{}{}{\thepage}

\section{Bugs}

These are the known bugs:

\begin{enumerate}
\item Functors are permissable only in the main clause (object).
Therefore nesting of structures must be done using lists, not predicates
as in Prolog.
\item There is a limit to the size of strings read in (about 5000 bytes).
\end{enumerate}

\section{Future developments}

None envisaged as yet.

\input tutorial.tex
\input classes.tex
%
\addcontentsline{toc}{chapter}{Index}
\printindex
\end{document}
