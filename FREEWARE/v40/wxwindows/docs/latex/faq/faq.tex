\documentstyle[a4,makeidx,verbatim,texhelp,fancyhea,mysober,mytitle]{report}
% This is for dvi2tty only
%\setlength{\textwidth}{12 true cm}    % shorten line length
%\renewcommand{\baselinestretch}{.5}   % singlespacing
%
\input psbox.tex
\parskip=10pt
\parindent=0pt
\title{wxWindows Frequently Asked Questions Version 1.6}
\author{Julian Smart and others}
\date{January 1997}
\makeindex
\begin{document}
\maketitle
\pagestyle{fancyplain}
\bibliographystyle{plain}
\setheader{{\it CONTENTS}}{}{}{}{}{{\it CONTENTS}}
\setfooter{\thepage}{}{}{}{}{\thepage}%
\pagenumbering{roman}
\tableofcontents
%
\chapter{About this document}
\setheader{{\it FAQ}}{}{}{}{}{{\it FAQ}}%
\setfooter{\thepage}{}{}{}{}{\thepage}%

This is the questions-and-answers document for wxWindows, the free multi-platform
GUI C++ library. Please feel free to comment on this FAQ and submit new entries.

\chapter{General questions}

\section{What are the licensing considerations for wxWindows?}

None. You may make any use of any part of wxWindows, and no payment is
necessary. While it's nice if you acknowledge the author(s) of wxWindows
in your own work, it's not necessary.

Correspondingly, there is no warranty with wxWindows, as you have probably
guessed by now.

\section{Why is wxWindows free, and will it ever be commercial or shareware?}

It's free because:

\begin{itemize}
\item it was never intended to compete as a commercial product, since
there are not sufficient resources at AIAI.
\item the feedback and bug fixes that AIAI get from the Internet community
are extremely valuable.
\item having used much free software in the past, it's only right
to put something back.
\item AIAI gains a small amount of publicity (well, so the story goes).
\item I got bored of writing code that never saw the light of day.
\end{itemize}

Neither I nor AIAI have any plans to produce a commercial or shareware version.

\section{Why the silly name?}

{\bf w} for MS Windows, {\bf x} for the X windowing system, Windows for those rectangular
things you see a lot of. Ok, so it's not exactly inspired.

\section{What bitmap loading facilities are available for wxWindows?}

There is Windows .BMP code that compiles under Windows distributed with
wxWindows 1.50k (utils/dib directory). wxWindows 1.60 will allow proper
colourmap setting with this code. DIB allows loading and saving BMP
files. The contributed Windows wxImage library (utils/wximage/win)
enables GIF loading (and other formats such as JPEG if extra graphics
libraries are installed). 

For X, the utils/image directory contains code to load GIFS, Windows
bitmaps and X bitmaps into a canvas (and optionally, into a wxBitmap).
The code has been taken from a pre-shareware version of the excellent
image viewer XV; it cannot be guaranteed that the code is free from
copyright issues.  See the file test.cpp for a scanty explanation of how
to use it.

XPM (colour X Pixmap) files are supported by the wxXPM package
now bundled with wxWindows, from 1.61. The package is in contrib/wxxpm,
and includes a utility XPMShow which allows conversion between XPM
and BMP files under Windows (only, at present).

For the maximum bitmap facilities, wx\_setup.h should be edited
and the following settings made:

\begin{itemize}
\item Set USE\_IMAGE\_LOADING\_IN\_X to 1
\item Set USE\_IMAGE\_LOADING\_IN\_MSW to 1
\item Set USE\_XPM\_IN\_X to 1
\item Set USE\_XPM\_IN\_MSW to 1
\end{itemize}

Then recompile the wxWindows, image (X), DIB (Windows) and XPM (X and Windows)
libraries and link in with your application. You can now load and save
XPMs (X and Windows), load BMP files (X and Windows), save BMP
bitmaps (Windows), and load GIFs (X), all through the wxBitmap interface.

It is recommended that you use XPMs for colour bitmap buttons,
at least under X. Note that colour buttons will not display correctly
on X terminals whose display depth does not match the bitmap depth,
so checking will need to be done in the application.

\section{How can I convert Windows bitmaps to UNIX formats?}

To convert Windows bitmaps to XBM (monochrome X bitmap), you can use the
shareware package PaintShop Pro to change coloured areas to black or
white, and then save the image as a PPM file.  On UNIX, use the image
manipulation package XV to convert from PPM to XBM. 

You can convert to XPM using XPMShow, in the wxWindos {\tt utils/xpmshow} directory.

\section{Can I put a canvas (or text subwindow) in a panel?}\label{nesting}

Before 1.61, this kind of nesting was not allowed, because XView
doesn't support it, and wxWindows was heavily influenced by XView.

From 1.61, such restrictions are being relaxed a bit for platforms
that support more flexibility. Under Motif and Windows, canvases and
text subwindows can be placed in panels as well as in frames.

Also, again from 1.61 on, wxPanel is a subclass of wxCanvas, under Windows
(only, at present). So drawing in panels is now possible (or placing panel items in
a canvas, whichever way you like to look at it). Hopefully this will soon be
extended to Motif; it is unlikely to be implemented for XView since
XView doesn't really support this way of working.

\section{Can I draw in a panel, or place panel items in a canvas?}

See \helpref{Nesting subwindows}{nesting}.

\section{How were the samples (and other code) created?}

The samples and other code were all created with a text editor, with nary
an Integrated Development Environment in sight. In future, as
wxBuilder and the wxWindows resource system matures, it is to be hoped
that some code will have been generated by wxBuilder.

See also \helpref{Can I use an IDE?}{ide}.

\section{How can I use more than the basic Windows colours in a 256-colour mode graphics adapter?}

Andrew Davison has the following advice for creating a colourmap. Once you have a valid wxColourMap,
you need to set the colourmap for the device context and window.

\begin{verbatim}
At 12:13 5/06/96 -0400, you wrote:
>This is what I did to set the wxColourMap :
>
>	const int nocol = 256 ;
>
>	unsigned char red[nocol] ;
>	unsigned char blue[nocol] ;
>	unsigned char green[nocol] ;
>
>	int i ;
>	for (i=0 ; i<nocol ; i++){
>		red[i] = i ;
>		green[i] = i ;
>		blue[i] = i ;
>	}
>
>	// wxCM is a wxColourMap
>	wxCM.Create(nocol, red, green, blue) ;
>}
>
>What I do after is wxDC::SetColourMap(&wxCM) with the wxDC's I use.   
>
>I'm not to sure with what values to fill the arrays I pass to ::Create.
>With these values (i.. [0..255]) it didn't change anything.
>
>	If anyone has a clue,
>				Patrick

I would suggest modelling the Netscape colour-cube, which attempts
to evenly divide up the spectrum. 

 Netscape uses only 216 colours plus another 4 for it's logo. How you fill
in the rest is up to you, but under under MSW Windows itself will always steal
20 colors.

Select colours as follows and load them into your wxColourMap. Using
r, g and b values of 0x00, 0x33, 0x66, 0x99, 0xCC, 0xFF respectively.

i.e

  unsigned char r[256], 
  unsigned i = 0;

  for (int r = 0x00; r <= 0xFF; r += 0x33)
      for (int g = 0x00; g <= 0xFF; g += 0x33)
          for (int b = 0x00; b <= 0xFF; b += 0x33)
          {
              red[i] = r;
              green[i] = g;
              blue[i] = b;
              i++;
          }

Given the above scenario it's always likely that your randomly chosen RGB value
will have a close match.

It would be nice of course if WXWINDOWS detected a 256 colour adapter and did
something similar.

Regards.

Andrew Davison
\end{verbatim}

\chapter{Compilation issues}

\section{General}

\subsection{What I can do to reduce executable size?}

wxWindows does produce large executables, but there's quite a lot
one can do to mitigate the problems.

\begin{itemize}\itemsep=0pt
\item Eliminate debug info, either at link time or compile time. The latter usually
has a slight size advantage over the former, but you may consider it not
worth the hassle of completely recompiling for delivery.
\item Optimise for space. Depending on the compiler, this can be quite significant.
With some compilers you can't optimise because of compiler bugs. However,
when I started using Watcom C++ to produce WIN32 executables, I got
a reduction from 3MB for my main tool (Hardy) compiled under VC++ 1.5,
to 2MB under Watcom, with optimisation. This is actually not a bad size for
a piece of software that has been growing for several years, and does
quite a lot.
\item Eliminate unnecessary modules in wxWindows through wx\_setup.h
and makefile settings, e.g. wxXPM, doc/view, wxPostScriptDC.
Through doing this, my WIN32 Tex2RTF LaTeX to HTML, WinHelp/RTF and wxHelp
converter is 633 KB (compiled with Watcom). Quite a reasonable
size these days.
\item Compress the executable with a suitable compressor (that can
make a self-extracting .exe, transparent to the user), or other
method for distribution. I don't do this myself, but it would be a neat way of
saving disk space.
\item On UNIX, create a shared version of the wxWindows library; see
contrib/wxshlib for GNU autoconf files for building shared and
static libraries.
\item Under Windows, make a DLL out of wxWindows: not yet possible, but will probably
be done in the next few months because WIN32 DLLs are apparently
easier to make than WIN16 ones. This will spectacularly reduce the
size of executables, at the cost of confusion with DLL versions.
But for saving disk space during development this would be a good option,
and will probably reduce link times too. See \helpref{Making a Windows DLL}{windowsdll} for notes on
a not-quite-successful attempt to make a DLL under VC++ 4.0.
\item Get a better optimising linker. Part of the size problem is that the
compilers don't throw away sufficient bits of unused code. They may
get better at doing this as time goes on and everyone's executables
get larger.
\end{itemize}

The Windows hello demo executable is around 700KB for VC++ 1.5, less for
Watcom. This demo differs from traditional hello demos in using quite a
lot of GUI functionality, and therefore pulling in a lot of code.
And most options are enabled in wx\_setup.h. 

Obviously, the larger your application, the less you notice the overhead
of wxWindows.

\subsection{What are ItsyBits, FAFA etc.? Which libraries do I really need to compile?}

From wxWindows 1.61, the makefiles have been altered so that several
`subordinate' libraries are compiled into wx.lib (or wx\_motif.a or
whatever). This means that configuration of wxWindows is much more
centralized, and it's not necessary to fiddle with many makefiles if you
decide to compile in a specific wxWindows feature.

These little libraries add optional functionality to wxWindows,
supported in the wxWindows class library but the bulk of the
functionality being implemented separately for modularity (and potential
copyright) reasons.

Unfortunately, you do have to edit {\it both} wx\_setup.h and the makefile
in src/x or src/msw in order to configure wxWindows. So it
may be easier to compile all libraries rather than try to configure
wxWindows, unless you're really having trouble compiling one of
the libraries.

Here's a list of the optional libraries (found in wx/contrib or wx/utils).
The relevant wx\_setup.h identifier is given in brackets.

\begin{description}
\item[CTL3D] Windows only: allows use of 3D style controls (CTL3D).
\item[FAFA] Windows only: allows use of bitmap buttons, messages and radiobuttons (FAFA\_LIB).
\item[ItsyBitsy] Windows only: supports tiny titlebars (USE\_ITSY\_BITSY).
\item[Gauge] Windows only: necessary for implementation of wxGauge class (USE\_GAUGE).
\item[xmGauge] Motif only: necessary for implementation of wxGauge class (USE\_GAUGE).
\item[wxXPM] All platforms: necessary for implementation of XPM pixmap functionality (USE\_XPM\_IN\_X, USE\_XPM\_IN\_MSW).
\item[DIB] Windows only: necessary for implementation of BMP loading/saving functionality (USE\_IMAGE\_LOADING\_IN\_MSW).
\item[wxImage] X only: necessary for implementation of BMP, GIF loading functionality (USE\_IMAGE\_LOADING\_IN\_X).
\item[PROLOGIO] All platforms: necessary for .WXR wxWindows resource-loading functionality (USE\_MSW\_RESOURCES).
\item[RCPARSER] Windows only: necessary for dynamic icon loading (USE\_RESOURCE\_LOADING\_IN\_MSW).
\end{description}

Note that if you don't compile in DIB, you could still use wxLoadBitmap in an application
and link with dib.lib separately in your application makefile. Similarly, you can use PROLOGIO and RCPARSER
independently without them being compiled into wx.lib.

\subsection{Is CTL3D required?}

Here's an excerpt from the wxWindows manual.

It is recommended that CTL3D is used under Windows, since the 3D effects
are good-looking and will be standard with Windows 4.0. If you want to
use it and don't already have CTL3D installed, copy the files in
contrib/ctl3d to appropriate places (ctl3dv2.lib/ctl3d32.lib into your compiler lib
directory, ctl3d.h into an include directory, and ctl3dv2.dll into
windows/system). You may need to find a compiler-specific version of ctl3dv2.lib
or ctl3d32.lib. Define CTL3D to be 1 in wx\_setup.h and link your executables with ctl3dv2.lib
or ctl3d32.lib.

If both CTL3D and FAFA are set to 1, then all controls except wxButton
will use CTL3D and have 3D appearances. wxButton will have the ability
to use bitmaps. This is the recommended configuration.

Windows 95 update: dialogs can be marked with the Win95 3D look
by specifying the DS\_3DLOOK. But this doesn't apply to panels.
The WIN32 SDK documentation says that the style WS\_EX\_CLIENTEDGE can be
used for controls, to give them all 3D looks. However, this doesn't appear to
work (and causes strange 2-column behaviour in wxListBox). Even marking the
executable as Windows 4.0 only gives a wxChoice items a 3D look.
So it seems that for now, CTL3D is still required for Windows 95 applications.

\subsection{I need a drink. Why is compilation so difficult on some platforms?}

It's a good question; you may be lucky enough to sail through wxWindows
installation without a hitch, or you may exhaust your vocabulary of
expletives before you're done compiling the first sample application.

There are a number of possible reasons for things to go wrong:

\begin{itemize}
\item Makefiles need to be adjusted (especially make.env) to add include and
library paths, and library flags, specific to that OS or compiler.
\item I've messed up the distribution. Occasionally I edit a file at the last
minute without testing it properly... Normally these problems become apparent
quite quickly.
\item There's an honest-to-goodness bug in wxWindows. Sorry! but wxWindows is
quite complex, and bugs happen. Whether you can classify not coping with
a particular setup a bug, I don't know, but there will be occasions when
installation reveals a bug. Mostly, though, real bugs are only identified
when applications get complex.
\item Your compiler has not been installed properly. This is often signalled
by missing libraries such as iostream.
\item There's a compiler incompatibility. This is extremely rare, since
wxWindows uses a very limited subset of C++ syntax, and steers
clear of unportable constructs such as templates.
\item There's a bug in the compiler. This happens surprisingly often, particularly
with GNU C++ where the latest release might have a brand new bug. This can manifest
itself as a bizarre link error, or run-time problem such as the message "You must define an instance
of wxApp!" (globals haven't been initialized properly by the compiler).
\item There's a bug in the OS, such as a lack of certain include files (it happened
with some versions of SunOS).
\item You're using a compiler and/or OS that no-one's tested wxWindows out on before.
If you're really unlucky (and intrepid) you could find yourself doing a 'port'
to an environment never before encountered. In fact, the changes involved are
usually quite small, and are nearly always centred around wx\_utils.cpp and wx\_ipc.cpp
which make heavy demands on operating system-sensitive areas.
\end{itemize}

In general, the reason why compiling wxWindows can be more troublesome than other packages
is that with conventional application building, you {\it gradually} use more and more
parts of the operating system or GUI toolkit. With wxWindows, because it covers
a large `surface area', you're encountering these possible troublespots all at once
when you compile the library.

The good side of all this is that once you {\it have} ironed out the initial compilation
and run-time problems, these particular headaches ought to be minimal from then on.
So don't be too discouraged if installation is initially difficult!

Borland C++ seems to generate the most traffic for installation problems. I'm
not exactly sure why this is, since although I don't have Borland C++, various
people have contributed tips and makefiles. I suspect that something about
the design of Borland C++ makes it difficult to compile a large project without a lot
of in-depth knowledge about the compiler options.

\section{UNIX}

\subsection{Is there a GNU configure script for wxWindows?}

Yes, it's in the contrib/wxshlib directory of the distribution, from 1.66 onwards.

\subsection{Linux issues}

\subsubsection{Segmentation fault on startup}

Some versions of gcc are buggy and cause problems with wxWindows and other software.

From Wolfram Gloger:

\begin{verbatim}
I'm afraid the answer is probably `don't use Slackware' (for C++
development, that is, it may well be a great distribution otherwise).
Slackware has been well known for shipping inconsistent
compiler/library versions.  At this stage, you should really only use
gcc-2.7.2 with libg++-2.7.1.4 (note the trailing `.4' indicating
H.J.'s patchlevel).

A good test would be to compile `#include <stream.h> int main() { cout
<< "Hello\n"; }' and see if that runs.  If it doesn't, you have
obviously no chance to run wxWindows, either.

Regards,
Wolfram.
\end{verbatim}

The experience of another Linux user:

\begin{verbatim}
Hi Julian,

you wrote:
>This is quite a common experience under Linux and it is solvable - I must 
>try to get a coherent story on what the problem and fix is.

you are right, it's very difficult to get a coherent story on what it depends.
After getting some experience I can definitively say:

1) 	At first it is a problem of libg++-2.7.0 -> libg++-2.7.1.3
	Don't use them, they will not work.  SEG-FAULT !!
	You have to use libg++-2.7.1.4 . FOUR !! is very importent, 
	thanks Wolfram.

2)	It is a problem of gcc/g++ 2.7.0 too, exspecially of the linker.
	Maybe it's only the libc.a . I have changed only whole packages.
	It caused a problem with the slider.

3)	libiostream.a came with libgxx-2.7.0 .
	In the package libgxx-2.7.1.4 wasn't any libiostream.
	{{So I couldn't upgrade this library. 
	  I still have problems, now the choise-box causes failure
	  Wolframs little program 'cout' starts with SEG-FAULT.}}
	  I just heard it's now itegrated in the new stdc++ and I
	  have to delete the old one.

Unfortunately there is no revisions-number and patchlevel-number in the
library-name of libg, ldso-1 and libiostream and I don't know how to get it.

Point 1) + 2) will help other Linux user if they could find it your 
NOTES FOR LINUX USER.

That's all till now.
Thanks a lot for your help and to all the other helpfull people. 
Otherwise I didn't search for a solution at this point.

Regards,
Juergen
\end{verbatim}

\subsubsection{I get a link error for strchr}

Try adding -lstdc++ to the link flags.

\subsubsection{Why do the makefiles not work?}

You may be using `pmake': the wxWindows makefiles
require you to be using the default GNU make, which has
a slightly different syntax (for example, the include
statement syntax is different).

\subsubsection{My binaries are enormous! What can I do?}

wxWindows 1.60 improves on 1.50 by the use of GCC pragmas to
specify which files are interfaces and which are implementation.

Also, if you compile everything without debugging information,
GCC will use dynamic link libraries for X11, XView and some others;
this reduces the size of the binary substantially.

You can also create a shared version of the wxWindows library; see
contrib/wxshlib for GNU autoconf files for building shared and
static libraries, and also \helpref{Building shared libraries}{shared}.

\subsubsection{Building shared libraries on UNIX}{shared}

See contrib/wxshlib for GNU autoconf files for building a shared version
of wxWindows. Here's another way to do it in Linux:

{\small
\begin{verbatim}
Date: Mon, 10 Feb 1997 21:42:29 +0100
From: Erwin Nijmeijer <E.Nijmeijer@inter.NL.net>
Reply-To: E.Nijmeijer@inter.NL.net
To: julian.smart@ukonline.co.uk
Subject: Shared library & gcc & IOU1

After looking at your homepage I discovered that there is a special
script for creating a shared wxWindows library. To create a shared
version of the library, I used a much easier way :

a) Add -fPIC to CPPFLAGS
b) create the static library as described in the documentation
c) copy the library to a special temporary directory and extract all
   the objects using the archiver :
   ar x libwx_motif.a

d) rebuild the shared library using these objects :
   gcc -shared -o libwx_motif.so *.o

maybe I'm thinking just too simple but it seems to work fine with me !
\end{verbatim}
}

\subsubsection{How can I reduce wxWindows compilation times on Linux?}

A solution from Giovanni Agostino Andrea Giorgi (giorg@dsi.unimi.it).

\begin{verbatim}
For Linux Users with at least 8Mb of Ram.....
About wxWin 1.63 compilation Speed. 
I have found a small solution  for this  problem:

Rules are simple....:)

	1) DO NOT run  X-Windows BEFORE compiling
	2) Include DIRECTLY the ".h" files
	3) Options for gcc:
		-O0 [ -w ]
	4) Try to use more files, to link togther at the end...
With this method I reduced compilation time (and linking, of 
course !) of minimal.cpp from over 4' to  about 1':30''.

I think this is good, because I have only 8 Mb....

Thank to all !
Regards
\end{verbatim}

\subsubsection{Why does the Xfree ATI Mach32 server hang when drawing
graphics?}

(This FAQ is probably obsolete by now).

Harri Pasanen has discovered a bug in the Mach32 server.
It hangs if using pens with wxDOT linestyle, and width zero.

This has been reported to the Xfree developers.

\subsection{Solaris 2.x issues}

\subsubsection{I get some warnings and link errors. What gives?}

You need to:

\begin{itemize}
\item Compile with -DSVR4. Add this to OPTIONS line in each
makefile.unx.
\item Add the following to LDFLAGS: -lgen -ldl -lsocket -lnsl
\end{itemize}

Note that the libgen.a lives in /usr/ccs/lib, if you
have installed the programming tools option.

Version 4.0 of Sun C++ is apparently more pedantic than
older versions, and requires the use of CC -migration to
help with the necessary changes. You may need to include the file strings.h
where the file string.h is included, with CC, e.g. in wb\_utils.cpp.

Someone reported that link errors on a SPARCStation were
cured by adding -lucb and -I/usr/ucbinclude/sys.

For dynamic linking under Solaris 2.3, the following
changes are required:

\begin{itemize}
\item In wxinstal, add:

\begin{verbatim}
export OPTIONS
OPTIONS=-fPIC
\end{verbatim}%
\item in src/x/makefile.unx, add:

\begin{verbatim}
WXLIB = $(WXDIR)/lib/libwx$(GUISUFFIX).so.0

$(WXLIB): $(OBJECTS) $(BASEOBJECTS)
      	ld -G -o $(WXLIB) $(OBJECTS) $(BASEOBJECTS)
\end{verbatim}

where the ld line replaces the ar+ranlib command.
\end{itemize}

Here are my (JACS) own experiences. As of 25th May 1995, I finally got a clean
compilation under Solaris (using XView), though I needed to change a lot
of files, e.g. in wxXPM and wxImage. The following is a script I wrote
to save editing make.env, for Solaris compilation. It shows the kinds of
settings required.

I think for some environments you may need to add -L/usr/ccs/lib -lgen
to the COMPLIBS line.

The changes required to compile under Solaris will be in version 1.62 beta (b).
New versions of Solaris and the SunPro compiler may break all this, of course.

\begin{verbatim}
#!/bin/sh
# makeunix
# Invokes makefile with specific XLIB and XINCLUDE settings,
# IFF your version of make can take the -e flag
# (environment variables take precedence.)
export XINCLUDE
export XLIB
export CC
export CCC
export CCLEX
export DEBUG
export WARN
export RANLIB
export COMPLIBS
export OPTIONS
CC=CC
CCC=cc
CCLEX=cc
OPTIONS=-DSVR4
COMPLIBS='-ldl -lsocket -lnsl'
XINCLUDE=-I/usr/openwin/include
XLIB='-L/usr/local/X11/lib -L/usr/openwin/lib'
DEBUG=
WARN=
RANLIB=echo
make -f makefile.unx -e $@
\end{verbatim}

\subsection{Compiling on OSF/1}

Here's how.

\begin{verbatim}
From: Asociacion Fisica Universidad <afu2@eucmos.sim.ucm.es>
Subject: Ported wxWin 1.60 to OSF1
To: J.Smart@ed.ac.uk

        Hi!

        I've been playing around with wxWin (great package!) and I've 
make it to compile under OSF/1 with motif. I send you the modified 
make.env, just in case.

        There is a minor change in a file I can't remember which is, but 
is in someplace in which it makes a wait and you say it's bad, that it 
has to be remade. There's a conditional compilation there, and where we 
can fidn #if !defined(SVR4) && .. etc, just include a !defined(OSF1). It 
shoudl work all right.

        Well, here's the make.env.osf1. If you have any comments, let me know!

# make.env

# slightly touched by Iniaky Perez Gonzalez (afu2@fis.ucm.es, 2:341/5.31) 
# to work fine under OSF/1

# Common makefile settings for wxWindows programs
# This file is included by all the other makefiles, thus changes 
# made here take effect everywhere (except where overriden).
#
# An alternative to editing this file is to create a shell script
# to export specific variables, and call make with the -e switch
# to override makefile variables. See wx/install/install.txt.
# And you can override specific variables on the make command line, e.g.
#
# make -f makefile.unix DEBUG=''
#

########################## Compiler ##################################

# C++ compiler
#CC = gcc-2.1
CC = cxx

# C compiler for pure C programs
# Typical: CC=g++ , CCC=gcc
#          CC=cl386 /Tp, CCC=cl386
#
# (Used only for XView, file sb_scrol.c)
#
CCC = cc

# Compiler used for LEX generated C
CCLEX=$(CCC)

########################## Compiler flags #############################

# Miscellaneous compiler options
# May need to add -D_HPUX_SOURCE_ for HPUX
# Solaris: add -DSVR4
OPTIONS= -Dosf1 -DOSF1 -D__OSF1

# Debugging information
#DEBUG = -g
DEBUG =

# Warnings
WARN = 

# Which GUI, -Dwx_xview or -Dwx_motif (don't change this)
GUI = -Dwx_motif

# Optimisation
# OPT = -O
OPT =

# Options for ar archiver
# AROPTIONS = crs # For IRIX. Also, comment out ranlib line.
AROPTIONS = sruv

# Compiler libraries: defaults to GCC libraries
# Sun with Sun CC: -lc
# Solaris: -lgen -ldl -lsocket -lnsl
#   and/or possibly -lucb, whatever that is...
# SGI:     -lPW
COMPLIBS=-lc -lm -lcxx

# Compiler or system-specific include paths
# E.g. some SPARCStations need
# -I/usr/ucbinclude/sys
COMPPATHS=-I/usr/include/cxx

# HP-specific compiler library: an AIAI convenience
HPCOMPLIBS=

# LDLIBS for specific GUIs
MOTIFLDLIBS = -lwx_motif -lXm -lXt -lX11 -lm $(COMPLIBS)
XVIEWLDLIBS = -lwx_ol -lxview -lolgx -lX11 -lm $(COMPLIBS)
HPLDLIBS=-lwx_hp -lXm -lXt -lX11 -lm

# Default LDLIBS for XView (don't change this)
LDLIBS = $(XVIEWLDLIBS) -lbsd

# _ol or _motif (don't need to change, the makefiles will take
# care of it if you use motif/hp/xview targets)
GUISUFFIX=_motif

########################## Directories ###############################

# Replace X include/lib directories with your own
INCLUDE=-I/usr/include -I/usr/include/X11 -I/usr/include/Xm
LIB=-L/usr/local/X11/lib -L/usr/lib/Xm
#XINCLUDE=-I/aiai/packages/motif1.2.1/motif/include -I/aiai/packages/X.V11R5/inc
lude
#XLIB=-L/aiai/packages/motif1.2.1/motif/sun4/lib -L/aiai/packages/X.V11R5/lib

# A convenience, for HP compilation
HPXINCLUDE=-I/usr/include/Motif1.2 -I/usr/include/X11R5
HPXLIB=-L/usr/lib/Motif1.2 -L/usr/lib/X11R5

# Shouldn't need to change these...
WXINC = $(WXDIR)/include/x
WXBASEINC = $(WXDIR)/include/base
WXLIB = $(WXDIR)/lib/libwx$(GUISUFFIX).a
INC = -I$(WXBASEINC) -I$(WXINC) $(COMPPATHS)

# Directory for object files (don't change)
OBJDIR = objects$(GUISUFFIX)

# You shouldn't need to change these...
CPPFLAGS = $(XINCLUDE) $(INC) $(OPTIONS) $(GUI) $(DEBUG) $(WARN) $(OPT)
CFLAGS = $(XINCLUDE) $(INC) $(OPTIONS) $(GUI) $(DEBUG) $(WARN) $(OPT)
LDFLAGS =  $(XLIB) -L$(WXDIR)/lib

# Extra patch link for XView
XVIEW_LINK = $(WXDIR)/src/x/objects_ol/sb_scrol.o
\end{verbatim}

Also:

\begin{verbatim}
Hi,

just one word on how to compile Wx166b on DEC/OSF

The DEC cxx compiler does'nt understand the new .cpp extensions as C plus plus
source files.

Solution
---------

modify the make.env file to add -x cxx to the compiler options

->   OPTIONS = -Dosf1 -DOSF1 -D__OSF1 -x cxx

the -x cxx argument forces to compile any source to C plus plus

(see the FAQ for a complete list of other changes to introduce in the
make.env file)
Facultes Universitaires Catholiques de Mons (F.U.Ca.M.)
Bart JOURQUIN
Departement "Informatique et Gestion Quantitative"
Groupe "Transport et Mobilite"
Chaussee de Binche, 151a
7000 Mons (Belgique)
Tel: (32) 65 32.32.93
Fax: (32) 65 31.56.91
E-mail: jourquin@message.fucam.ac.be
\end{verbatim}

\subsection{Compiling on HP kit}

Here's how.

\begin{verbatim}
Date: Fri, 21 Apr 1995 09:03:32 -0600
From: Bruce Lee <lee@abraham.et.byu.edu>
Apparently-To: J.Smart@ed.ac.uk
Status: REO

Julian,

Thank you for the suggestions concerning the wxEntry problem I was having.  I
changed main.c to a c++ file and commented out the extern C wxEntry in wx_main.c
c
and all was well.  If you or the wxWindows users are interested the following
are the changes I had to make to get wx161 to compile on an HP 7xx/8xx machine
using HP's C++ compiler:

        * Add the compiler flag +a1 to the options field in src/make.env
          This tells the compiler to be ASNI strict.

        * You _must_ use flex _and_ bison to compile y_tab.c in the prologio
          stuff.  Also I found gcc works best to build y_tab.o.

        * In contrib/xmgauge/gauge.c, change #ifdef 0 to #if 0

NOTE: The standard c compiler cc on the HP will warn you that +a1 is an invalid
      option when building non-C++ files.  Gcc will bomb if that option is used
      when building y_tab.c.  You can hack the makefile or create a new macro
      to provide the proper options.

Thank you,

Bruce
\end{verbatim}

\subsection{Compiling Sun dynamic libraries}

(See also contrib/wxshlib).

From Frank Brueggemann:

\begin{verbatim}
From: Frank Brueggemann <fjb@newton.fb5.uni-siegen.de>
Date: Fri, 17 Mar 95 09:07:35 +0100
To: wxwin-users@aiai.ed.ac.uk
Subject: Re: Dynamic libraries on Sparc question from Keith

Yesterday my colleague Dominik and I succeded in compiling 
wxwin as a dynamic library with gnu 2.6.3 and libg++ 2.6.2 for
SunOS 4.1.3. and openwin 3.0.

The behaviour mentioned is a normal result of the dynamic library system,
but it not so obvious at the first moment. It took some time for us 
to figure out what to do. SUN distinguishes between funtional shared
libraries (so called .so files) 
and libraries that exports initialized data (so called .sa files).

If you wish using the wxwin library as a dynamic library you have to
create a libwx_??.sa file too. This is necessary because the files
    src/base/objects_ol/wb_main.o \
    src/base/objects_ol/wb_obj.o \
    src/base/objects_ol/wb_types.o \
    src/x/objects_ol/wx_main.o 
contain such global data. Thus you have to use a command like
ar rv libwx_ol.sa.0.0 \
    src/base/objects_ol/wb_main.o \
    src/base/objects_ol/wb_obj.o \
    src/base/objects_ol/wb_types.o \
    src/x/objects_ol/wx_main.o 

to build this library in addition to the libwx_ol.so.0.0.

I think the compilation is very similar on the SOLARIS OS.
If you need further details please reply.
\end{verbatim}

From Mikhail Tcheznychev:

\begin{verbatim}
Sme words about dynamic library for Solaris 2.4 with
gcc-2.7.0

    In make.env, options:

        OPTIONS=-DSVR4 -fPIC

    In src/x/makefile.unx :

$(WXLIB): $(BASEOBJECTS) $(OBJECTS) $(EXTRAOBJS)
	gcc -G -o $(WXLIB)  -h libwx_ol.so.0 $(EXTRAOBJS) $(OBJECTS) $(BASEOBJECTS)
\end{verbaim}


\subsection{I get link errors for wxEntry, LexFromFile etc.}

Sometimes you might get some or all of these symbols undefined when
linking a sample:

\begin{verbatim}
  wxEntry(int, char**)
  LexFromFile
  PROIO_yyparse
  LexFromString
\end{verbatim}

For PrologIO, try setting CCLEX in make.env to use the C compiler, not the C++ compiler.
With the wxEntry problem, if all else fails, change wxEntry in wx\_main.cpp to main, and
don't link with main.o.

\subsection{I get a link error under SunOS: the symbol XtShellStrings is resolved.}

Tako Schotanus (sst@bouw.tno.nl) writes:

\begin{verbatim}
I was finally able to solve it by adding the following define
to "make.env" :

  -DXTSTRINGDEFINES

This has the effect that whenever there's a reference in the
sourcecode to XtN..... (XtNiconName for example) a #define with
the proper string will be used instead of a global array containing
the names.

System: SunOS 4.1.3
libXt:  4.10
gcc:    2.5.8
\end{verbatim}

Another cause may be having multiple versions of libraries (such as the Motif library)
in your path.

\subsection{I get a libXmu link or run-time error.}

Put -lXmu in your LDLIBS. This library is used by the ComboBox widget; if you can't get rid of the error,
try setting USE\_COMBOBOX to 0 in wx\_setup.h and recompiling wxWindows and the application.

\subsection{How do I link applications statically with X and Motif libraries?}

Sometimes it's desirable to link an application statically, if the
recipient of the executable may not have the appropriate dynamic
link libraries. The tradeoff is larger executable sizes.

Here are some responses to a query I put to wxwin-users.

\begin{verbatim}
If you're doing this on Solaris, try at the end of your $(CC) command

 -Wl,-Bstatic -lXm -lXt -lXmu -lX11 -Wl,-Bdynamic -lgen -lnsl -lsocket

for SunOS - just do -Wl,-Bstatic

for HPUX or AIX I wouldn't worry about it and I'm not sure about
other platforms - Linux might be a problem ...

Hope this helps.

-jb

------------------------------------------------------------------

gcc -static
works for me.

You can also just put the .a files on the command line if you only want 
to link some libs statically, eg
gcc -o prog ob1.o ob2.o /usr/X11/lib/libXm.a  \
	/usr/X11/lib/libXt.a /usr/X11/lib/libX11.a -lm

------------------------------------------------------------------

Just give the  -Bstatic flag to the linker command line, after the -l flags.

Rajive

------------------------------------------------------------------

It's not hard.  If you are using gcc, then supply -static before the list of
link libraries -- i.e. the set of -l parameters (such as -lXm).  If you are
using Sun compilers use -Bstatic.  Just add this flag to the LDFLAGS
parameter in make or imake.  You know you've succeeded when the ldd command
for the created executable returns `statically linked.'

------------------------------------------------------------------

Hi,

> Just give the  -Bstatic flag to the linker command line, after the -l flags.

This is in SunOS, presumably.  On Linux/ELF, it's -static, and
_before_ all -l flags (since you can turn on/off dynamic linking
on a per-library basis).  It's a pity not all Un*x linkers are
the same...

Regards,

Wolfram.
\end{verbatim}

\subsection{What to do if GCC gives non-wxWindows link errors}

Here's a success story from David Starr (dave@doom.sbi.com),
who got the following link errors using GCC 2.5.5. He upgraded
to 2.6.2.

\begin{verbatim}
ld: Undefined symbol 
   _mktemp__FPc 
   ___9streambufRC9streambuf 

Blymn,

I am writing to thank you for your help in carrying me through
to a successful build, at last, of the wxwin 1.63 release.

Especially since the outcome was positive, I thought I would
share with you my interpretations and the specific results
of your recommendations.

You wrote:

 >> One thing that did give me these sorts of problems was a faulty libg++
 >> library.  The problem is that Sun's make handles the output of shell
 >> commands on command lines differently to that of GNU make (Suns dumps
 >> the output as one big argument on the command line while gmake uses
 >> the IFS shell env variable to split the output into arguments).  What
 >> it boils down to is that if you did not compile libg++ using gmake
 >> then the library is seriously broken.  If you try running nm -p on
 >> libg++.a and get a fairly short list of things defined then the
 >> library is broken, if the list goes on and on then look elsewhere for
 >> the problem.

I discounted the suggestion of the faulty libg++ at first because
the indication from running the nm command was that there were lots of
references.  However, as time went along, I became more convinced that
this was the problem, and it was further compounded by libg++ and gcc 
being out of sync.  I had been running gcc 2.5.5.

I went to my sa and asked him to upgrade my gcc.  Being the
enthusiastic sort, he went about upgrading /usr/local/bin/gcc which
clobbered the 2.5.5 compiler in use by about 10 members of our group.
After that was restored, and everyone settled down, I continued.

Our sa then gave me a separate gcc2.6.3.  Then, also thanks to you,
I got by ftp libg++-2.6.2 from prep.ai.mit.edu (no one here seemed
to know the exact site name), and was able to compile
it successfully.  By this time I had forgetton your suggestion to use
gmake, but the result still worked.

 >> Uhh the symbols given are the c++ mangled names.  The GNU binutils has
 >> a name demangler called c++filt.  The names you are having problems
 >> demangle to:
 >> 
 >> mktemp(char *)
 >> streambuf::streambuf(streambuf const &)
 >>
 >> You could try defining these two and see what happens but I doubt if
 >> you will get far.  I still reckon that your g++ lib is stuffed.

However, I still was getting 
  ld: Undefined symbol 
     _mktemp__FPc 

So, thanks to your post, I had learned about c++filt, and was able to 
define mktemp(char*) in my source, resulting in successfully linked
load modules.
\end{verbatim}


\subsection{Why does the XView or Motif file selector crash?}

\begin{verbatim}
> Hi, I compiled wxwin 1.61 beta with gcc 2.6.3 on a Sun sparc.  Everything work
s
> except in the hello demo when I try to open the file selector the program 
> crashes.  Any clue on this is appreciated. (The same problem does not
> occur using gcc 2.4.5. Also I use motif 1.2.)

I had the same problem.  I believe it is *not* the fault of wxWindows.
Instead, the problem appears to lie in the librx (regular expression)
code that is distributed with the most recent version of libg++
(2.6.2?) -- at least, that's where it's crashing.

What I did was to delete all the librx code and all references to it
in the libg++ makefiles, then recompile libg++.  (Just for safety's
sake, I recompiled the wx library as well; I wasn't sure whether it
would have picked up any of the librx code.)  When you relink your
application, the regular expression code in the system libraries will
be used instead.  File selectors now work fine for me.

You may be able to achieve the same effect by making sure that libg++
is the absolute *last* library searched by your compiler (after the
system libraries, in particular), but I haven't tried this.

-------------------------+----------------------------------------
   //  Scott Maxwell:    |  
\\//      maxwell@       | ``Unlike most of you, I am not a nut.''
 XX natasha.jpl.nasa.gov |     -- Homer Simpson
\end{verbatim}

Here's another solution that doesn't involve recompiling libg++:

\begin{verbatim}
------------------
wxWindow 1.62e
HP-UX arcturus A.09.04 U 9000/887
gcc version 2.6.3
------------------

to extract rx.o from libg++:

ar d libg++.a rx.o

Warning !
In case you created a copy of libg++.a, let say libwx_g++.a, whithout
rx.o (cp libg++.a libwx_g++.a; ar d libwx_g++.a rx.o), but let the 
original version of lig++.a in place not to disturb other users of
the library; be sure to call the compiler from your makefiles as "gcc",
because "g++" is an alias which automatically generates a reference to
libg++ (-lg++). Thus even if you would mention -lwx_g++ in you makefiles, 
your changes might seem not to be operative.
\end{verbatim}

And yet another, even simpler solution:

\begin{verbatim}
By the way, I want you to know that libg++ should go before 
libXm.  The way you have it set up in makefile.unx, make.env,
the file - load core dumps in XmCreateFileSelectionDialog.
The reason is that libg++ has a re_create function which 
apparently is the same name as a motif function used by
XmCreateFileSelectionDialog.

David Starr
\end{verbatim}

\subsubsection{ULTRIX compilation}

\begin{itemize}\itemsep=0pt
\item Link the iostream library or you will get link errors.
\item Define NEED\_STRDUP to 1 in controb/wxxpm/libxpm/libxpm.34b/lib/xpm34p.h.
\item Apparently some of the extra libraries (prologio, image etc.) need to be specified
in the sample or application makefile to avoid link errors: I don't know why this should be
since the objects should be linked into the main libwx\_motif.a file.
\end{itemize}

\subsection{Under AIX, wxTheApp does not initialize properly and causes a wxWindows error message.}

After getting wxWindows 1.61 (b) to compiler under AIX, Dirk Eller writes

\begin{verbatim}
The major problem was the initialisation problem of the wxApp-object
also reported in install/install.txt.
The fix:
  #ifdef __aix
  extern wxApp *wxTheApp=1;
  #endif
doesnt work on my system. It looks also very bad ?

After a little testing I figured out that wxApp isn't initialized at the time
when main() (->wxEntry) is called.
The fix is to move the call of main() to the file where the global object
is supposed to be created. (e.g. hello.cpp)

The reason is that (stroustroup) c++ compiler does not HAVE to create a global
object before main, but only before any functions are called in the file that
the object is in. (thanks Eugene)
The fix is not very elegant, but it works. 
\end{verbatim}

\subsection{When deleting a frame or dialog box, the program crashes.}

On some systems, you should not use the delete operator to delete
frames or dialog boxes. Use wxPostDelete instead, to get the object
deleted when X has finished processing all other messages.

Note for version 1.66 and later: wxWindow::Close has been introduced for
delayed deletion of frames and dialogs. Please use this instead of wxPostDelete.

\subsection{How can I compile PROLOGIO successfully under UNIX?}

Check the CCLEX variable in src/make.env; set it to use a bog-standard
(or GNU) C compiler for compiling LEX-generated files.

Using FLEX instead of LEX sometimes helps, too.

Most of the warnings when compiling PROLOGIO are spurious; however,
there may be an error buried inside the warnings. If so, you may need
to change a prototype in a generated .c file to get it compiled.
Hopefully this type of error is getting much rarer now.

\subsection{How can I compile wxXPM successfully under UNIX?}

Check the XPMCOMPILER variable in contrib/wxxpm/makefile.unx; set it to use a bog-standard
(or GNU) C compiler.

\subsection{I get libXmu.so.4 (or similar) not found on linking.}

This library is currently only needed for the ComboBox implementation. If this is causing trouble,
switch off ComboBox compilation in wx\_setup.h and src/x/makefile.unx, and recompile.

\section{VMS}

These are Stefan Hammes' notes for compiling wxWindows on VMS.
They were written for wxWindows 1.61b but most points should hold true for later
versions.

\begin{verbatim}
This port of wxWindows 1.61b is for the DEC C++ compiler on VMS 
for ALPHA and VAX. I'm using an ALPHA, so I cannot guarantee for 
no problems on a VAX, but if there are problems, they will be 
minor ones (mostly with include files etc.).

This port is not a complete one, but all graphical features work.
Timer works. Most other things work. 
The toolbar works fine. All(!) samples (except the IPC one) work.
Things which do not work: IPC (not yet because I don't have sockets),
PROLOGIO and most other utilities (I'm working on this) and the 
contrib stuff.

Most problems occur because of the directory structure and
filenaming of VMS. Beside this, some system dependent headerfiles
of UNIX are not present under VMS.

The directory structure is the same as under UNIX.

Be warned: The DEC C++ compiler is very slow. On a VAXstation 3100 the
compilation time is about 24 hours (!!!). It also needs much much 
memory (surely you have to raise your pagefile size ;-).

Files
-----
Get a copy of wxWindows and put the directory hierarchy on
your disk. Copy the file \include\base\wx_setup.vms to \include\base\wx_setup.h.
Then you should have a ready version of the source code. 

Environment
-----------
Under all circumstances you should make the following definition
in your 'login.com' file:

$ make == "mms/descrip=makefile.vms/macro=(ALPHA=1,WXDIR=[hammes.wxw161)"

Without this, nothing works :-)
Instead of ALPHA=1 you can use VAX=1 if you are on a VAX.
The WXDIR should point to the directory of wxWindows and you MUST omit the
trailing ']' !!! Replace the string 'hammes.wxw161' with the correct one
for your system.

In $(WXDIR).src] we have the two files 'makevms.env' and 'motif.opt'. 
They define options and locations of directories and libraries. 
Edit them for your system. For the first try no editing is necessary.
'makevms.env' will be included in the makefiles so you need to
define the things only once.

!!!! IMPORTANT !!!!
Now you should delete the file 

'$(WXDIR).include.base]wxstring.h' 

and copy the file 

'$(WXDIR).include.base]wx_setup.vms' to '$(WXDIR).include.base]wx_setup.h'
!!!! IMPORTANT !!!!

Makefiles
---------
I have included makefiles with the name 'makefile.vms' in several directories. 
If you have defined the above symbol 'make' and have 'mms' installed on
your system you need only to type 'make' as in UNIX.

N.b.: Instead of having makefiles in [.src.x] AND [.src.base] we only have
      one makefile in [.src.x] which makes the whole library.

You can look at the makefiles and build similar ones for other libraries and
for executables.

Compiling
---------
If everything goes right, you have only to type 'make' in the 
top wxWindows directory and then go to sleep or something else. 
This global make-command will build the wxWindows library, 
the toolbar library, the wxstring library and all sample programs.

As stated above this will take a lot of time (1-2 days :-() for
compiling all components). On an ALPHA it will be somewhat
faster.

VMS port was done by Stefan Hammes (stefan.hammes@urz.uni-heidelberg.de).
In the source code I have marked my changes and additions with 'steve'
for the old changes and 'steve161' for the newest changes.

If you have problems, please send an E-mail to me. I wish you success,

Stefan Hammes
\end{verbatim}

\section{MS Windows}

\subsection{What's the best compiler to use for Windows programming with wxWindows?}

There will be a variety of views on this, but here's my view (Julian Smart).

{\bf Borland C++ 4.x} generates a lot of mailing list traffic with people experiencing
a bewildering variety of problems. The size and scope of wxWindows exacerbates
any problems with a Windows compiler, and Borland is no exception (see later
sections for more details). The debugger is particularly bad, but then
this is a general problem with Windows compilers. Borland has the major advantages
of wide use and 32-bit WIN32S, but I would not recommend it over Visual C++ 1.x
if you are starting out with a choice and do not need 32-bit compilation.

{\bf Visual C7} should work with wxWindows but I'd recommend upgrading.

{\bf Visual C++ 1.5} is what I use for wxWindows and therefore the makefiles
are the most developed, and there will be least trouble in using wxWindows with
Visual C++ 1.5. The CodeView debugger, while boring, is at least reliable and
functional unlike many of the others. However, for 32-bit compilation you'll
need a separate compiler (e.g. VC++ 2.x which comes with VC++ 1.5 on the same
CD-ROM). I haven't used VC++ 2.x but it sounds at least as reliable as 1.5.

{\bf Watcom C++ 10.0} is famous for its generated code speed and ability
to compile in 32-bit mode. Users have spent some time and trouble making
Watcom C++ and wxWindows see eye-to-eye, although there are still a few
wrinkles to be resolved. However 16-bit wxWindows compilation with
Watcom is a no-no in my experience. The text-mode debugger seems OK but flashes like mad
between text screen and Windows screen so CodeView looks like heaven in
comparison. It's possible that your video card, if not your eyes, would
give up the ghost after a lot of this kind od switching. The Windows-hosted debugger
is itself to buggy to use, unfortunately. Watcom boasts a wide range
of targets for the compiler, including 16-bit Windows, 32-bit standard Windows API,
32-bit WIN32 and WIN32s API, and even OS/2 if you purchase more kit from
IBM for Presentation Manager programming. Unfortunately you won't be
able to debug WIN32s programs under Windows 3.1: you'll need Windows 95 or NT
for that. Watcom compile speed is effectively very slow because the precompiled
header criteria are too strict to be practical; but this is partially
compensated for by quick link times. So if you're doing small changes
to the source and lots of linking, Watcom is much better than Visual C++.

{\bf Symantec C++} I have no experience of, though some people use it with
wxWindows. It has a famously nice IDE, but so far I've avoided IDEs because
of their lack of flexibility compared with makefiles, and they have a habit
of multitasking badly.

{\bf DJGPP (a port of GNU G++)} is not a contender (see section on this below).

{\bf In summary:} if you want to be reasonably sure of reliability and
compatibility with future versions of Windows and wxWindows, I would
recommend going for Microsoft's VC++ 2.x (a 32-bit compiler with 16-bit
VC++ 1.5 included). Not because I enjoy swelling Microsoft's profits, but
I believe this is the most reliable and hassle-free solution. If your
application doesn't need 32-bits or fancy controls, then you could
do worse than compile 16-bit applications using VC++ 1.5 ensuring that
they'll run under NT/Windows 95; and then make the switch to 32-bit
compilation when you know most people can run your 32-bit applications.

With the hassles implicit in GUI programming, the last thing you need is
an unreliable compiler, so with compiler prices dropping, I'd recommend
that you treat yourself to a decent compiler.

\subsection{Can I use an Integrated Development Environment?}\label{ide}

It is possible to use an existing IDE for writing code, but the
project file settings are tricky and often the IDEs do not allow
the flexibility that wxWindows requires, with its little libraries
tucked away in odd places. By adding enough switches, it should be
possible.

In particular, some IDEs (such as Microsoft's) do not like C++ source
files to have .cpp extensions, so renaming of .cpp files is necessary.


\subsection{How can I use wxWindows with Borland C++?}

Here are some tips for using Borland C++ with wxWindows.

\subsubsection{To create new IDE project files}

\begin{enumerate}
\item The IDE must recognize .cpp files to edit and compile wxWindows
source. You must add .cpp to the extension lists in the
Options/Tools/CppCompile, Options/Tools/EditText and
Options/Environment/Syntax Highlighting menus.
\item Create a new project file with the following settings:
\begin{itemize}
\item Platform: Win16 or Win32 (always the same)
\item Standard libs: No OWL, no Class Lib (unportables)
\item Advanced Ops/No source node
\end{itemize}
\item If the source files already exist, just type Ins on the target name
(*.exe or *.lib) of the project window. The selector file dialog must
appear. Then select all the relevant *.cpp files, and the *.rc and 
*.def files. 

If you are not sure about which files must be included, look into a
makefile, if it exists, for the sources. Don't worry about libraries,
BC4 has its own windows libraries.
\end{enumerate}

\subsubsection{To use old *.prj project files}

The IDE can read version 3.1 *.prj files, and then convert them to the
new format. You must repeat item 1 (above) each time you read
a *.prj file, and sometimes you'll have to select the .cpp files in 
the project window, and drop them on the target file (.exe or .lib), 
in order to obtain little bullets in left of the project window 
(ie. to make the IDE recognize the files). If there are no many .cpp 
files, I recomend to create a new *.ide file.

\subsubsection{To use makefiles}

Now the wxWindows distribution includes Borland makefiles.
The file "makefile.bcc" from \verb$samples\minimal$ directory could
be used as a template.

\subsubsection{To compile wxWindows projects}

Build \verb$wx.lib$ with the supplied makefiles. If you want to use the IDE
for your new applications:

\begin{enumerate}
\item Set the proper directories. In the menu Options/Project/Directories/Include,
add \verb$"(wxdir)\include\base;(wxdir)\include\msw"$, where (wxdir) must be
changed for the wxWindows base directory. Avoid absolute paths 
in your code, if you want it to be portable. 
\item In Options/Messages/Potential C++ errors menu, uncheck
"function f1 hides virtual function f2", to reduce
the amount of message warnings during compiling.
\item A typical Windows application (16bits mode) can't have a data segment
larger than 64k. To avoid 
\rtfsp\helpref{data segment overflow}{dataseg} and linking problems,
set in Options/Project/16-bit Compiler/Memory Model:
\begin{itemize}
\item Memory model: large
\item Asume SS = DS: never
\item Far virtual tables: on
\item Automatic far data: on
\item Far data threshold: 512
\end{itemize}
\end{enumerate}

Example files for minimal example:

\begin{verbatim}
 minimal[.exe]
     -minimal.cpp
     -minimal.rc
     -minimal.def
     -wx.lib
\end{verbatim}

\subsubsection{Borland C++ complains about a missing 1.cpp file.}

Copy wx\_panel.cpp, wx\_utils.cpp, wx\_dialg.cpp, wx\_timer.cpp and
wx\_clipb.cpp to corresponding .cpp files. You need not change the
names in the makefile. BCC will compile the .cpp modules.

Hopefully this workaround will be made redundant at some point
in the future.

\subsubsection{Using GNU wxString/wxRegex with Borland C++}

{\small
\begin{verbatim}
From: leif@danmos.dk
Date: Wed, 2 Apr 97 02:36 WST
To: wxwin-users@babbage.eng.nene.ac.uk
Subject: Borland C++ 5.0 and wxString problem

    Hi Pasquale and others,

On Fri, 21 Mar 1997 Pasquale Foggia wrote:

> leif@danmos.dk writes:
> 	[...]
> > Borland C++/5.0. I am using wxString and wxRegex from 'contrib', which
> > works fine in Windows 3.11 and Linux. Now I get a memory access violation
> > using wxString::Index (wxRegex (" \n\t]+", 0). I have tracked it down to
> 	[...]
> >     I have now tracked the problem down to being a question of signed/
> > unsigned char's. I am using an .ide to generate all the libraries, and
> > I have here marked 'Borland extentions' in the 'Compiler|Source code'
> > option. This seems *not* to set the define __STDC__ which is needed in
> > wxregex.cpp module. I tried to mark 'ANSI' instead, which made quite
> > a bit of difference. Now wxRegex worked :-). BUT, if I compile the
> > whole library with this 'ANSI' marked, I get an error in winreg.h :-(.
> > 
> >     I also tried to compile all of the libraries using the makefile.b32,
> > which generated a library having the first problem (wxRegex not working).
> > 
> 	[...]
> 
> The Borland C (correctly?) doesn't define __STDC__ when you enable the
> Borland extensions, because in this mode there are non-ANSI keywords like
> far (if you are compiling a 16 bit app) or _export that may be non compatible
> with a valid ANSI C program. The problem is that such keywords are needed
> to compile <windows.h>, so you must set the Borland extensions checkbox
> in your project. The solution to your problem is simply putting 
> 
> #if !defined(__STDC__) && defined(__BORLANDC__)
> #define __STDC__ 1
> #endif
> 
> at the beginning of every file that may need it.

    This was really a great help. I had a hard time, however, to track
down which files needed this, until I just tried to put '__STDC__=1' in
Options|Project|Compiler|Defines in the .ide as a general define for
compiling the whole library (it corresponds to the makefile.b32 file).
Then everything worked (and still does :-).

    Maybe it was an idea to put this '#define __STDC__ 1' thing into
the standard makefile.b32 !?


    Thanks,

    Leif Jensen (leif@danmos.dk)
\end{verbatim}
}

\subsubsection{Miscellaneous tips}

If Borland crashes or runs out of memory during a compile, you
may need to configure your DPMI swap file: see P. 21 of the Borland 4.0
User Guide. This is particularly necessary if you only have 8MB of physical
memory.

For precompiled headers to work properly on the wxWindows library,
you will need to uncomment the first

\begin{verbatim}
// #include "wx.h"
\end{verbatim}

line in each file in src/base. Yes, this {\it does} look like some files will
be included three times: but they won't really. The line has to be before
any pragmas of ifdefs, or the precompilation doesn't work.

If you don't recompile wx.lib often, this editing is not necessary.

\hrule
{\it Contributed by Aguilar Sierra Alejandro-CCA \verb$<asierra@servidor.unam.mx>$, Oliver Niedung \verb$<niedung@cip.med-informatik.uni-hildesheim.de>$ and
Andrew Davison \verb$<adavison@ozemail.com.au>$}

\subsection{How can I use wxWindows with Turbo C++ 3.1?}

Here are the settings you need to make in Turbo C++ 3.1:

\begin{verbatim}
  Compiler
   Code Generation:
      Model = Large
      Assume SS=DS = Never
      Defines = wx_msw
      Options = Treate enums as int
                Dup Strings merged
   Advanced Code Generation
      Options = Generate underbars
      Auto Far Data  - Far data threshhold = 4
  Make
    Run Librarian
  C++ Options
    C++ Always
  Link Libraries
    Container Class = none
    Object Windows Lib = none
    Standard Run Time Lib = Static
\end{verbatim}

You will probably need to turn off many components in wx\_setup.h.

\subsection{How can I use wxWindows with Borland C++ 3.1?}

The simple answer is probably ``don't try, get a more recent compiler", but if you really need
to, here are some tips to at least get it compiled, albeit with a GPF when running the samples.

From Daniel Schmitz (dschmitz@ch.hp.com):

\begin{verbatim}
Well, just in case you have users that still have Borland C++ 3.1 AND
windows 3.x...

src/makefile.bcc
    Change the Borland version to 3.1.  I'm not sure this has any
    affect.

wb_timer.cpp
    bcc 3.1 does NOT put underscores before timezone or the other
    variable.

msw/makefile.bcc
    make kept screwing up when recusive making wxstring.  It turns
    out that you pass the command-line option: DEBUG="0"
    make kept complaining that it couldn't find the target 0.  So I
    modified this makefile to pass: -DDEBUG="0"
    and it went through wxstring ok.  I double checked the command
    line options being passed to make.  They were valid, so there must
    be a bug in this version of make.

wx_setup.h
    There aren't any libraries that provide the SQL functions.  Set
    the flag USE_ODBC to 0.

wx.lib
    My stupid stupid stupid... copy of tlib absolutely refused to
    create a library larger than 1MB.  It didn't matter what values I
    gave to the -P option.  I could build a 990K library, then try to
    add one 20K object and it would instantly complain: Out of Memory
    I have 14MB free for tlib to add that extra object file so I don't
    think it's a ram limitation.  I went around this by splitting
    wx.lib into 5 libraries, wx1.lib...wx5.lib.  Before linking sample
    programs, I modified their makefiles to link against all of the wx
    libraries I built.

erase f1 f2 f3...
    The makefile.bcc files in the following contrib directories gave
    multiple filename patterns to the erase command for the clean
    targets.  This is not allowed.
        wxstring, gague, itsybits, fafa.
\end{verbatim}


\subsection{Why do I get "data segment overflow error" when linking using Borland compiler?}\label{dataseg}

You have a number of switches in compiler configuration that may need
adjusting. In order of my own preference:

\begin{enumerate}
\item Make sure that stack is located in a different segment from data.
     It saves you about 16k.

IDE:
\begin{itemize}
\item "Assume DS==SS" - Never, Never and Never! 
\item "Smart callbacks" -Off.
\end{itemize}

Command line:
\begin{itemize}
\item -WE -Fs-
\end{itemize}

DrawBacks: All functions that are exported to Windows must be 
explicitly declared so, and MakeProcInstance() call is neccessary.
(This is done in wxWin, just take care of your own callbacks).

For version 150k you have to uncomment one MakeProcInstance() in
src/msw/wx\_main.c (or wx\_win.c - don't remember exactly). 
The patch is sent to Julian.
   
\item Put virtual tables into code segment.
It saves you 4bytes*per\_virtual\_function\_per\_class. Inherited functions
also count. Result is more than you can expect. 

IDE:
\begin{itemize}
\item "Far virtual tables" - On.
\end{itemize}

Command line: 
\begin{itemize}
\item -Vf
\end{itemize}

Drawbacks: All C++ libraries and objects must be compiled with the
same settings of this switch or you'll get "undefined symbol" link
errors. Put this is in your default configuration file and 
recompile everything to save yourself from trouble later.
\item Turn on automatic far data.

IDE:
\begin{itemize}
\item "Automatic far data" - On,
\item "Far data threshold" - 4.
\end{itemize}

Command line:
\begin{itemize}
\item -Ff=4
\end{itemize}

Drawbacks: you can't run more than one instance of your program
anymore. This is architectural flaw in MS-Windows. If you need to,
make several copies of your .exe under different names and run 
one copy of each.
\item Put constant strings in code segments.

IDE:
\begin{itemize}
\item "Put constant strings in code segments"
\end{itemize}

Command line:
\begin{itemize}
\item -dc
\end{itemize}

Drawbacks: Not available with Borland 3.1. 
You probably wan't be able to modify any of these strings, since
they are in code segments. If windows does not prohibit write
access to code segments, then this will cause hard to find bugs.
(Code segments are marked as discardable and can be thrown away and
reloaded from .exe file at any moment).
(I did not tried it myself yet).
\end{enumerate}

\hrule
{\it Contributed by Alexei Vovenko \verb$<alv@au.edu.dstc>$}

\subsection{My Windows compiler doesn't contain all the required .lib files.}

Some compilers seem to be shipped without libraries such as ddeml.lib, shell.lib,
and commdlg.lib. Use implib to generate corresponding .lib files, e.g.

\begin{verbatim}
implib commdlg.lib c:\windows\system\commdlg.dll
\end{verbatim}

\subsection{Is it possible to use multithreaded library with wxWindows under MS-Windows?}

Yes, if you do it carefully. First of all make sure that your
compiler does not make any unreasonable assumptions about stack
segment, and better still it makes no assumptions about stack segment
at all. Make sure that both the library and your program are compiled 
this way. 

Do not expect wxWindows to be reentrant because MS-Windows is not.   
Most probably your thread implementation does not use preemptive scheduler,
so MS-Windows reentrancy is not a big problem.

Derive your member from "Bool wxApp::OnIdle(void)" 
and put a pthread\_yield() call there. If you use dialogs, take care about
yielding to threads while in a dialog.

\hrule
{\it Contributed by Alexei Vovenko \verb$<alv@au.edu.dstc>$}

\subsection{Can I use wxWindows with Watcom C++?}

Yes, Watcom C++ can be used in WIN386 (32-bit) mode and WIN32s mode. However, there
seems to be a problem running wxWindows programs compiled in 16-bit
mode.

To compile with Watcom in WIN386, set USE\_ODBC to 0 in wx\_setup.h. All
the other settings should work. A fix to allow the ODBC classes to be
used with Watcom (similar to that which allows CTL3D to be used) will
probably be released shortly.

Don't forget to set the INCLUDE and PATH variables appropriate to
the mode of compilation. For WIN386 compilation, these are:

\begin{verbatim}
PATH F:\WATCOM\BIN;F:\WATCOM\BINW;F:\WATCOM\BINB;<the rest of your path>
SET INCLUDE=F:\WATCOM\H;F:\WATCOM\H\win
\end{verbatim}

For WIN32 (NT) target compilation:

\begin{verbatim}
PATH F:\WATCOM\BINNT;F:\WATCOM\BINB;F:\WATCOM\BINW;F:\WATCOM\BIN;<the rest of your path>
SET INCLUDE=F:\WATCOM\H;F:\WATCOM\H\NT
\end{verbatim}

Edit the file wx/src/makewat.env and set the WATCOM, WXDIR and MODE
variables appropriately (see comments).

You will probably want to set up some batch files to switch between
settings, if you are targetting more than one platform. If you can't
find the BINNT directory, you probably didn't check the Windows NT
target option when installing. The naming is confusing, since you're
probably not targetting `genuine NT', but WIN32s.

Under WIN32s (NT) compilation, we unfortunately lose CTL3D and
ODBC capabilities. There may be a way of linking with libraries:
watch this space.
 
When running the hello.exe demo in WIN386, copying a metafile to the clipboard
produces a fairly catastrophic crash (often exiting Windows or
hanging completely). The culprit appears to be wxMetaFile::SetClipboard
and the calls to GlobalAlloc, GlobalLock and the subsequent memory access.
All the other functionality of hello.exe appears to work under Watcom C++,
and the crash does {\it not} occur in WIN32s mode.

Results with some of the other samples in WIN386 and WIN32s modes:

\begin{itemize}\itemsep=0pt
\item The animate sample doesn't show the animation in WIN386 mode,
but it works in WIN32 mode.
\item The toolbar sample works fine in WIN386 and WIN32 modes.
\item The buttnbar sample crashes Windows in WIN386 mode (the underlying buttonbar
code is 16-bit Windows code taken from a Microsoft sample: probably
needs sanitizing). It works fine in WIN32 mode.
\item The dialogs sample works fine in WIN386 and WIN32 modes.
\item The docview sample crashes Windows when print-previewing in WIN386 mode.
It works fine in WIN32 mode.
\item The form sample works fine in WIN386 and WIN32 modes.
\item The fractal sample works fine in WIN386 and WIN32 modes.
\item The ODBC sample can't be compiled yet since we can't link 32-bit
apps with 16-bit ODBC.
\item The layout sample works fine in WIN386 and WIN32 modes.
\item The mdi sample works fine in WIN386 and WIN32 modes.
\item The minimal sample works fine in WIN386 mode and WIN32 modes.
\item The types sample works fine in WIN386 and WIN32 modes.
\item The panel sample crashes windows in WIN386 mode. I suspect the gauge code, again
Microsoft-derived 16-bit code. It works fine in WIN32 mode.
\end{itemize}

\subsection{How can I compile PROLOGIO with Borland C++?}

Borland users have been finding that it's impossible to compile
the LEX and YACC-generated parser files in PROLOGIO from wxWindows
1.50. PROLOGIO is required for compiling wxBuilder.

Peter Trattler and Edward Zimmermann have improved PROLOGIO to allow
FLEX to be optionally used instead of LEX; see the PROLOGIO manual for
more details. If you have FLEX (the default Linux lexical analyser
generator), you can generate a lex\_yy.c which compiles more cleanly.

LEX and FLEX-generated versions of lex\_yy.c are supplied, as
lexyy.c and flexyy.c respectively.

FLEX and YACC are freely available for DOS; most major DOS ftp
sites should have them.

Remember {\it not} to compile lex\_yy.c explicitly. It's included by y\_tab.c.

\subsection{I get a compilation error in wb\_item.cpp using Borland C++.}

In release 1.50 k there's a bug: edit wb\_item.cpp, go down
to the wxbRadioBox constructor and replace

\begin{verbatim}
#ifndef _turboc
\end{verbatim}

with 

\begin{verbatim}
#ifndef __BORLANDC__
\end{verbatim}

\subsection{Can I use DJGPP with wxWindows?}

No. Although there is now a Windows 3.1 version of RSXDK which
allows limited 32-bit Windows compilation under Windows using
DJGPP, it doesn't have enough functionality to be useful for
wxWindows. This has been checked out by Arjen Duursma (arjen@capints.uucp).

\subsection{Can I use GNU-WIN32 with wxWindows?}

Yes. Thanks to Keith Garry Boyce, from 1.66E onwards wxWindows supports
GNU-WIN32, with a few patches to the compiler headers. See install.txt for more
details.

\subsection{Can I make a DLL out of wxWindows to reduce executable size?}\label{windowsdll}

Unfortunately, this is not yet possible. However, an attempt has been made
for VC++ 4.0. The following reproduces the file docs/dll.txt.

{\small
\begin{verbatim}

 Attempt to create a DLL of wxWindows using VC++ 4.0
 ---------------------------------------------------
 
 The wxWindows source has been changed for a nearly (!) successful
 attempt to create a DLL out of wxWindows, in order to substantially reduce
 executable size (and link time).
 
 Perhaps someone else will be able to fathom what I'm doing wrong.
  
 To compile for DLL
 ------------------
 
 Edit wx_setup.h and src/msw/makefile.unx and set
 MINIMAL_WXWINDOWS_SETUP to 1 in both files.
  
 Compiling library:
 
 > cd c:\wx\src\msw
 > nmake -f makefile.nt dll     ; Makes c:\wx\lib\wx166.dll/.lib
 > nmake -f pch                 ; Makes dummy.obj and wx.pch for apps
                                ; to use

 Compiling samples:
 
 Edit samples/minimal/makefile.nt and set WXUSINGDLL to 1 just before
 including ntwxwin.mak.
 
 > nmake -f makefile.nt         ; Make minimal.exe
 > copy c:\wx\lib\wx166.dll .
 > minimal.exe
 
 Similarly for hello.
 
 What goes wrong
 ---------------
 
 When running minimal.exe (which is a gratifying 80K or so in size),
 the button and message do not appear.
 
 On tracing the code, it seems that in SetSize, a call to
 GetWindowText to get the message or button label, doesn't work.
 It copies zero characters to the buffer. Therefore the size
 is miscalculated and the height is zero - so the items don't show.
 The extended error code is 120, which means 'This function is
 only valid in Win32 mode', according to winerror.h.
 
 The mystery is why this function should not work. Hunting through
 on-line help, there is a comment about Set/GetWindowText not working
 across applications (but we're not using it across apps) and
 something in the Knowledge Base about SetWindowText not working properly
 across threads. Nothing very relevant for DLL usage.
 
 When running hello.exe, there's a kernel GPF when calling the
 constructor for the subwindow. Unfortunately there's absolutely
 no indication of what the error might be caused by.
 
 Notes on making the DLL
 -----------------------
 
 I've added a WXDLLEXPORT definition (defined in wx_defs.h) and
 sprinkled class and function declarations liberally with this
 keyword, as per VC++ documentation. This changes according to
 whether the DLL is being made (WXMAKINGDLL defined) or being
 used by an app (WXUSINGDLL defined).
 
 For an app to indicate that it's using the DLL version instead
 of the statically linked version, you just need to put
 
   WXUSINGDLL=1
   
 in the makefile _before_ you include ntwxwin.mak.
 
 Note that the symbols defined in wb_data.obj remain undefined
 when linking with wx166.lib, so I've had to duplicate them
 in dummy.cpp, whose object will be linked to the application in order
 to conform with precompiled header rules. Perhaps someone can
 find a way of eliminating this requirement - possibly the module
 is not linked properly because it doesn't have any functions in it?
 Similarly, WinMain has to be defined in an object linked statically
 to the app, so this is in dummy.cpp too.
 
 If anyone can finish the job, I'll be delighted and will ensure that
 future versions of wxWindows will be DLL-friendly, at least for
 32-bit VC++.
 
 Good luck!
 
 Julian Smart
 2nd October 1996
\end{verbatim}
}

\chapter{C++ issues}

\section{How can I have class member functions as callbacks
for buttons?}

Here's some correspondence on the subject.

\verbatiminput{members.txt}
%
\section{How can I display debugging messages?}

If you use the function wxDebugMsg, messages will be displayed on either
the standard error stream (X) or the debugging stream (Windows). To read
these messages under Windows, you must either be running a debugger,
or (perhaps more convenient), using a program such as Microsoft's DBWIN
which displays these debugging messages in a text window.

Using DBWIN on a regular basis has the advantage of showing up some GDI
errors that you might otherwise miss, but that could cause severe
problems in future. DBWIN is available in the Windows SDK or
on the Developer's Network CD-ROM, and probably by ftp from Microsoft's
site.

DBWIN doesn't work under Windows 95 or Windows NT, unfortunately.

wxWindows supports debug logs: see the documentation for wxDebugContext.

\chapter{Platforms}

\section{Is there a Mac version of wxWindows under development?}

There is a volunteer port in progress and alphas can be downloaded from the
AIAI ftp site. An official AIAI Mac port is due to be completed in early July 1996.

\section{Is there an X version of wxBuilder?}

Sort of... the XView version needs some work, particularly because there
are occasions when 2 levels of modal dialogs are used, which XView doesn't
like.

The Motif version is coming on well and will be released (a binary for Suns and HPs)
by the end of September.

\chapter{Run-time problems}

\section{How do I install CTL3DV2.DLL correctly?}

A program that uses CTL3DV2.DLL must be installed so that
the DLL is in the windows/system directory, and NOT in
the application directory, or it will not run correctly.

It is tempting to copy CTL3DV2.DLL into as many directories
as possible to hedge your bets. Unfortunately this is exactly
the wrong thing to do: the DLL must be in windows/system, ONLY.

CTL3D.DLL is not required if you're linking with CTL3DV2.LIB:
it's the old version.

\section{Why does my program exit abnormally when initializing?}

You may be declaring a pen, brush, icon, cursor or colour globally.
These objects automatically add themselves to global lists which may
not be initialized before the object constructors are called, and so
only global pointers to these objects may be declared. After or during
\rtfsp{\bf OnInit} is called, these objects may be created with impunity.

\section{After using memory DCs and bitmaps under Windows, I get system crashes.}

Here's some correspondence that helps explain this phenomenon and its
solution.

\begin{verbatim}
From: Markus Meisinger <Markus.Meisinger@at.ac.uni-linz.risc>
Message-Id: <199406241203.AA09143@melmac.risc.uni-linz.ac.at>
To: wxwin-users@ed.aiai
Subject: SOLUTION to strange problems of yesterday
Status: RO

hi wxWindows programers

i wrote yesterday:

>I have encountered some strange problems with the sample programs under
>MS-WINDOWS 3.11. (they work fine under UNIX/LINUX)
>
>The problems arise if i e.g quit the minimal sample program. During the run
>of the program no problems appear. But after it stops than the whole
>system crashes (not always, but quite often :)!
>
>If the system crashes, not only the sample program produces an
>UNRECOVERABLE APPLICATION ERROR, but also the most other programs running
>at the same time (PROGMAN, FILEMAN, WINMETER, ...) stop their work with
>a severe error. :(
>
>Any help would be greatly appreciated! Or is there a patch available, because
>a guess their is a bug in freeing the system resources ...

The actual problem causing the strange behaviour was a buggy wxWindows program
of mine running some minutes before and not the sample programs, CTL3D or
FAFALIB

Because i used a wxMemoryDC in which i selected a dynamicaly created bitmap.
During the quiting process i freed this bitmap which was selected in the
wxMemoryDC. But wxWindows also freed this bitmap during deleting the DC and
... the rest you know

SO, BE AWARE IF YOU DELETE BITMAPS! 

Never delete a bitmap which is selected into a DC if the DC will be deleted.
The bitmap will be deleted by the DC!
But i did it !!! :) and it resulted in a system wide crash of MS-WINDOWS
(in Motif it works fine (why?))
The funny thing with this crash is that the system doesn't immediately crash,
instead it works fine for a few minutes until ANY other program gets into
some memory conflict. After this application error each program still running
complains problems with the memory and stops. The last thing you see is
the standard windows background color (if you are lucky:)
I never saw such a system wide crash WOU :) in some sense it was funny :)

Hope you don't make the same errors as i did!

Markus
\end{verbatim}

\section{Why does my canvas not have the keyboard focus under Windows?}

Under MS Windows, if you have more than one subwindow per frame,
you need to call wxWindow::SetFocus explicitly, or wxWindows will not know
which subwindow needs the focus. A good place to do this is in wxFrame::OnActivate.

\section{Why do panel items not size or position correctly under Motif?}

Absolute positioning in Motif doesn't mix well with constraint-based positioning,
as used by wxWindows to implement left-to-right, top-to-bottom layout.

If you know that all your widgets are going to be positioned and sized explicitly,
you should pass the style flag {\bf wxABSOLUTE\_POSITIONING} to the panel constructor,
in which case a Motif bulletin board widget will be used instead of a form widget.

{\it NOTE:} from wxWindows 1.60, the wxABSOLUTE\_POSITIONING flag is
obsolete, because all Motif positioning is now done this way.

One reason for a widget not appearing can be that the widget
is sized just too big for the panel, and then Motif gets very
confused. Try reducing the size of the widget.

Another possibility is to try setting the widget sizes first, and
finally setting the panel size.

Widgets that are too close together may cause problems.

\subsection{Possible cure for some Motifs}

Some Motifs under some platforms appear to have very bad panel item
problems in dialogs. Once cure seems to be to change the resize policy
in src/x/wx\_dialg.cpp: find where two lines are marked TROUBLE SPOT (from 1.61 beta 2),
in the 'else' clause of the 'invisibleResize' condition. One line has XmRESIZE\_ANY,
the second has XmRESIZE\_NONE.

Comment out or uncomment the first line, and uncomment or comment out
the second line (whichever is not already the case). However, doing this
seems to make a dialog box's size change dynamically when (for example)
a listbox next to the dialog box edge is updated; so only change this if
absolutely necessary.

\section{Under Motif, the status line does not appear.}

The fix seems to be to call Fit before calling CreateStatusLine,
or try calling SetSize after frame creation.

\section{Under Windows, dialog boxes refuse to appear. Why?}

You probably forgot to include the file wx.rc in your resource script
This is required for dialog boxes to work correctly.

\section{Under Motif, quitting windows from the File menu causes a crash.}

There is a bug in the X toolkit that manifests itself on some platforms.

Here's a bug fix from Jim Huggins:

\begin{verbatim}
I've been investigating this problem for the last few
weeks.  (where the problem is:  closing child window
in motif causes core dump.)

The core dump stack is usually:

>0  0x3ff80c5f8e4 in InSharedMenupaneHierarchy() RowColumn.c:6759
#1  0x3ff80c5fc44 in SetCascadeField() RowColumn.c:6849
#2  0x3ff80c6d924 in MenuProcedureEntry() RowColumn.c:11685
#3  0x3ff80bd0a88 in Destroy() CascadeBG.c:2350
#4  0x3ff803c9d3c in Phase2Destroy() Destroy.c:113
#5  0x3ff803c9c04 in Recursive() Destroy.c:72
#6  0x3ff803c9b94 in Recursive() Destroy.c:60
#7  0x3ff803ca0b8 in XtPhase2Destroy() Destroy.c:205
#8  0x3ff803ca220 in _XtDoPhase2Destroy() Destroy.c:252
#9  0x3ff803d1a90 in XtDispatchEvent() Event.c:1127
#10 0x120026b7c in ((wxApp*)0x1400011b8)->MainLoop() wx_main.cpp:177

>From what I can tell, this is a bug in an older version
of Xt library version X11R5.  I think there was a patch
for it but I have no idea what number or when it was produced.

The bug has to do with recursive destruction of Menu widgets.
(I don't know too much detail...)

I tried posting a note to comp.windows.x.intrinsics and didn't
get any working fixes or explanations.  (My idea about a Xt bug
and possible patch comes from a note I found in an internal DEC
notes conference.)

I found a work around for this problem in wxWindows though:
in file srx/x/wx_item.cpp,

add the follow code to the bottom of function wxMenu::~wxMenu(void)
(I think around line 1070)

#ifdef wx_motif
  if (handle) {
      DestroyChildren();
      wxWidgetHashTable->Delete((long)handle);
      Widget w = (Widget)handle;
//     XtDestroyWidget(w);
      handle = NULL;
  }
#endif

This is basically doing what the ~wxWin function would have done
except the XtDestroyWidget(w) has been commented out to avoid
the Xt bug.  (i.e. if you uncomment that line above, the core dump
will occur.)

Jim
\end{verbatim}

\section{Under XView, I get a SERVER\_IMAGE\_BITMAP\_FILE warning mesage.}

{\tt XView warning: SERVER\_IMAGE\_BITMAP\_FILE: Server image creation failed (Server Image package)}

The application may be looking for an icon file which does not exist or
has been referenced by a relative icon pathname. Use an absolute path, or
include the icon image in the source file using conditional compilation
(see the reference manual's {\bf wxIcon} class entry).

\section{Under Windows, MDI child windows don't size properly.}

For some reason, it's not possible to programmatically resize an MDI
child frame just after creation, so using {\it Fit} to size an MDI
child frame around a subwindow does not work.

Also, it does not seem to be possible (on creation at least) to hide an
MDI child frame, resize its subwindows, then show it.

If you are using {\it Fit} to size the child window, you must
put up with seeing the windows repaint themselves: the child
window will come up already visible.

However, if you know the size of the window in advance, give it an
absolute size, create it with the wxMINIMIZE window style, draw
the contents, then call {\it Iconize(FALSE)} to restore the
window. This at least avoids seeing half-painted windows whilst
things get initialized.

\section{Functions that return string values cause strange behaviour
on some platforms.}

To resolve the question of who is responsible for allocating and
deallocating memory, wxWindows maintains a policy (unless
documented otherwise) of returning a {\it temporary} pointer
to a static string buffer from functions such as {\it wxText::GetValue}.

If you don't immediately take a copy of this value, it's possible that
subsequent wxWindows calls will use the same memory, causing
unpredictable results. This tends to be more the case under Windows than
other platforms, where pointers to internal XView or Motif widget
strings are often returned.

\chapter{What you can't do in wxWindows}

There are many exceptions and provisos in the wxWindows API, mostly
because the individual platforms don't support some functionality,
or it hasn't yet been implemented, or some other reason. These
exceptions are being eliminated where possible, but inevitably
some will remain.

This section will start to give an at-a-glance guide to what {\it not}
to try, and perhaps save some grief.  It is not a complete list though.

\begin{description}
\item[Fonts in panel items.] XView doesn't allow you to
set panel item fonts individually.
\item[wxTextWindow::OnChar.] XView doesn't allow interception
of character input in a text subwindow.
\item[OnSetFocus, OnKillFocus.] Not called for panel items (yet).
May not be called for other windows either; beware.
\item[Bitmaps and arcs in PostScript device context.] The PostScript device
context does not support drawing bitmaps, or arcs.
\item[Menu items cannot be deleted dynamically.] Dynamic menu item deletion
has not been implemented. It is not technically infeasible on any known
platform, though.
\item[Custom cursor creation.] Not yet implemented.
\item[Bitmap buttons.] Bitmaps loaded dynamically from .BMP or .GIF
files under UNIX, cannot be used for bitmap buttons yet. XBM and XPM
files should work fine meanwhile.
\item[OLE-2.] Not supported; an OLE++ project has been started but is currently in limbo.
\end{description}

\begin{comment}
\begin{helpglossary}
\setheader{{\it GLOSSARY}}{}{}{}{}{{\it GLOSSARY}}%
\setfooter{\thepage}{}{}{}{}{\thepage}%

\gloss{API}

Application Programmer's Interface - a set of calls and classes defining
how a library (in this case, wxWindows) can be used.

\gloss{Canvas}

A canvas in XView and wxWindows is a subwindow on which graphics (but
not panel items) can be drawn. It may be scrollable. A canvas has a
{\it device context}\/ associated with it.

\gloss{DDE}

Dynamic Data Exchange - Microsoft's interprocess communication protocol.
wxWindows provides an abstraction of DDE under both Windows and UNIX.

\gloss{Device context}

A device context is an abstraction away from devices such as windows,
printers and files. Code that draws to a device context is generic since
that device context could be associated with a number of different real
device. A canvas has a device context, although duplicate graphics calls
are provided for the canvas, so the beginner doesn't have to think in
terms of device contexts when starting out. wxWindows supports device
contexts for canvas, Windows printer, and Encapsulated PostScript files on UNIX.

\gloss{Dialog box}

In wxWindows a dialog box is a convenient way of popping up a window
with panel items, without having to explicitly create a frame and a
panel. A dialog box may be modal or modeless. A modal dialog does not
return control back to the calling program until the user has dismissed
it, and all other windows in the application are disabled until the
dialog is dismissed.  A modeless dialog is just like a normal window in
that the user can access other windows while the dialog is displayed.

\gloss{Frame}

Under XView (and wxWindows), a visible window usually consists of a
frame which contains zero or more subwindows, such as text subwindow,
canvas, and panel. Under Windows 3, windows can be nested arbitrarily,
but this is not currently supported in wxWindows.

\gloss{GNU C++}

A free, solid C++ compiler which may be used to compile the UNIX side of
wxWindows applications.

\gloss{GUI}

Graphical User Interface, such as Windows 3 or X.

\gloss{Menu bar}

A menu bar is a series of labelled menus, usually placed near the top
of a window. It is popular in Windows 3 and Motif applications, and as
such is a supported feature, but wxWindows has to `simulate' the menu
bar under XView by using a panel and several menu buttons.

\gloss{Metafile}

Microsoft Windows-specific object which may contain a restricted set of
GDI primitives. It is device independent, since it may be scaled without
losing precision, unlike a bitmap. A metafile may exist in a file or in
memory. wxWindows implements enough metafile functionality to use it to
pass graphics to other applications via the clipboard.

\gloss{Open Look}

A specification for a GUI `look and feel', initiated by Sun
Microsystems. XView is one toolkit for writing Open Look applications
under X, and wxWindows sits on top of XView.

\gloss{Panel}

A panel in XView and wxWindows terminology is a subwindow on which a
limited range of panel items (widgets or controls for user input) can be
placed. wxWindows allows panel items to be placed explicitly, or laid
out from left to right, top to bottom, which is a more platform
independent method since spacing is calculated automatically at run time.
Panel items cannot be placed on a canvas, which is specifically for
drawing graphics.

\gloss{RPC}

Remote Procedure Call - a method of interprocess communication akin to
procedure call, where the client process makes a call to a server, which
sends back a result. The AIAI-supplied PROLOGIO library supports a
simple RPC protocol based on DDE (but working under both UNIX and
Windows).

\gloss{Status line}

A status line is often found at the base of a window, to keep the user
informed (for instance, giving a line of description to menu items, as in the
{\bf hello} demo). XView has a status line (or footer) capability, but
wxWindows implements the feature explicitly under Windows 3 and Motif.

\gloss{Text subwindow}

In XView and Motif, a text subwindow is supplied for displaying and
retrieving text from a window. It has a rich set of features, only a
small subset of which is currently catered for by wxWindows.

\gloss{XView}

An X toolkit supplied by Sun Microsystems, initially just for porting
SunView applications to X, but which has become a popular toolkit in its
own right due to its simplicity of use. XView implements Sun's Open Look
`look and feel' for X, but is not the only toolkit to do so.

\end{helpglossary}
\end{comment}
\newpage
\addcontentsline{toc}{chapter}{Index}
\printindex
\end{document}
