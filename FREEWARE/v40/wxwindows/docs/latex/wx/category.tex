\chapter{Classes by category}\label{classesbycat}
\setheader{{\it CHAPTER \thechapter}}{}{}{}{}{{\it CHAPTER \thechapter}}%
\setfooter{\thepage}{}{}{}{}{\thepage}%

A classification of wxWindows classes by category.

\section{Managed windows}

There are several types of window that are directly controlled by the
window manager (such as MS Windows, or the Motif Window Manager).
Frames may contain \helpref{subwindows}{catsubwindows}, and
dialog boxes have their own built-in subwindow similar to a panel.

\begin{itemize}\itemsep=0pt
\item \helpref{wxFrame}{wxframe}
\item \helpref{wxDialogBox}{wxdialogbox}
\item \helpref{wxEnhDialogBox}{wxenhdialogbox}
\end{itemize}

See also \helpref{wxWindow}{wxwindow}.

\section{Subwindows}\label{catsubwindows}

Subwindows should be created as children of frames. The panel
subwindow may contain panel items (controls or widgets).

\begin{itemize}\itemsep=0pt
\item \helpref{wxCanvas}{wxcanvas}
\item \helpref{wxPanel}{wxpanel}
\item \helpref{wxTextWindow}{wxtextwindow}
\item \helpref{wxToolBar}{wxtoolbar}
\item \helpref{wxButtonBar}{wxbuttonbar}
\item \helpref{wxSplitterWindow}{wxsplitterwindow}
\end{itemize}

See also \helpref{wxWindow}{wxwindow}.

\section{Common dialogs}\label{catcommondialogs}

\overview{Overview}{commondialogsoverview}

Common dialogs are ready-made dialog classes or functions.

\begin{itemize}\itemsep=0pt
\item \helpref{wxColourDialog}{wxcolourdialog}
\item \helpref{wxFileSelector}{wxfileselector}
\item \helpref{wxGetMultipleChoice}{wxgetmultiplechoice}
\item \helpref{wxGetSingleChoice}{wxgetsinglechoice}
\item \helpref{wxGetSingleChoiceIndex}{wxgetsinglechoiceindex}
\item \helpref{wxGetSingleChoiceData}{wxgetsinglechoicedata}
\item \helpref{wxGetTextFromUser}{wxgettextfromuser}
\item \helpref{wxFontDialog}{wxfontdialog}
\item \helpref{wxPrintDialog}{wxprintdialog}
\item \helpref{wxMessageBox}{wxmessagebox}
\end{itemize}

See also \helpref{wxDialogBox}{wxdialogbox}.

\section{Panel items}

These are widgets (in Motif terminology) or controls (in MS Windows terminology) that can be
placed on panels and dialog boxes, with the exception of wxMenu and wxMenuBar.

\begin{itemize}\itemsep=0pt
\item \helpref{wxButton}{wxbutton}
\item \helpref{wxCheckBox}{wxcheckbox}
\item \helpref{wxChoice}{wxchoice}
\item \helpref{wxComboBox}{wxcombobox}
\item \helpref{wxGauge}{wxgauge}
\item \helpref{wxGroupBox}{wxgroupbox}
\item \helpref{wxItem}{wxitem}
\item \helpref{wxListBox}{wxlistbox}
\item \helpref{wxMultiText}{wxmultitext}
\item \helpref{wxMenu}{wxmenu}
\item \helpref{wxMenuBar}{wxmenubar}
\item \helpref{wxMessage}{wxmessage}
\item \helpref{wxRadioBox}{wxradiobox}
\item \helpref{wxRadioButton}{wxradiobutton}
\item \helpref{wxSlider}{wxslider}
\item \helpref{wxText}{wxtext}
\end{itemize}

See also \helpref{wxWindow}{wxwindow}.

\section{Window layout}

\overview{Overview}{constraintsoverview}

These are the classes and functions relevant to using automated window layout.

\begin{itemize}\itemsep=0pt
\item \helpref{wxIndividualLayoutConstraint}{wxindividuallayoutconstraint}
\item \helpref{wxLayoutConstraints}{wxlayoutconstraints}
\item \helpref{wxWindow::Layout}{wxwindowlayout}
\item \helpref{wxWindow::SetConstraints}{wxwindowsetconstraints}
\item \helpref{wxWindow::GetConstraints}{wxwindowgetconstraints}
\end{itemize}

\section{Device contexts}

\overview{Overview}{dcoverview}

Device contexts are surfaces that may be drawn on, and provide an
abstraction that allows parameterisation of your drawing code
by passing different device contexts.

\begin{itemize}\itemsep=0pt
\item \helpref{wxCanvasDC}{wxcanvasdc}
\item \helpref{wxDC}{wxdc}
\item \helpref{wxMemoryDC}{wxmemorydc}
\item \helpref{wxMetaFileDC}{wxmetafiledc}
\item \helpref{wxPanelDC}{wxpaneldc}
\item \helpref{wxPostScriptDC}{wxpostscriptdc}
\item \helpref{wxPrinterDC}{wxprinterdc}
\end{itemize}

\section{Graphics device interface}

\overview{Bitmaps overview}{wxbitmapoverview}

These classes are related to the Graphics Device Interface, in MS Windows terminology.

\begin{itemize}\itemsep=0pt
\item \helpref{wxColour}{wxcolour}
\item \helpref{wxBitmap}{wxbitmap}
\item \helpref{wxBrush}{wxbrush}
\item \helpref{wxBrushList}{wxbrushlist}
\item \helpref{wxCursor}{wxcursor}
\item \helpref{wxFont}{wxfont}
\item \helpref{wxFontList}{wxfontlist}
\item \helpref{wxIcon}{wxicon}
\item \helpref{wxPen}{wxpen}
\item \helpref{wxPenList}{wxpenlist}
\item \helpref{wxColourMap}{wxcolourmap}
\end{itemize}

\section{Events}

\overview{Overview}{eventhandlingoverview}

Some member functions that an application overrides are passed
event objects containing information about the event.

\begin{itemize}\itemsep=0pt
\item \helpref{wxCommandEvent}{wxcommandevent}
\item \helpref{wxEvent}{wxevent}
\item \helpref{wxKeyEvent}{wxkeyevent}
\item \helpref{wxMouseEvent}{wxmouseevent}
\end{itemize}

\section{Data structures}

These are the data structure classes offered by wxWindows.

\begin{itemize}\itemsep=0pt
\item \helpref{wxDate}{wxdate}
\item \helpref{wxHashTable}{wxhashtable}
\item \helpref{wxList}{wxlist}
\item \helpref{wxNode}{wxnode}
\item \helpref{wxObject}{wxobject}
\item \helpref{wxString}{wxstring}
\item \helpref{wxStringList}{wxstringlist}
\end{itemize}


\section{Run-time class information system}

\overview{Overview}{runtimeclassoverview}

wxWindows supports run-time manipulation of class information, and dynamic
creation of objects given class names.

\begin{itemize}\itemsep=0pt
\item \helpref{wxClassInfo}{wxclassinfo}
\item \helpref{wxObject}{wxobject}
\item \helpref{Macros}{macros}
\end{itemize}

\section{Debugging features}

\overview{Overview}{debuggingoverview}

wxWindows supports some aspects of debugging an application through
classes, functions and macros.

\begin{itemize}\itemsep=0pt
\item \helpref{wxDebugContext}{wxdebugcontext}
\item \helpref{wxDebugStreamBuf}{wxdebugstreambuf}
\item \helpref{wxObject}{wxobject}
\item \helpref{wxTrace}{wxtrace}
\item \helpref{wxTraceLevel}{wxtracelevel}
\item \helpref{WXDEBUG\_NEW}{debugnew}
\item \helpref{WXTRACE}{trace}
\item \helpref{WXTRACELEVEL}{tracelevel}
\end{itemize}

\section{Interprocess communication}

\overview{Overview}{ipcoverview}

wxWindows provides a simple interprocess communications facilities
based on DDE.

\begin{itemize}\itemsep=0pt
\item \helpref{wxClient}{wxclient}
\item \helpref{wxConnection}{wxconnection}
\item \helpref{wxHelpInstance}{wxhelpinstance}
\item \helpref{wxServer}{wxserver}
\end{itemize}

\section{Document/view framework}

\overview{Overview}{docviewoverview}

wxWindows supports a document/view framework which provides
housekeeping for a document-centric application.

\begin{itemize}\itemsep=0pt
\item \helpref{wxDocument}{wxdocument}
\item \helpref{wxView}{wxview}
\item \helpref{wxDocTemplate}{wxdoctemplate}
\item \helpref{wxDocManager}{wxdocmanager}
\item \helpref{wxDocChildFrame}{wxdocchildframe}
\item \helpref{wxDocParentFrame}{wxdocparentframe}
\item \helpref{wxTransferFileToStream}{wxtransferfiletostream}
\item \helpref{wxTransferStreamToFile}{wxtransferstreamtofile}
\end{itemize}

\section{Printing framework}

\overview{Overview}{printingoverview}

A printing and previewing framework is implemented to
make it relatively straighforward to provide document printing
facilities.

\begin{itemize}\itemsep=0pt
\item \helpref{wxPreviewFrame}{wxpreviewframe}
\item \helpref{wxPreviewCanvas}{wxpreviewcanvas}
\item \helpref{wxPreviewControlBar}{wxpreviewcontrolbar}
\item \helpref{wxPrintData}{wxprintdata}
\item \helpref{wxPrintDialog}{wxprintdialog}
\item \helpref{wxPrinter}{wxprinter}
\item \helpref{wxPrinterDC}{wxprinterdc}
\item \helpref{wxPrintout}{wxprintout}
\item \helpref{wxPrintPreview}{wxprintpreview}
\end{itemize}

\section{Database classes}

\overview{Database classes overview}{odbcoverview}

wxWindows provides a set of classes for accessing Microsoft's ODBC (Open Database Connectivity)
product.

\begin{itemize}\itemsep=0pt
\item \helpref{wxDatabase}{wxdatabase}
\item \helpref{wxQueryCol}{wxquerycol}
\item \helpref{wxQueryField}{wxqueryfield}
\item \helpref{wxRecordSet}{wxrecordset}
\end{itemize}

\section{Miscellaneous}

\begin{itemize}\itemsep=0pt
\item \helpref{wxApp}{wxapp}
\item \helpref{wxForm}{wxform}
\item \helpref{wxIntPoint}{wxintpoint}
\item \helpref{wxPathList}{wxpathlist}
\item \helpref{wxPoint}{wxpoint}
\item \helpref{wxTimer}{wxtimer}
\item \helpref{wxTypeTree}{wxtypetree}
\end{itemize}

\section{wxString member functions}\label{wxstringcategories}

\overview{Overview}{wxstringoverview}

This section describes categories of \helpref{wxString}{wxstring} class
member functions.

\subsection{Assigment}

\begin{itemize}\itemsep=0pt
\item \helpref{wxString::operator $=$}{wxstringoperatorassign}\\
\end{itemize}

\subsection{Classification}

\begin{itemize}\itemsep=0pt
\item \helpref{wxString::IsAscii}{wxstringIsAscii}
\item \helpref{wxString::IsWord}{wxstringIsWord}
\item \helpref{wxString::IsNumber}{wxstringIsNumber}
\item \helpref{wxString::IsNull}{wxstringIsNull}
\item \helpref{wxString::IsDefined}{wxstringIsDefined}
\end{itemize}

\subsection{Comparisons (case sensitive and insensitive)}

\begin{itemize}\itemsep=0pt
\item \helpref{wxString::CompareTo}{wxstringCompareTo}
\item \helpref{Compare}{wxstringCompare}
\item \helpref{FCompare}{wxstringFCompare}
\item \helpref{Comparisons}{wxstringComparison}
\end{itemize}

\subsection{Composition and Concatenation}

\begin{itemize}\itemsep=0pt
\item \helpref{wxString::operator $+=$}{wxstringPlusEqual}
\item \helpref{wxString::Append}{wxstringAppend}
\item \helpref{wxString::Prepend}{wxstringPrepend}
\item \helpref{wxString::Cat}{wxstringCat}
\item \helpref{operator $+$}{wxstringoperatorplus}
\end{itemize}

\subsection{Constructors/Destructors}

\begin{itemize}\itemsep=0pt
\item \helpref{wxString::wxString}{wxstringconstruct}
\item \helpref{wxString::~wxString}{wxstringdestruct}
\end{itemize}

\subsection{Conversions}

\begin{itemize}
\item \helpref{wxString::operator const char *}{wxstringoperatorconstcharpt}
\item \helpref{wxString::Chars}{wxstringChars}
\item \helpref{wxString::GetData}{wxstringGetData}
\end{itemize}

\subsection{Deletion/Insertion}

\begin{itemize}\itemsep=0pt
\item \helpref{wxString::Del}{wxstringDel}
\item \helpref{wxString::Remove}{wxstringRemove}
\item \helpref{wxString::Insert}{wxstringInsert}
\item \helpref{Split}{wxstringSplit}
\item \helpref{Join}{wxstringJoin}
\end{itemize}

\subsection{Duplication}

\begin{itemize}\itemsep=0pt
\item \helpref{wxString::Copy}{wxstringCopy}
\item \helpref{wxString::Replicate}{wxstringReplicate}
\end{itemize}

\subsection{Element access}

\begin{itemize}\itemsep=0pt
\item \helpref{wxString::operator[]}{wxstringoperatorbracket}
\item \helpref{wxString::operator()}{wxstringoperatorparenth}
\item \helpref{wxString::Elem}{wxstringElem}
\item \helpref{wxString::Firstchar}{wxstringFirstchar}
\item \helpref{wxString::Lastchar}{wxstringLastchar}
\end{itemize}

\subsection{Extraction of Substrings}

\begin{itemize}\itemsep=0pt
\item \helpref{wxString::At}{wxstringAt}
\item \helpref{wxString::Before}{wxstringBefore}
\item \helpref{wxString::Through}{wxstringThrough}
\item \helpref{wxString::From}{wxstringFrom}
\item \helpref{wxString::After}{wxstringAfter}
\item \helpref{wxString::SubString}{wxstringSubString}
\end{itemize}

\subsection{Input/Output}

\begin{itemize}\itemsep=0pt
\item \helpref{wxString::sprintf}{wxstringsprintf}
\item \helpref{wxString::operator \cinsert}{wxstringoperatorout}
\item \helpref{wxString::operator \cextract}{wxstringoperatorin}
\item \helpref{wxString::Readline}{wxstringReadline}
\end{itemize}

\subsection{Searching/Matching}

\begin{itemize}\itemsep=0pt
\item \helpref{wxString::Index}{wxstringIndex}
\item \helpref{wxString::Contains}{wxstringContains}
\item \helpref{wxString::Matches}{wxstringMatches}
\item \helpref{wxString::Freq}{wxstringFreq}
\item \helpref{wxString::First}{wxstringFirst}
\item \helpref{wxString::Last}{wxstringLast}
\end{itemize}

\subsection{Substitution}

\begin{itemize}\itemsep=0pt
\item \helpref{wxString::GSub}{wxstringGSub}
\item \helpref{wxString::Replace}{wxstringReplace}
\end{itemize}

\subsection{Status}

\begin{itemize}\itemsep=0pt
\item \helpref{wxString::Length}{wxstringLength}
\item \helpref{wxString::Empty}{wxstringEmpty}
\item \helpref{wxString::Allocation}{wxstringAllocation}
\item \helpref{wxString::IsNull}{wxstringIsNull}
\end{itemize}

\subsection{Transformations}

\begin{itemize}\itemsep=0pt
\item \helpref{wxString::Reverse}{wxstringReverse}
\item \helpref{wxString::Upcase}{wxstringUpcase}
\item \helpref{wxString::UpperCase}{wxstringUpperCase}
\item \helpref{wxString::DownCase}{wxstringDownCase}
\item \helpref{wxString::LowerCase}{wxstringLowerCase}
\item \helpref{wxString::Capitalize}{wxstringCapitalize}
\end{itemize}

\subsection{Utilities}

\begin{itemize}\itemsep=0pt
\item \helpref{wxString::Strip}{wxstringStrip}
\item \helpref{wxString::Error}{wxstringError}
\item \helpref{wxString::OK}{wxstringOK}
\item \helpref{wxString::Alloc}{wxstringAlloc}
\item \helpref{wxCHARARG}{wxstringwxCHARARG}
\item \helpref{CommonPrefix}{wxstringCommonPrefix}
\item \helpref{CommonSuffix}{wxstringCommonSuffix}
\end{itemize}

