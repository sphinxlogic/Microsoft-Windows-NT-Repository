\documentstyle[a4,makeidx,verbatim,texhelp,fancyhea,mysober,mytitle]{report}
\input psbox.tex

% Remove this for processing with dvi2ps instead of dvips
%\special{!/@scaleunit 1 def}

\parskip=10pt
\parindent=0pt
\title{Technical Reference Manual for wxWindows}
\author{Julian Smart\\Artificial Intelligence Applications Institute\\
University of Edinburgh\\EH1 1HN}
\date{March 1995}

\makeindex
\begin{document}
\maketitle

\pagestyle{fancyplain}
\bibliographystyle{plain}
\setheader{{\it CONTENTS}}{}{}{}{}{{\it CONTENTS}}
\setfooter{\thepage}{}{}{}{}{\thepage}
\pagenumbering{roman}
\tableofcontents

\chapter*{Copyright notice}
\setheader{{\it COPYRIGHT}}{}{}{}{}{{\it COPYRIGHT}}%
\setfooter{\thepage}{}{}{}{}{\thepage}

\begin{center}
Copyright (c) 1995 Artificial Intelligence Applications Institute,
The University of Edinburgh\\
\end{center}

Permission to use, copy, modify, and distribute this software and its
documentation for any purpose is hereby granted without fee, provided that the
above copyright notice, author statement and this permission notice appear in
all copies of this software and related documentation.

THE SOFTWARE IS PROVIDED ``AS-IS'' AND WITHOUT WARRANTY OF ANY KIND, EXPRESS,
IMPLIED OR OTHERWISE, INCLUDING WITHOUT LIMITATION, ANY WARRANTY OF
MERCHANTABILITY OR FITNESS FOR A PARTICULAR PURPOSE.

IN NO EVENT SHALL THE ARTIFICIAL INTELLIGENCE APPLICATIONS INSTITUTE OR THE
UNIVERSITY OF EDINBURGH BE LIABLE FOR ANY SPECIAL, INCIDENTAL, INDIRECT OR
CONSEQUENTIAL DAMAGES OF ANY KIND, OR ANY DAMAGES WHATSOEVER RESULTING FROM
LOSS OF USE, DATA OR PROFITS, WHETHER OR NOT ADVISED OF THE POSSIBILITY OF
DAMAGE, AND ON ANY THEORY OF LIABILITY, ARISING OUT OF OR IN CONNECTION WITH
THE USE OR PERFORMANCE OF THIS SOFTWARE.

\chapter{About this technical reference}
\pagenumbering{arabic}%
\setheader{{\it CHAPTER \thechapter}}{}{}{}{}{{\it CHAPTER \thechapter}}%
\setfooter{\thepage}{}{}{}{}{\thepage}

wxWindows is a class library for C++ providing GUI (Graphical User
Interface) and other facilities on more than one platform. This document
describes technical aspects, likely to be of interest to those wishing
to port wxWindows onto new platforms, or adding functionality.

This is a first attempt at a technical reference and will initially
take the form of rough and random notes on specific issues.

Please refer to the wxWindows user and reference manuals for information
about the wxWindows Application Programmer's Interface (API).

\chapter{Architecture of wxWindows}

This chapter summarizes the overall design of wxWindows.

wxWindows is a set of C++ classes encapsulating miscellaneous GUI and operating
system functionality. All classes derive from the class wxObject, which
has one member variable: {\bf \_\_type}. This variable allows any object to identify
its type dynamically, and the application can query the type using the {\bf wxSubType} function.

Most (but not all) classes have one base, platform-independent version,
and several platform-dependent versions (one for each platform). This allows base functionality
to be coded once and for all, reimplementing only the minimum of specific functionality
for each platform. The base headers and source reside in wx/include/base and wx/src/base,
and the platform-specific headers and source have their own directories under wx/include
and wx/src.

The file {\bf wx\_defs.h} implements some common types and identifiers, and {\bf wx.h}\rtfsp
includes common.h plus many headers which are likely to be used by an application.

\section{Makefiles}

Currently, one set of makefiles is required for UNIX, and another set
for every different MS Windows compiler. Variation between compiler
switches and make utilities seems to be greater under DOS than
under UNIX.

For UNIX makefiles, there are some standard targets.

\begin{description}\itemsep=0pt
\item[all] The default target. Calls the makefile recursively.
\item[xview] XView compilation. Calls the makefile recursively.
\item[motif] Motif compilation. Calls the makefile recursively.
\item[hp] Hewlett Packard-specific target (for local AIAI convenience). Calls the makefile recursively.
\item[clean...] Cleans the object files, executable and library archive (if any). The ellipsis
indicates a GUI suffix e.g. \_motif, \_ol.
\item[cleanall...] Cleans the object files, executable and library archive (if any),
plus any associated utilities (possibly including the wxWindows library itself).
The ellipsis indicates a GUI suffix e.g. \_motif, \_ol.
\end{description}

For DOS targets, the {\bf all} target is usually the default target, and there
are often {\bf clean} and {\bf cleanall} targets.

The file {\bf wx/src/make.env} has common settings that are included
by all UNIX makefiles. Currently there is no equivalent for DOS makefiles.

\section{Setup and peripheral libraries}

Not all wxWindows code is in the main directories; other peripheral utilities
and contributions are included according to settings in the file {\bf wx\_setup.h}\rtfsp
and in the main wxWindows makefile.

It is possible to switch on and off various parts of wxWindows according to need,
via the {\bf wx\_setup.h} file, whether or not the code resides in separate libraries.
For example, the PostScript support may be deactivated.
See the FAQ document for details on these settings.

\section{Platform independence}

Wherever possible, functionality described in the reference manual is the same
across all platforms. However, this is not always possible, but taking a strict lowest
common denominator approach would often be foolish. So, some functionality is
implemented on specific platforms only: but in the same wxWindows style, so that
the ifdef-ing required is minimal and the programmer has little difficulty
in writing platform-specific code. In the case of the Windows metafile device context,
which is only available under MS Windows, drawing code which is used for other device contexts
(such as wxCanvasDC or wxPostScriptDC) can be used immediately, so the amount
of extra programming required to implement this platform-specific feature is absolutely
minimal.

Similarly, Windows MDI support is obtainable with the use of a window style at run-time,
making it possible to enhance a Windows-hosted wxWindows application with very little fuss.

If you decide to add functionality to wxWindows, {\it try} to be generic, but be pragmatic
if total platform-dependence is not feasible.

\chapter{Implementation}

This chapter describes some of the details of implementation.

\section{Motif}

\section{MS Windows}

Much of the implementation of (sub)windows centres
around wx\_win.cc. The approach taken is to use a set of {\it intermediate} classes,
whose base class is called {\it wxWnd}. {\bf wxSubWnd} is used for subwindows
and dialog boxes. The members of these classes are called by the main
window or dialog procedure (also in wx\_win.cc), and themselves call
the member functions of the `real' wxWindow-derived classes.

This introduction of intermediate classes helps deal with Windows-specific
issues without too much confusion with the more abstract wxWindows classes. Also,
there is not always a one-to-one mapping between these classes: for example,
wxStatusWnd is used to implement the fields of a status line, which is
not an individual window in wxWindows terms.

The intermediate classes have their own {\it handle} member which points
to the actual HWND. The wxWindows abstract class uses its {\it handle} member
to point to the intermediate window class. For convenience, use wxWindow::GetHWND to
get the actual HWND for any wxWindows window.

\subsection{Window procedures}

The window procedures wxWndProc (for subwindows, frames) and and wxDlgProc (dialogs)
intercept a range of normal Windows messages, and call member functions in the
intermediate wxWnd classes. Because a wxWnd may not have been associated with
a wxWindow just as the window is being created, there has to be a temporary
`hook' (wxWndHook) which is set just before creation, so that the window
procedure knows about the wxWnd when it gets a message during the creation
process.

wxWnd::DefWindowProc defines a default procedure to be called if a message
is not handled.

\subsection{wxWindow Destructor}

There is some code in the wxWindow destructor that looks like it should belong
in the destructors of individual classes. However, it's necessary to do some cleanup
in the wxWindow destructor because the ordering of destructor calling conflicts
with what's required for window cleanup.

\chapter{Adding new window classes}

\section{Adding a panel item}

To add a new panel item, you need to write a wrapper for it in the same manner that
existing widgets are done. For example, the Motif and Windows gauge widgets are
totally separate C code from wxWindows (contrib/xmgauge and
contrib/gauge respectively). I've created the appropriate classes: wxbGauge
in include/base/wb\_gauge.h, and wxGauge in include/msw/wx\_gauge.h and
include/x/wx\_gauge.h.

If the widget is likely to be useful for all wxWindows users, it
can be integrated into the wxWindows distribution. A contributor
can send me code and a bit of explanation and I'll include it.
If it's of more limited interest, it can be organized as a separate
library or patch file, and I can put it on our ftp site. Or it
may be something you can't distribute externally and you'll
want to find an easy way to patch it into new releases of wxWindows
as you get them.

There's no way of slotting in custom widgets by simply providing a DLL,
and I don't really see how this would be possible because, unlike with
(say) plain Windows, you're always going to have to write the C++
wrapper class (which might, of course, still call a DLL).

\section{Adding a subwindow}

Similar to adding a panel item. See the implementation of wxTextWindow
in wx\_text.cc for an example.

\chapter{Documentation}

For a cross-platform project, it is important for the documentation to
be available in a wide range of paper and online formats.

For this reason, all documentation is written in a subset of LaTeX and
converted to the following formats:

\begin{enumerate}\itemsep=0pt
\item PostScript, via LaTeX and dvips
\item Windows Help, via Tex2RTF and the Windows help compiler
\item Word processor RTF (especially Word for Windows), via Tex2RTF
\item World Wide Web HTML, via Tex2RTF
\item wxHelp .XLP, via Tex2RTF
\end{enumerate}

Please read the Tex2RTF manual if you are going to write or modify
wxWindows documentation, and ensure that the LaTeX source conforms
to Tex2RTF guidelines.

\chapter{Using wxWindows in non-C++ languages}

Bindings to other languages are available or being developed. Examples of these
are wxCLIPS, wxPython, wxPerl, wxScheme and wxOPL (Object Prolog).

To help in this effort, a library called wxExtend has been devolved
from the original wxCLIPS code. This library maps from most
wxWindows C++ classes to C functions that take a long integer argument
representing the actual C++ object. Dynamic type checking is performed
by the library, and it also helps the implementer register
C callbacks representing the C++ overridable member functions,
which can then call the appropriate foreign language callback or
member function.

This library is available from AIAI at:\\
ftp.aiai.ed.ac.uk/pub/packages/wxwin/utils/wxextend.


\addcontentsline{toc}{chapter}{Index}
\printindex
\setheader{{\it INDEX}}{}{}{}{}{{\it INDEX}}%
\setfooter{\thepage}{}{}{}{}{\thepage}

\end{document}

