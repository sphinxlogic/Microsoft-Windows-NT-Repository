	\def\IDENT{01-014}L%%%%%%%%%%%%%%%%%%%%%%%%%%%%%%%%%%%%%%%%%%%%%%%%%%%%%%%%%%%%%%%%%%%%%%%%%%%%L% Copyright  1989--1991  by  Hunter  Goatley.   This  code  may  be freely %L% distributed and modified for non-commercial purposes  as  long  as  this %L% copyright notice is retained.   Please notify the author of any fixes or %"% improvements you make.						   %L%%%%%%%%%%%%%%%%%%%%%%%%%%%%%%%%%%%%%%%%%%%%%%%%%%%%%%%%%%%%%%%%%%%%%%%%%%%%%%  File:	NEWSLETTER_FORMAT.TEX%%  Abstract:%?%	This file defines TeX control sequences required to produce a4%	newsletter.  It assumes plain.tex has been loaded.%%  Author:	Hunter Goatley%		VAX Systems Programmer%		Western Kentucky University%		Academic Computing, STH 226%		Bowling Green, KY 42101%		Voice:  502-745-5251#%		E-mail: goathunter@WKUVX1.BITNET%5%		Partially based on examples from _The TeXbook_, by9%		Donald E. Knuth, and various other sources.  Virtually;%		all of the macros from other sources have been rewritten%		or at least modified.%%  Date:	June 3, 1989%%  Modified by:%+%	01-014		Hunter Goatley		21-AUG-1991 10:47<%		Rewrote double-quote macros (they work now!).  Cleaned up)%		a little bit.  Added \slant and \ital.%+%	01-013		Hunter Goatley		25-JAN-1991 08:23<%		Added \newpage.  Added \tt definitions to \*point macros.>%		Fixed problem with \quotee (quotee name was getting split).;%		Added setting of \hyphenpenalty and \emergencystretch to%		\newspage macro.%+%	01-012		Hunter Goatley		15-JAN-1991 07:03@%		Added macro \round to help with keeping page & figure heights>%		even.  It helps some.  Re-worked \beginquote and \endquote.9%		Add \Quote and \quotee.  Added \listindent and dimens.=%		Modified \coltitle to include \noindent; changed amount of9%		\vglue.  Modified \beginlist and \endlist to check for8%		a \parskip of 0pt; if 0, skip .5\baselineskip.  Fixed%%		\centerbox (was \vbox, now \hbox).%+%	01-011		Hunter Goatley		 5-JAN-1991 23:40B%		Changed \ednote macro so that \sl is redefined as \rm, not \ss.:%		Changed \say macro to look like LaTeX's \typeout macro.%+%	01-010		Hunter Goatley		 6-OCT-1990 02:551%		Commented out \ss calls at end of font macros.:%		Placed newsletter hsize commands in a macro (\newspage)3%		so that NEWTEX can be used like TeX normally is.%+%	01-009		Hunter Goatley		10-MAR-1990 16:059%		Added \par to the beginning of definition of \endlist.%+%	01-008		Hunter Goatley		 7-DEC-1989 22:07@%		Added \farright (place hbox flush right or on next line if it>%		won't fit).  Modified definition of \eoa to call \farright.>%		Added \pmb -- "poor man's bold."  Added font cmssxb.  Added%		\eldots and \edots.%+%	01-007		Hunter Goatley		 1-OCT-1989 03:418%		Added figure support for multiple column environment.%		Added \centerbox.%+%	01-006		Hunter Goatley		17-AUG-1989 10:11=%		Modified so that "@" is an active character throughout the?%		format file.  Added new header/footer commands that are more:%		flexible.  Improved appearance of shadow box created by%		\leftshadowbox.%+%	01-005		Hunter Goatley		10-AUG-1989 22:09:%		Modified \onepageout output routine so that it does not<%		disable interline skip (commented out \offinterlineskip).9%		This was causing the header and footer to appear flush9%		with the main body of text on the page.  Added routine;%		\checkdqbalsub.  Modified \include to automatically call2%		\checkdqbalsub after the file has been read in.%+%	01-004		Hunter Goatley		30-JUL-1989 03:10;%		Added macros to handle font size changes.  Added code to%		let " be used.%+%	01-003		Hunter Goatley		29-JUL-1989 19:26@%		Modified double column routines so that the number of columns>%		can be specified.  The routines now work for 2 to 6 columns
%		of output.%+%	01-002		Hunter Goatley		23-JUL-1989 22:21=%		Added more comments.  Fixed double column routines so that#%		page sizes are handled properly.%+%	01-001		Hunter Goatley		 2-JUL-1989 21:28)%		Added macros to handle bibliographies.%+%	01-000		Hunter Goatley		 3-JUN-1989 14:56%		Original version.%7\def\say#1{{\let\protect\string\immediate\write10{#1}}};\say{TeX Input file for Newsletter format - version \IDENT}*\say{Copyright 1989-1991,  Hunter Goatley}S\everyjob{\say{TeX Newsletter version \IDENT. Copyright 1989-1991, Hunter Goatley}}%L%  The \catcode command below lets us use "@" as a letter.  It can thereforeG%  be used in command and variable names.  PLAIN TeX uses this to avoid1%  conflicts with user code, so we'll do it, too.%@\catcode`@=11			% borrow the private macros of PLAIN (with care)\say{Loading fonts...}%9%  Load fonts and define commands to switch between fonts%N\font\twelverm=cmr12  \font\tenrm=cmr10  \font\ninerm=cmr9  \font\eightrm=cmr8N\font\twelvei=cmmi12  \font\teni=cmmi10  \font\ninei=cmmi9  \font\eighti=cmmi8P%\font\twelvesy=cmsy12 \font\tensy=cmsy10 \font\ninesy=cmsy9 \font\eightsy=cmsy8O\font\twelvebf=cmbx12 \font\tenbf=cmbx10 \font\ninebf=cmbx9 \font\eightbf=cmbx8O\font\twelvett=cmtt12 \font\tentt=cmtt10 \font\ninett=cmtt9 \font\eighttt=cmtt8O\font\twelveit=cmti12 \font\tenit=cmti10 \font\nineit=cmti9 \font\eightit=cmti8O\font\twelvesl=cmsl12 \font\tensl=cmsl10 \font\ninesl=cmsl9 \font\eightsl=cmsl8O\font\twelvess=cmss12 \font\tenss=cmss10 \font\niness=cmss9 \font\eightss=cmss8A\font\twelvessi=cmssi12 \font\tenssi=cmssi10 \font\ninessi=cmssi9,\font\eightssi=cmssi8  \font\tenssb=cmssbx10%%  11-point font is scaled 10pt%'\font\elevenrm=cmr10 scaled\magstephalf'\font\eleveni=cmmi10 scaled\magstephalf(\font\elevenbf=cmbx10 scaled\magstephalf(\font\eleventt=cmtt10 scaled\magstephalf(\font\elevenit=cmti10 scaled\magstephalf(\font\elevensl=cmsl10 scaled\magstephalf(\font\elevenss=cmss10 scaled\magstephalf*\font\elevenssi=cmssi10 scaled\magstephalfM\font\seventeenrm=cmr17  \font\seventeenss=cmss17  \font\seventeenssi=cmssi17\def\seventeenpoint{%C	\def\sl{\seventeenssi}\def\it{\seventeenssi}\def\bf{\seventeenss}%C	\def\rm{\seventeenrm}\def\ss{\seventeenss}\def\ssi{\seventeenssi}%,	\baselineskip=19pt%			% Change baselineskip&	\rm%					% By default, use sans serif}\def\twelvepoint{%8	\def\sl{\twelvesl}\def\it{\twelveit}\def\bf{\twelvebf}%C	\def\rm{\twelverm\let\sl=\twelvesl}\def\ss{\twelvess\let\sl=\ssi}%(	\def\ssi{\twelvessi}\def\tt{\twelvett}%,	\baselineskip=14pt%			% Change baselineskip&	\rm%					% By default, use sans serif}\def\elevenpoint{%8	\def\sl{\elevensl}\def\it{\elevenit}\def\bf{\elevenbf}%C	\def\rm{\elevenrm\let\sl=\elevensl}\def\ss{\elevenss\let\sl=\ssi}%	\def\ssi{\elevenssi}%	\def\tt{\eleventt}%,	\baselineskip=13pt%			% Change baselineskip&	\rm%					% By default, use sans serif}\def\tenpoint{%/	\def\sl{\tensl}\def\it{\tenit}\def\bf{\tenbf}%	\def\ssb{\tenssb}%:	\def\rm{\tenrm\let\sl=\tensl}\def\ss{\tenss\let\sl=\ssi}%	\def\ssi{\tenssi}%	\def\tt{\tentt}%,	\baselineskip=12pt%			% Change baselineskip&	\rm%					% By default, use sans serif}\def\ninepoint{%2	\def\sl{\ninesl}\def\it{\nineit}\def\bf{\ninebf}%=	\def\rm{\ninerm\let\sl=\ninesl}\def\ss{\niness\let\sl=\ssi}%	\def\ssi{\ninessi}%	\def\tt{\ninett}%,	\baselineskip=11pt%			% Change baselineskip&	\rm%					% By default, use sans serif}\def\eightpoint{%5	\def\sl{\eightsl}\def\it{\eightit}\def\bf{\eightbf}%@	\def\rm{\eightrm\let\sl=\eightsl}\def\ss{\eightss\let\sl=\ssi}%	\def\ssi{\eightssi}%	\def\tt{\eighttt}%,	\baselineskip=10pt%			% Change baselineskip&	\rm%					% By default, use sans serif}!\font\HUGE=cmr17 %scaled\magstep2;\font\verysmallrm=cmr5			% Used to make small copyright "C";\font\smallsy=cmsy7			% Used to make small copyright circle8\def\tiny{\eightpoint\ss}		% Equate \tiny to \eightpoint\let\em=\eightssi5\font\quotefont=cmss12 at 14.4truept	% Quotation font+\font\quoteefont=cmcsc10		% Font for quotee\say{Defining macros...}0\newdimen\normalhsize			% Create a new dimension0\newdimen\normalvsize			% Create a new dimension\normalhsize=\hsize%%  \newspage%H%	Set up the page for a newsletter (7" by 9" of text, higher tolerance).%\def\newspage{%-	\global\topskip=0pt		% Set 1 inch top margin>	\global\hoffset=-.25 true in	% Move output .25 in to the leftB	\global\pretolerance=1000	% Set tolerance up (before hyphenation)3	\global\tolerance=1500		% ...  (after hyphenation)3	\global\hyphenpenalty=500	% Discourage hyphenation	\global\emergencystretch=30pt	\global\normalhsize=7in5	\global\hsize=\normalhsize	% Pages are 7 inches wide,	\global\vsize=9in		% ...  and 9 inches tall'	\global\abovedisplayskip=\baselineskip'	\global\belowdisplayskip=\baselineskip	\global\pagewidth=\hsize	\global\pageheight=\vsize}B\clubpenalty=500 %-1000			% Set penalties for club and widow lines\widowpenalty=1000			% ...%P%=-=-=-=-=-=-=-=-=-=-=-=-=-=-=-=-=-=-=-=-=-=-=-=-=-=-=-=-=-=-=-=-=-=-=-=-=-=-=-=2%                           Multiple Column OutputP%=-=-=-=-=-=-=-=-=-=-=-=-=-=-=-=-=-=-=-=-=-=-=-=-=-=-=-=-=-=-=-=-=-=-=-=-=-=-=-=%0%	Define macros to handle multiple column output%F%	These macros support figures, but the figures must be defined insideH%	the multiple-column environment.  Each column can have up to 3 figuresG%	of varying sizes; the figures are referred to as "top", "middle", andG%	"bottom".  Figures are identified via page number, column number, andG%	the position.  In addition, two special figures that span all columns:%	can be specified using column 0 with "top" and "bottom".%F%	For example, {3}{2}{middle}  refers to the middle figure in column 2%	of output page 3.%D%	The figures are stored as temporary inserts (the inserts are freedF%	after \endcolumns).  The inserts are treated as an array whose first%	entry is \c@lfigstart.%I%	Note: column figures are not handled on the last page of column output.<%	\balancecolumns does not understand how to deal with them.%<%	While in X column mode, the text is arranged as one columnB%	\columnwidth wide and \pageheight*X tall.  This narrow column is*%	then split into X boxes of equal height.%(%***************************************%N%     Allocate a bunch of dimens, boxes, and counts for use in multiple-column
%     macros.%+\newdimen\columnwidth			% Width of a column9\newdimen\columnseprule			% Width of rule between columns;\newdimen\columnsep			% Width of whitespace between columns+\newdimen\pagewidth			% Total width of page-\newdimen\pageheight			% Total height of page0\newdimen\pageheightkeep		% Total height of page&\newdimen\ruleht			% Height of \hrules7\newif\ifcolfigs			% Create a new \if (true if figures)=\newbox\partialpage			% Box to hold partial page before cols.=\newbox\partialpagetop			% Box to hold figure for top of pageB\newbox\partialpagebottom		% Box to hold figure for bottom of page,\newcount\mulc@lbegin			% Allocate a counter,\newcount\mulc@lpart 			% Allocate a counter,\newcount\numberofcols			% Number of columns=\newcount\c@lpageno			% Current page # for column environment-\newcount\maxcolfigs			% Maximum # of figures;\newcount\c@lfigstart			% Starting array slot # for inserts9\newcount\c@lslot			% Work counter to hold current slot #=\newcount\maxfigp@ges			% Maximum # of pages of column output8\newcount\figsperp@ge			% Number of figures on each page%2%  Allocate multi-column work counters and dimens.%<\newdimen\tmp@			% Dimen used to calculate pageheights, etc.>\newdimen\tmp@one		% Dimen used to calculate pageheights, etc.*\newcount\tmp@two		% Usually # of cols * 2\newcount\tmp@three\newcount\tmp@four\newcount\tmp@five\newcount\tmp@six\newcount\tmp@seven*\newdimen\rtmp@			% Dimen used in rounding%E%  Initialize the \column* dimens.  If \columnwidth is still 0pt when<%  \begincolumns is called, the correct \columnwidth will beG%  calculated from the current \hsize.  This lets the user set a column,%  width, without requiring that it be done.%*\columnsep=20pt				% Space between columns5\columnseprule=.4pt			% Width of rule between columns2\columnwidth=0pt			% Initialize columnwidth to 0pt%.% Initialize pagewidth, pageheight, and ruleht%1\pagewidth=\hsize \pageheight=\vsize \ruleht=.5pt%C%  Define constant values used to indicate type of figure stored inH%  figure boxes.  Probably inefficient to store them as macros, but it'sC%  easier that way (and they're not hardcoded in the macros below).%/\def\TopFig@{500}				% Figure is at top of page2\def\BotFig@{600}				% Figure is at bottom of page0\def\RegFig@{321}				% Figure is a corner figure2\def\topc@lpos{1}				% The top figure position (1)8\def\middlec@lpos{2}				% The middle figure position (2)8\def\bottomc@lpos{3}				% The bottom figure position (3)%%  Macro \onepageout%@%	Output routine used to actually shipout pages to the DVI file.%D\def\onepageout#1{\shipout\vbox{ % here we define one page of output@    {\hsize=\pagewidth\makeheadline}		% Do normal width headline3    %\offinterlineskip				% butt the boxes together:    \vbox to \pageheightkeep{			% Create a vbox big enough0      \boxmaxdepth=\maxdepth			% Set boxmaxdepth)      #1					% Now insert the information#      }						% \vbox to \pageheight@    {\hsize=\pagewidth\makefootline}		% Do normal width footline    }						% End of \vbox1  \advancepageno				% Advance current page number!  }						% End of \def\onepageout@%\output{\onepageout{\unvbox255}}		% Send out any current output% %  When \begincolumns is called:%5%  \output is set to perform the following functions:A%	Store the current vbox in the partialpage box (with some vskip)A%	Eject the page (executing \output - page is not really ejected) %  Set new definition of output:/%	Call \multiplecolumnout to output the columns%  Set \hsize = \columnwidthN%  Set \vsize = (\pageheight * X) - (\ht\partialpage * X) --- each column will5%	now be just as tall as the space below \partialpage%M%   Sometimes the first \output was executed twice in a row, which caused theM%   first \partialpage to be lost.  The code below includes a counter that isK%   used to determine if the output routine is called a second time.  If itM%   is, the routine ships the previously stored partial page and then creates%   a new \partialpage.%E%  Originally, \begincolumns set \vsize = X * \pageheight (it did notK%  take the height of the \partialpage into account).  If the complete textJ%  would fit in the full \vsize, the balancing routine would have problemsG%  balancing and you'd end up with lots of whitespace on a page and theI%  multiple columns would show up on the next page.  Make sense?  If not,%%  just trust me; it caused problems.%)%  The solution to the problem was to set%4%	\vsize = (\pageheight * X) - (\ht\partialpage * X)%L%  and then set \vsize back to (X*\pageheight) after the first page had beenM%  ejected.  This worked fine and dandy except for the fact that grouping wasE%  causing the changes to not hold.  The solution (and it is a littleL%  dangerous) was to use \global\vsize so that the \vsize changes were neverE%  local to the group.  \endcolumns then resets \vsize = \pageheight.(%  Again, trust me; it seems to work OK.%\def\begincolumns#1{O  \ifnum#1<2 \errmessage{Invalid number of columns -- #1; must be 1 < X < 7}\fiO  \ifnum#1>6 \errmessage{Invalid number of columns -- #1; must be 1 < X < 7}\fi$  \begingroup				% Begin a new group;  \global\mulc@lbegin=1			% Set "first time" output counter@  \global\mulc@lpart=1			% Set "first time" outputting dbl cols.9  \global\pageheightkeep=\vsize		% Initialize \pageheight5  \global\pageheight=\vsize		% Initialize \pageheightC  \round{\pageheight}{\baselineskip}{<}	% Round the pageheight down3  \global\pagewidth=\hsize		% Initialize \pagewidth2  \global\tmp@two=#1  \global\advance\tmp@two by-19  \global\numberofcols=#1		% Initialize number of columns3  %  Calculate column width unless user set a value<  \ifdim\columnwidth=0pt		% If user did not set \columnwidth?     \columnwidth=\hsize		% Set columnwidth = normal page width@     \tmp@=-\columnsep			% Copy neg. amount of columnsep to tmp@H     \multiply\tmp@ by\tmp@two		% Multiply by X (total amount of colsep)H     \advance\columnwidth by\tmp@	% Subtract width space between columnsN     \divide\columnwidth by\numberofcols  % Divide by X to get X column output+  \fi					% End of \columnwidth calculation	  \output={%3        \ifnum\mulc@lbegin=1				% 1st time through?%G	  \global\setbox\partialpage=\vbox{\unvbox255\bigskip}	% Store partial/	  \global\mulc@lbegin=2				% Increment counteru!	\else						% 2nd time through...IB	  \onepageout{\vbox{\unvbox\partialpage}}	% Ship previous partialD	  \global\setbox\partialpage=\vbox{\unvbox255\bigskip}	% Make a new	\fi							% ... partial1	}\eject				% Force the output routine to execute4  \output={\multiplecolumnout}				% Output X columns2  \hsize=\columnwidth					% Set hsize = col. width:  % Set \vsize = (\pageheight * X) - (\ht\partialpage * X)5  \global\vsize=\pageheight				% Set \vsize=X*\pageht	9  \global\multiply\vsize by\numberofcols		% Multiply by XMM  \tmp@=-\ht\partialpage \multiply\tmp@ by\numberofcols	% Subtract (height ofl:  \global\advance\vsize by\tmp@				% ... partial page * X)2  \round{\vsize}{\baselineskip}{>}			% Round it up   }							% End of \begincolumns%e0%  \endcolumns performs the following functions:%F)%    Sets \output to call \balancecolumnsg%    \vfills page and ejects ito%    Terminates the groupe"%    Resets vsize to original size)%    Resets pagegoal = to original \vsize !%    Skips parskip vertical spacel@%    Signals that this is a good place for a break, if necessary%r0\def\endcolumns{\output={\balancecolumns} \ejectO  \endgroup \global\vsize=\pageheightkeep \pagegoal=\vsize \bigskip \goodbreak}e% %  \multiplecolumnout\%u %    Sets splittopskip = topskip"%    Sets splitmaxdepth = maxdepth6%    Sets TeX register \dimen@ = to height of the page:%    Subtracts the height of the partial page from \dimen@C%    Subtracts the heights of the top & bottom figures from \dimen@mL%    Splits up the current output box into X boxes of size \dimen@, handling%	figures, if present	,%    Calls \onepageout to dump the new boxesB%    Resets \vsize = (X * \pageheight) if first time through macro'%    Frees up current output vbox (255)e%eF\def\multiplecolumnout{\splittopskip=\topskip \splitmaxdepth=\maxdepth8  \dimen@=\pageheight \advance\dimen@ by-\ht\partialpageH  \advance\dimen@ by-\ht\partialpagetop		% Subtract height of top figureK  \advance\dimen@ by-\ht\partialpagebottom	% Subtract height of bottom fig.nJ  \tmp@four=\numberofcols   \multiply\tmp@four by2	% Calculate upper box #*  \tmp@five=0					% Start boxes with box 0-  \ifcolfigs					% Are there figures defined?46     \splitfigc@ls				% Yes - go handle text & figures-  \else						% No - split \box255 into X colsr    %yM    %  Split box255 into a box of dimen@ height.  Loop until X boxes created.     %	
    {\loopD	\global\setbox\tmp@five=\vsplit255 to\dimen@  \advance\tmp@five by26	\ifnum\tmp@five<\tmp@four \repeat}	% Loop if not done  \fi						% \ifcolfigsr7  \onepageout\pagesofar				% Send this page to DVI filep7  \ifnum\mulc@lpart=1				% Does vsize need to be reset? .	\global\mulc@lpart=2			% Yes - change counter:	\global\vsize=\pageheight		% Set \vsize = X * \pageheight-	\global\multiply\vsize by\numberofcols	% ...  \fi						% ...(  \ifcolfigs					% If figures defined...D     \global\advance\c@lpageno by\@ne		% Bump column output page no.3     \handlefigures				% Handle next page's figuresh  \fi						% \ifcolfigs ?  \unvbox255 \penalty\outputpenalty		% Free current output vbox	!}						% End of \multiplcolumnoutb%l%  \splitfigc@ls%tB%  This macro splits \box255 into X columns, handling the figures.% H%    Get figure array slot number for figure 1, column 1 on current page=%    Loop for each column (starting with column in \tmp@five)t!%	Make a copy of \dimen@ -> \tmp@p*%	If a figure is defined for top of column/%	   Decrease \tmp@ by the height of the figurep%	   Set box 1 = the figureG%	 Else3%	   Set box 1 = null vboxF%	Bump figure array slot number - now points to slot for middle figure-%	If a figure is defined for middle of column /%	   Decrease \tmp@ by the height of the figuren%	   Set box 5 = the figurey%	 Elsel%	   Set box 5 = null vboxF%	Bump figure array slot number - now points to slot for bottom figure-%	If a figure is defined for bottom of column /%	   Decrease \tmp@ by the height of the figureX%	   Set box 9 = the figurel%	 Else %	   Set box 9 = null vboxH%	Bump figure array slot number - now points to slot for next top figureB%	Split \box255 into 1 or 2 pieces (2 if there is a middle figure);%	Set \box\tmp@five = \box1 + \box3 + \box5 + \box7 + \box9=;%	Advance \tmp@five by 2 and loop until X "columns" createdm%2\def\splitfigc@ls{  %tK  %  Split box255 into a box of dimen@ height.  Loop until X boxes created.f  %i   \tmp@two=1				% Figure counterO  \calcc@lslot{\c@lpageno}{1}{\tmp@two}	% Get slot for figure 2 on current pagef  {\loop				% Start loop0	\tmp@=\dimen@			% Make a working copy of dimen@	%	%  Handle top figuref	%G	\ifnum\count\c@lslot=\RegFig@	% If this slot holds a single-column fig1F	   \advance\tmp@ by-\ht\c@lslot	% Subtract figure's height from \tmp@:	   \setbox1=\vbox{\unvbox\c@lslot}	% Copy figure to \box1L\say{Figure \the\tmp@two\space on page \the\c@lpageno, document page \folio}4	\else	\setbox1=\vbox{}	% Else set box 1 = null page	\fi				% \ifnum	%	%  Handle middle figure	%1	\advance\c@lslot by\@ne			% Point to next figuree3	\advance\tmp@two by\@ne			% Advance figure counteraG	\ifnum\count\c@lslot=\RegFig@	% If this slot holds a single-column figdF	   \advance\tmp@ by-\ht\c@lslot	% Subtract figure's height from \tmp@:	   \setbox5=\vbox{\unvbox\c@lslot}	% Copy figure to \box3L\say{Figure \the\tmp@two\space on page \the\c@lpageno, document page \folio}4	\else	\setbox5=\vbox{}	% Else set box 3 = null page	\fi				% \ifnum	%	%  Handle bottom figure	%0	\advance\c@lslot by\@ne		% Point to next figure2	\advance\tmp@two by\@ne		% Advance figure counterG	\ifnum\count\c@lslot=\RegFig@	% If this slot holds a single-column figpF	   \advance\tmp@ by-\ht\c@lslot	% Subtract figure's height from \tmp@:	   \setbox9=\vbox{\unvbox\c@lslot}	% Copy figure to \box3L\say{Figure \the\tmp@two\space on page \the\c@lpageno, document page \folio}4	\else	\setbox9=\vbox{}	% Else set box 3 = null page	\fi				% \ifnum9	\advance\c@lslot by\@ne		% Point to next figure in arrayf2	\advance\tmp@two by\@ne		% Advance figure counter	%G	%  Here, top figure is in \box1, middle figure is in \box5, and bottome	%	figure is in \box9.B	%  Split \box255 to fill the remaining column space, splitting in$	%	half if there is a middle figure.F	%  Special case: if height of \box1 = \pageheight, \box\tmp@five=null	%2	\ifdim\ht1=\pageheight			% Is top = whole column?6	   \global\setbox3=\vbox{}		% Yes - set \box3 to null6	   \global\setbox7=\vbox{}		% Yes - set \box7 to null'	\else					% No - go ahead and do splitt0	   \ifdim\ht5>0pt			% Is there a middle figure?7	     \divide\tmp@ by2			% Yes - break text to 2 pieces\@	     \global\setbox3=\vsplit255 to\tmp@	% Set \box3 = 1st piece@	     \global\setbox7=\vsplit255 to\tmp@	% Set \box7 = 2nd piece"	   \else				% No middle column...@	     \global\setbox3=\vsplit255 to\tmp@	% Put all text in \box32	     \global\setbox7=\vbox{}		% Set \box7 to null	   \fi					% \ifdim\ht5...%	\fi					% \ifdime	%>	%  Now put the 5 pieces together as one vbox in \box\tmp@five	%;	\global\setbox\tmp@five=\vbox to\dimen@{\offinterlineskip% '		\ifdim\ht1=0pt\else\vbox{\unvbox1}\fis'		\ifdim\ht3=0pt\else\vbox{\unvbox3}\fit'		\ifdim\ht5=0pt\else\vbox{\unvbox5}\fii'		\ifdim\ht7=0pt\else\vbox{\unvbox7}\fif'		\ifdim\ht9=0pt\else\vbox{\unvbox9}\fi9		\vfil\vfilneg		% Cancels spurious vglue ???? (it works)n		}a,	\advance\tmp@five by2			% Bump column box #6	\ifnum\tmp@five<\tmp@four \repeat}	% Loop if not done}						% End of \splitc@lfigs%
%  \pagesofar%%   Releases \partialpage boxtA%   Sets width of X boxes (from box 0 to box X) = to \columnwidthlN%   Creates an hbox = pagewidth that consists of box0 + separator rule + box 2%	+ separator rule + box X8%   This new hbox is \box255 and can be used to \shipout%t"\def\pagesofar{\unvbox\partialpage0  \unvbox\partialpagetop				% Include top figureG  \tmp@four=\numberofcols \multiply\tmp@four by2	% Loop boundary numbera%  \tmp@five=0						% Start with box 0	  %e:  %  For each box, set the width equal to the column width  %l9  {\loop \wd\tmp@five=\columnwidth  \advance\tmp@five by2i8		\ifnum\tmp@five<\tmp@four \repeat}	% Loop until X done  %p.  %  Now put all the boxes together like this:  %e  %		box  |  box  | ... |  box  %+  \tmp@five=0						% Start with box 0 again=L  \hbox to\pagewidth{\box\tmp@five\advance\tmp@five by2%  Do the left column;	\loop \hfil\vrule width\columnseprule \hfil \box\tmp@five%=,		\advance\tmp@five by2			% Bump box counter6		\ifnum\tmp@five<\tmp@four \repeat	% Loop if not done	}						% End of hboxh6  \unvbox\partialpagebottom				% Include bottom figure}							% End of \pagesofarl%nA%  \balancecolumns - Balance both columns at end of X column pages%h%   Sets box 0 = current pageo-%   Sets dimen@ = height of the vbox in box 0e %   Adds topskip value to dimen@+%   Subtracts X * \baselineskip from dimen@ G%   Divides dimen@ by X  -- dimen@ now has target height of each column D%   Sets splittopskip = to topskip so it can be added to all columns
%   Loops %	Copies box 0 to box 3o0%	Splits box 3 to dimen@ and stores in box 1 & 3F%	Increments dimen@ by 1pt and loops if column in box 3 exceeds dimen@G%	Note: Splitting of box is actually implemented as a loop that createso%	      X boxessL%   Moves columns in odd-numbered boxes to corresponding even-numbered boxes+%   Calls pagesofar to make the hbox for ith% ;\def\balancecolumns{\setbox0=\vbox{\unvbox255} \dimen@=\ht0m  %hA  % Subtract X * \baselineskip from dimen@ and divide dimen@ by Xi  % /  \tmp@two=\numberofcols  \advance\tmp@two by-1*J  \tmp@=-\baselineskip  \multiply\tmp@ by\tmp@two  \advance\dimen@ by\tmp@C  \divide\dimen@ by\numberofcols \splittopskip=\topskip		% Dimen@/X	  %iK  %  Split the column X times so we have X columns of equal height.  If our:  %  last column is > dimen@, bump dimen@ by one and loop.  %w:  \tmp@four=\numberofcols   \multiply\tmp@four by2	% X * 2E  \tmp@seven=\tmp@four  \advance\tmp@seven by-1		% Work box:  (X*2)-1o8  {\vbadness=10000 \loop \global\setbox\tmp@seven=\copy07    \tmp@five=0  \tmp@six=1				% Work box starts with 1a"    {\loop						% Loop for X boxes@	\global\setbox\tmp@six=\vsplit\tmp@seven to\dimen@	% Vsplit box+	\advance\tmp@six by2				% Bump box counterB7	\ifnum\tmp@six<\tmp@seven \repeat}		% Loop if not doneiE    \ifdim\ht\tmp@seven>\dimen@ \global\advance\dimen@ by1pt \repeat}  %oK  %  Here we have X columns of equal height.  Note that the last column may#   %  not be equal to the others.  %cN  %  Copy the columns to the even-numbered boxes in preparation for \pagesofar  %   \tmp@five=0   \tmp@six=1A  {\loop \global\setbox\tmp@five=\vbox to\dimen@{\unvbox\tmp@six} ,	\advance\tmp@five by2  \advance\tmp@six by2#	\ifnum\tmp@five<\tmp@four \repeat}n1  \pagesofar}					% Call \pagesofar to build pagee%a%  \c@lnewinsert%iP%  This macro is used to perform a \newinsert for temporary usage (inside a grp)%eF\def\c@lnewinsert{\advance\insc@unt by\m@ne	% Decrement insert counter9  \ch@ck0\insc@unt\count			% Make sure count is availablem9  \ch@ck1\insc@unt\dimen			% Make sure dimen is available\8  \ch@ck2\insc@unt\skip				% Make sure skip is available6  \ch@ck4\insc@unt\box				% Make sure box is available% NEED TO EMPTY BOX!!!+  \count\insc@unt=0				% Set the count to 0e;  \allocationnumber=\insc@unt			% Set the allocation number}						% End of \c@lnewinsertt%e%  \definefigs%sI%  This macro is called to establish the figure environment inside of theeJ%  multiple-column environment.  It allocates (until \endcolumns) an arrayE%  of inserts (boxes, counts, dimens, and skips) to handle all of thehD%  figures per page for the given number of pages.  When the figuresJ%  are defined using \definefig, the proper box is filled with the figure.% 
%  Inputs:% D%	#1	- Number of pages of multiple-column output (should be as large;%		  as the total number of pages between \begincolumns and%		  \endcolumns)	%%\def\definefigs#1{(	\colfigstrue			% Set column figure flag/	\global\c@lpageno=1		% Get current page numberf8	\maxfigp@ges=#1			% Set maximum # of pages with figures=	\figsperp@ge=\numberofcols	% Calculate # of figures per page B	\multiply\figsperp@ge by3	%  Figs/Page = 2 + (3 * number of cols)	\advance\figsperp@ge by2	% ...DB	\maxcolfigs=\maxfigp@ges	% Figure out how many inserts are neededA	\multiply\maxcolfigs by\figsperp@ge	% ... for all of the figuresi?	\tmp@two=\maxcolfigs		% Start there and work down to first box.	\loop	\c@lnewinsert  \advance\tmp@two by\m@ne		\ifnum\tmp@two>0 \repeat5	\c@lfigstart=\insc@unt		% Save starting array slot #i 	% We've allocated all boxes now	}				% End of \definefigs%\
%  \definefigh%kM%  This macro stores figure information in the appropriate slot in the insertaM%  array.  After calculating the proper slot number for the figure, it stores\K%  the figure in the corresponding box and sets the corresponding \count to 2%  a code identifying the box as holding a figure.%o
%  Inputs:%g%	#1	Output page number %	#2	Column Number#%	#3	Position (top, middle, bottom)	%	#4	The vbox for the figure% \def\definefig#1#2#3#4{t-	\ifcolfigs				% Has \definefigs been called?\	\elseP   \errmessage{Illegal use of \string\definefig\space before \string\definefigs}	\fi					% \ifnum\2	\ifnum#2>\numberofcols			% Illegal column number?;	   \errmessage{Column number #2 exceeds number of columns}e	\fi					% \ifnum F	\calcc@lslot{#1}{#2}{\csname #3c@lpos\endcsname}   % Calculate slot #"	\ifcase\csname #3c@lpos\endcsname7	   \or\setbox\c@lslot=\vbox{#4\vskip\belowdisplayskip}o7	   \or\setbox\c@lslot=\vbox{\vskip\abovedisplayskip#4%n			\vskip\belowdisplayskip} 7	   \or\setbox\c@lslot=\vbox{\vskip\abovedisplayskip#4})		\tmp@=\ht\c@lslot			% Round up the size (		\round{\tmp@}{\baselineskip}{>}		% ...1		\tmp@one=\tmp@  \advance\tmp@one by-\ht\c@lslot -		\setbox\c@lslot=\vbox to\tmp@{\box\c@lslot},	\fi/	\ifnum#2=0				% If column # is 0, special fig.g=	   \ifnum\csname #3c@lpos\endcsname=1\count\c@lslot=\TopFig@ A	     \else\count\c@lslot=\BotFig@\fi	% Top or Bottom figure thatn"	\else					% ... spans all columns=	      \count\c@lslot=\RegFig@		% Otherwise, identify the box	%	\fi					% ... as containing a figurea=\say{Processed #3 figure for column #2\space on page #1\spacea	- slot \the\c@lslot} $	\say{The height is \the\ht\c@lslot}}a%x%  \calcc@lslot %tH%  This macro is called to calculate the array slot number for a figure.%%  The formula for normal figures is:\%b+%	((column# - 1) * 3figs/column) + Positionr% O%  This macro assumes there can be 3 figures per column (top, middle, & bottom)r% 
%  Inputs:% %	#1	- Page number@%	#2	- Column number (0 = special figure that spans all columns)6%	#3	- Figure number (1 = top, 2 = middle, 3 = bottom)%v%  Returns:f%l<%	\c@lslot	- Slot number for given figure.  This slot number5%			  identifies the allocated insert for the figure."%t\def\calcc@lslot#1#2#3{u%	\tmp@seven=#1				% Start with page #i)	\advance\tmp@seven by\m@ne		% (Page - 1)z<	\multiply\tmp@seven by\figsperp@ge	% (Page - 1) * # of figs,	\ifnum#2=0				% If column is 0, special one5	   \ifnum#3=1\tmp@six=1\else\tmp@six=\figsperp@ge\fia4	   \advance\tmp@six by\m@ne		% Decrement for slot #	\else	   \tmp@six=#2				% Column #y-	   \advance\tmp@six by\m@ne		% (Column# - 1)b.	   \multiply\tmp@six by3		% (Column# - 1) * 39	   \advance\tmp@six by#3		% (Column# - 1) * 3 + Positioni(	\fi					% Really + 1, but -1 negates it8	\advance\tmp@seven by\tmp@six		% Add figure # to page #7	\advance\tmp@seven by\c@lfigstart	% Figure slot number 2	\global\c@lslot=\tmp@seven		% Set the slot number	}					% End of \calcc@lslot%\%  \handlefigures	%uJ%  This macro is called to step through all of the figures for the current>%  page and subtract the height of each from the total \vsize.%iJ%  The two special figures (top (1) and bottom (\figsperp@ge)) are handledM%  differently; because each spans all of the columns on the page, the heightaL%  of each is multiplied by the number of columns before subtracting it fromK%  \vsize.  The top figure is then placed in \partialpagetop and the bottoml*%  figure is placed in \partialpagebottom.%5%  Returns:	Adjusted \vsize.%	\def\handlefigures{e2	\global\vsize=\pageheight		% \vsize to pageheightB	\global\multiply\vsize by\numberofcols	% Multiply by # of columns0	\tmp@three=\figsperp@ge			% Start with figure X=	\calcc@lslot{\c@lpageno}{0}{3}		% Start with last figure boxh	{\loop%B	    \ifnum\count\c@lslot=\TopFig@	% If box is top section of page@		\global\setbox\partialpagetop=\vbox{\unvbox\c@lslot} % Copy it+		\tmp@=\ht\partialpagetop	% Get the heightc@		\multiply\tmp@ by\numberofcols	% Multiply by number of columns8		\global\advance\vsize by-\tmp@	% (vsize - figure size)	    \else				% Else:		\ifnum\count\c@lslot=\BotFig@	% If box is bottom section<		   \global\setbox\partialpagebottom=\vbox{\unvbox\c@lslot} 		   \tmp@=\ht\partialpagebottom#		   \multiply\tmp@ by\numberofcolss;		   \global\advance\vsize by-\tmp@ % (vsize - figure size)a(		\else				% Else, see if regular figure?		   \ifdim\ht\c@lslot>0pt\all@wfigure{\ht\c@lslot}\fi	% \vsizea		\fi				% End \ifnum 	     \fi				% End \ifnumg4	     \advance\tmp@three by\m@ne	% Point to next box2	     \advance\c@lslot by\m@ne	% Point to next box3	     \ifnum\tmp@three>0 \repeat}	% Loop until doneS	}					% End of \handlefigures\def\all@wfigure#1{S	\tmp@=#1		% Height of figureiA	\global\advance\vsize by-\tmp@	% Subtract figure size from vsizem	}P%%%%%%%%%%%%%%%%%%%%%%%%%%%%%%%%%%%%%%%%%%%%%%%%%%%%%%%%%%%%%%%%%%%%%%%%%%%%%%%%%  Macro:	\round#1#2#3% H%  Purpose:	Round a dimen parameter (#1) to the nearest even multiple of?%		parameter #2.  Primarily used to ensure that the page heightu>%		or a figure height is an even multiple of the baselineskip.%\@%		NOTE: this macro assumes there is no stretch or shrink to #2.%n
%  Inputs:2%		#1	Dimen variable to change (e.g., \pageheight)=%		#2	Dimen variable to use as multiple (e.g., \baselineskip)l6%		#3	Symbol indicating round up (>) or round down (<)(%		\tmp@	Work dimen (saved and restored)%l%  Example:b,%	\pageheight=598.213pt   \baselineskip=12ptC%	\round{\pageheight}{\baselineskip}{<}	%yields \pageheight=588.0ptsC%	\round{\pageheight}{\baselineskip}{>}	%yields \pageheight=600.0pt5%		o5\def\round#1#2#3{\bgroup%			%Keep \tmp@ changes locale 	\rtmp@=0pt				%Initialize \tmp@	\loop					%Begin a loop)	\advance\rtmp@ by #2			%Bump \tmp@ by #2m1	\ifdim#1>\rtmp@ \repeat			%Loop until \tmp@ > #1o	\ifx<#3					%If #3 = "<" then6	   \advance\rtmp@ by-#2			%... subtract #2 from \tmp@
	\fi					%...o'	\global#1=\rtmp@			%Reset parameter #1e	\egroup					%End the group\}						%End of macroP%%%%%%%%%%%%%%%%%%%%%%%%%%%%%%%%%%%%%%%%%%%%%%%%%%%%%%%%%%%%%%%%%%%%%%%%%%%%%%%%%\%  Define list macrosy%m	%	Dimens: %9%		\llistindent	- Amount of left indent  (0pt by default).9%		\rlistindent	- Amount of right indent (0pt by default)u%a	%	Macros: %a5%		\beginlist	- Begin list (skips space, sets indent)	%		\endlist	- Terminates a lista=%		\beginlistt	- Begin list with glue (used for list headers) '%		\endlistt	- Terminates a \beginlistts$%		\dotitem	- Itemize with a dot "o"% (\newdimen\llistindent   \llistindent=0pt(\newdimen\rlistindent   \rlistindent=0pt1\def\listindent#1{\llistindent=#1\rlistindent=#1} \def\beginlist{\begingroup%i9	\ifdim\parskip=0pt \vskip.5\baselineskip	% Skip 1/2 lined,	\else \vskip\parskip				%   or the \parskip	\fi						% 1	\parindent=10pt\parskip=0pt%			% Reset parindentsA	\leftskip=\llistindent \rightskip=\rlistindent}	% Indent marginso\def\endlist{\par\endgroup%u)	\ifdim\parskip=0pt \vskip.5\baselineskipc	\else \vskip\parskip\fi} ,\def\dotitem#1\par{\item{$\bullet$} #1 \par})\def\beginlistt#1{#1\vglue0pt\begingroup%b>	\divide\parskip by2\vskip\parskip\parindent=10pt\parskip=0pt%0	\leftskip=\llistindent \rightskip=\rlistindent}-\def\endlistt{\par\endgroup} %\vskip\parskip}t%g8%  Define a small copyright (for use with 8-point type).%=\def\smallcir{\smallsy\char13}0\def\smallcopyright{\leavevmode\raise.25ex\hbox{:	\ooalign{\hfil\raise.03ex\hbox{\kern .16em\verysmallrm C}	\hfil\crcr\smallcir}}}pP%%%%%%%%%%%%%%%%%%%%%%%%%%%%%%%%%%%%%%%%%%%%%%%%%%%%%%%%%%%%%%%%%%%%%%%%%%%%%%%%% $%  Define macros to manipulate boxes%l	%	Dimens: %S9%		\boxitrule=Xpt     - Width of rules used to draw boxess@%		\boxitspace=Xpt    - Space between box rules and box contents-%		\boxshadowsize=Xpt - Width of shadow boxes %n	%	Macros:g%?%	\articletitle{Title}{byline}	- Do article title in double boxo/%	\coltitle{Title}	- Do a title box in a column\D%	\shadowbox{some_box}    - Draw a shadow box around an hbox or vboxD%	\leftshadowbox{somebox} - Draw a left-hand shadow box around a box4%	\centerbox{somebox}     - Center a \vbox on a page=%	\boxit{some_box}        - Draw a box around an hbox or vboxm5%	\ednote                 - Do editor's note in a boxx%=?\newdimen\boxitspace  \newdimen\boxitrule  \newdimen\boxitwidth\"\boxitspace=3pt   \boxitrule=1.2pt:\boxitwidth=\boxitspace  \advance\boxitwidth by\boxitspaceB\advance\boxitwidth by\boxitrule  \advance\boxitwidth by\boxitrule%m=%  Define macro to write an article title inside a double boxi% %  Parameters:%t%	#1	- Title of articlet
%	#2	- Byline %\I%  For both \articletitle and \coltitle, the width of the box(es) must be\M%  subtracted from the current hsize in order for centering and justificationL%  to work right (otherwise the letters will run into the lines of the box).%e\def\articletitle#1#2{-	\bestbreak{				% Say that this is best breaksA	\advance\hsize by -\boxitwidth		% Bring margin in before \centere5	\advance\hsize by -\boxitwidth		% Do for both boxes!m)	\vskip 10pt plus 5pt			% Skip some spaceo	\boxit{					% Box the box9	\divide\boxitrule by 2			% Make inside box lines thinnere	\boxit{					% Box the texto"	\vbox{\noindent				% Create a box<	\centerline{\seventeenpoint\ss #1}	% Print title of article6	\centerline{\ninepoint\ss #2}		% Print article byline 	}}}}}					% Create the text box2\def\coltitle#1\par{{%				% Swallow next paragraph3	\advance\hsize by -\boxitwidth		% Bring margins in )	\boxit{%				% Draw a box around the textl9	\vbox{\ss\noindent #1}}}%		% Create a vbox that contains 9	\vglue.5\baselineskip%			% Skip some non-breakable space=,	\noindent}				% Don't indent next paragraph\def\ednote#1{{{!	\sl					% Switch to slanted fontg!	\def\sl{\/\rm}				% Redefine \sl\3	\advance\hsize by -\boxitwidth		% Bring margins int)	\boxit{					% Draw a box around the textfC	\vbox{\noindent	Editor's note: #1}}}	% Create a vbox that contains 	\vglue 0pt}				% Finish it up<\def\boxit#1{\vbox{\tithrule\hbox{\titvrule\kern\boxitspace%-	\vbox{\kern\boxitspace #1 \kern\boxitspace}%N&	\kern\boxitspace\titvrule}\tithrule}}&\def\tithrule{\hrule height\boxitrule}%\def\titvrule{\vrule width\boxitrule}t%7
%  \centerboxf%	L%  Create a \vbox that contains a centered \hbox.  The centering is relative%  to the current \hsize.e%%
%  Inputs:%o9%	#1	- \vbox to center   ->   \centerbox{\shadowbox{...}}1%tL\def\centerbox#1{\hbox{\hfil#1\hfil}}	% Create a \vbox containing a centered;\newdimen\oboxht  \newdimen\oboxwd  \newdimen\boxshadowsizex.\boxshadowsize=4pt				% Shadow box size is 4pt%	H%  Draw a righthand shadow box.  This is accomplished by building a vboxJ%  containing an hbox that is the boxed text and an hbox that is the right@%  hand shadow.  This vbox is then joined with a vbox that forms%  the bottom shadow.a%a\def\shadowbox#1{{3	\setbox0=\vbox{\boxit{#1}}		% Set box after \boxiti3	\oboxht=\ht0  \oboxwd=\wd0		% Store the dimensionsfA	\advance\oboxwd by-\boxshadowsize	% Subtract shadow size from wds$	\vbox{					% Put it all in one vbox,	\offinterlineskip			% Butt \vboxes together*	\vbox{					% Create a vbox of whole thing5	\hbox{\vbox{\unvbox0}			% Create box with text box +r	\hskip-\fontdimen2\font:	\lower\boxshadowsize		% Draw the right-hand boxshadowsizeI	\hbox{\vrule width\boxshadowsize height\oboxht}} % ...  Finish off \hboxv	}					% End of the \vboxp$	\advance\boxshadowsize by\boxitrule5	\vskip-\boxshadowsize			% Back up to bottom of \vboxt	\vbox{					% Start a new \vboxe?	\hbox{\kern\boxshadowsize\vbox{		% Create \hbox that is shadowt1	\hrule height\boxshadowsize width\oboxwd}}	% ...i	}					% End of the \vboxo	}					% End of \vboxp	}}					% End of \shadowboxu%\J%  Draw a lefthand shadow box.  This is accomplished by building a loweredH%  vbox containing an hbox that is the left hand shadow and an hbox thatL%  contains the boxed text.  This vbox is then joined with a vbox that forms%  the bottom shadow.c%m\def\leftshadowbox#1{{3	\setbox0=\vbox{\boxit{#1}}		% Set box after \boxite3	\oboxht=\ht0  \oboxwd=\wd0		% Store the dimensions A	\advance\oboxwd by-\boxshadowsize	% Subtract shadow size from wdi	\vbox{m,	\offinterlineskip			% Butt \vboxes together*	\vbox{					% Create a vbox of whole thing	\hbox{					% Create an hbox:	\hskip-\fontdimen2\font			% Move left one character width9	\hskip-\boxshadowsize			% Move left = size of shadow box	B	\advance\boxitrule by\boxshadowsize	% Make shadow a tad bit wider3	\lower\boxshadowsize			% Move down the same amount @	\hbox{\vrule width\boxitrule height\oboxht}	% Draw the left box	}					% ...  Finish off \hboxA	\vskip-\oboxht\vskip-\boxshadowsize	% Move back up to top of box 9	\hbox{\vbox{\unvbox0}			% Create box with text inside itt	}}					% End of vboxes*@	\advance\boxshadowsize by\boxitrule	% Make box a little thicker&						% ... so it overlaps bottom line4	\vskip-\boxitrule			% Move up height of bottom line	\vbox{					% Start a new \vbox\.	\hbox{\vbox{				% Create \hbox that is shadow1	\hrule height\boxshadowsize width\oboxwd}}	% ...w	}					% End of the \vboxe	}					% End of \vboxs	}}					% End of \shadowboxm!\def\bestbreak{\par\penalty-1000}u%mG%  Begin a quotation.  The quote is separated from the main text by twon/%  hrules and is indented from the normal text.d%s\def\beginquote{-	\begingroup				% Define beginning of a groupm	\quotefont\baselineskip=16.8pt !	\hrule height2pt			% Draw a linex-	\parindent 5pt				% Indent paragraphs by 5pto6	\vglue\medskipamount			% Use some glue so rule sticks/	\narrower\narrower			% Bring margins in (10pt)%	\noindent				% Don't indent	}					% End of macrog%a%  End a quotation.t%e\def\endquote{6	\vglue\medskipamount			% Use some glue so rule sticks!	\hrule height2pt			% Draw a linea#	\endgroup				% End the quote group 	}E\def\quotee#1{{\hfill\break\hbox{}\nobreak\hfill\hbox{\quoteefont #1}p	\finalhyphendemerits=0}}o/\def\Quote#1#2{\vbox{\vskip1.2pt\beginquote #1%x'		\quotee{#2}\endquote\vskip1.20003pt}}n%p %  Insert current month and year%\G\def\DATE{\ifcase\month\or January\or February\or March\or April\or Mayn?	\or June\or July\or August\or September\or October\or Novembere#	\or December\fi\space\number\year}f%l%  Include a TeX file.% 8\def\include#1{\immediate\write10{Including TeX file #1} 	\input #1				% Read the file in)	% Things to do after formatting the filed	}% 2%  Separate articles with some vskip and an \hrule% /\def\articlesep{				% Rule to separate articlesr*	\vglue 10pt plus2pt minus4pt		% Use vglue3	\hrule %height.4pt			% Draw a rule equal to \hsizec9	\vskip 10pt plus2pt minus2pt		% Skip some vertical space		}E% Put release flush right.  If it won't fit, put it on the next line.o% From TeXbook, Chapter 14.m9\def\farright#1{{\unskip\nobreak\hfill\penalty50\hskip2emo9  \hbox{}\nobreak\hfill \hbox{#1}\finalhyphendemerits=0}}o%  %  Define end-of-article marker.%h;\def\eoa{\farright{\vrule height1.5ex width1.5ex depth0pt}}t%f*%  Generate a blank page and a blank line.%o?\def\nullpage{\eject\line{}\vfil\eject}		% Define an empty pagem?\def\nullline{\break\hbox{}\hfil\break}		% Define an empty linea%h%  Start on a new page.e%e\def\newpage{\vfill\eject}%c%  Get rid of underfill errors%\C\def\ignoreunderfill{\vbadness=10000\hbadness=10000\tolerance=2000}uP%=-=-=-=-=-=-=-=-=-=-=-=-=-=-=-=-=-=-=-=-=-=-=-=-=-=-=-=-=-=-=-=-=-=-=-=-=-=-=-=$% 			Macros for bibliography entriesP%=-=-=-=-=-=-=-=-=-=-=-=-=-=-=-=-=-=-=-=-=-=-=-=-=-=-=-=-=-=-=-=-=-=-=-=-=-=-=-=%)%  Sample usage:%g%	\beginbibliography%	\bibbook{The Wolf's Hour}t7%	\ENUM	New York: Pocket Books, March 1989  (paperback)g%	\endbibliography%o*\newcount\enumno				% New counter - item #%t,%  \beginbibliography - Begin a bibliography%n2\def\beginbibliography{\begingroup\global\enumno=1	\tiny					% Use 8pt font@.	\parskip=1pt plus 1pt			% Skip up to 2 points"	}					% End of \beginbibliographyD\def\endbibliography{\par\endbiblist\endgroup}	% End of bibliography%m;%  \beginbiblist - Begin a list of bibliographic referencesr%l\def\beginbiblist{\begingroup &	\vglue0pt\parindent=30pt\parskip=0pt}%nM%  \beginanotherlist - Begin a list inside a list of bibliographic referencesy% !\def\beginanotherlist{\begingroup 	\divide\parskip by 2u5	\vglue\parskip\advance\parindent by10pt\parskip=0pt} '\def\endbiblist{\par\endgroup\vskip4pt}i"%  ENUM   - Number items in a list:%  ENum   - No number, but spaced as if number was presentN%  NoENUM - Only one reference is present.  Start reference where number would0%	    normally start (hanging into left column).%o%  Examples:% !%	\ENUM	First one			1.  First ones)%	\ENUM	Second one	Yields		2.  Second one\!%	\ENum	Third one			    Third ones!%	\NoENUM	Fourth one			Fourth one\%aA\def\ENUM#1\par{\item{\the\enumno.}\advance\enumno by 1 #1 \par }\5\def\ENum#1\par{\item{}\advance\enumno by 1 #1 \par }\D\def\NoENUM#1\par{\advance\enumno by 1\par\hang\hskip-10pt #1 \par }%t %  \bibshort, \bibbook, \bibview%	J%  Macros to begin a new bibliography entry for a short story, a book, andI%  and interviews.  These macros will terminate the previous bibliographyo1%  entry (if there is one) and begin a new entry.	%IF\def\bibshort#1{\ifnum\enumno>1 \bestbreak\endbiblist\fi	% Short story#	\noindent{\story{#1}}\beginbiblisto	}6\def\bibbook#1{\ifnum\enumno>1 \endbiblist\fi			% Book	\noindent{\sl #1}\beginbiblist 	};\def\bibview#1{\ifnum\enumno>1 \endbiblist\fi			% Interview		\noindent{#1}\beginbiblistr	}@\def\subbib#1{\hskip-20pt #1\hfill}		%Subheading for a bib entry7\def\bibsectitle#1{				%Title of bib section (BOOKS...)}/	\vskip 8pt plus1pt minus1pt		% Skip some spaces	\hrule					% Draw an hruley'	{\tenss #1}				% Add text in 10pt fonta5	\vglue 10pt plus1pt minus1pt		% Skip some more spacec	}P%=-=-=-=-=-=-=-=-=-=-=-=-=-=-=-=-=-=-=-=-=-=-=-=-=-=-=-=-=-=-=-=-=-=-=-=-=-=-=-=%d%  File:	QUOTE.TEX%I%  Author:	Hunter Goatleyb%%  Date:	August 14, 1991%c%  Abstract:%pG%	This file defines the macros \begindoublequotes and \enddoublequotes,A%	which let TeX replace the double-quote character (") with TeX'sh9%	left double-quote and right double-quote.  For example:h%i4%		"This is a test."     --->    ``This is a test.''% F%	The double-quote character is still available via \dq.  (\" is still %	treated as the umlaut accent.)% C%	This macro makes a couple of assumptions about the double-quotes:n%3A%	1.  Double-quotes are assumed to come in pairs.  When replacing\F%	    double-quotes, the macro alternates between `` and ''.  The only.%	    exception to this is noted in (2) below.F%	2.  A double-quote at the beginning of a paragraph is always treatedB%	    as ``.  This correctly handles the case where a quotation is(%	    continued into a second paragraph:% $%		"This is the first paragraph.\par4%		"This is the second paragraph of the same quote."%\@%	Normal TeX spacing after `` and '' is maintained by saving and%	restoring the \spacefactor.i%aP%%%%%%%%%%%%%%%%%%%%%%%%%%%%%%%%%%%%%%%%%%%%%%%%%%%%%%%%%%%%%%%%%%%%%%%%%%%%%%%%%t%  HOW IT WORKS:%rF%	The double-quote character (") is made active by \begindoublequotes.E%	The " macro keeps track of left-quote/right-quote pairs and inserts )%	the appropriate `` and '' in its place. %hG%	Each character has a \spacefactor associated with it, which specifiessF%	the amount of stretch or shrink that a space following the characterH%	can have.  Most characters have a factor of 1000, but some punctuationF%	marks have higher spacefactors, most notably the period, which has aD%	\spacefactor of 3000.  This means the space following a period canF%	stretch up to 3 times more than the space after a regular character,=%	accounting for the increased space at the end of sentences. %fG%	The `` and '' ligatures are assigned \spacefactor's of 0, so that thebH%	\spacefactor that is applied to the next character is the same as thatI%	of the character preceding the quotes.  Because " has been redefined aseI%	a macro, any spaces following " are swallowed by TeX.  It was necessary E%	to have this macro re-insert any needed space so that the followingc%	cases worked correctly:a%a@%		"This is a test," she said. --> ``This is a test,'' she said.>%		"This is in a list"; etc.   --> ``This is in a list''; etc.%e5%	Without the added space, the first example becomes:s%i%		``This is a test,''she said.c%lF%	The solution was to save the current \spacefactor before inserting a?%	right double-quote, then resetting the \spacefactor after thetD%	insertion.  The net effect was that the " macro has a \spacefactor3%	of 0, which matches the way TeX treats `` and ''.	%%P%%%%%%%%%%%%%%%%%%%%%%%%%%%%%%%%%%%%%%%%%%%%%%%%%%%%%%%%%%%%%%%%%%%%%%%%%%%%%%%%,{%					% Begin a group for which " is active2\catcode`\"=\active			% Make " an active character.\catcode`\@=11				% Make @ an active character%%%  \begindoublequotes2%E%	This macro makes " an active character, resets the control sequencel@%	\dblqu@te to L (left), and defines \dq as a replacement for ".%u9\gdef\begindoublequotes{%		% \begindoublequotes enables "s<    \global\catcode`\"=\active		% Make " an active character8    \global\chardef\dq=`\"		% Double-quote char. via \dqC    \global\let\dblqu@te=L		% Always start with a left double-quote,    }					% End of macro%lD%  Define the macro that will be executed whenever " is encountered.%d\gdef"{%B	%  If the double-quote is the first character in a new paragraph,?	%  make sure it becomes a left double-quote.  This case can besE	%  detected by checking to see if TeX is currently in vertical mode.@	%  If so, the double-quote is at the beginning of the paragraphA	%  (since " hasn't actually generated any horizontal mode tokenstB	%  yet, TeX is still in vertical mode).  If the mode is vertical,	%  set \dblqu@te equal to L.f	%+	\ifinner\else\ifvmode\let\dblqu@te=L\fi\fiu	%:	%  Now insert the appropriate left or right double-quote.	%:	%  If \dblqu@te is L, insert a `` and set \dblqu@te to R.	%(	\if L\dblqu@te``\global\let\dblqu@te=R%	%F	%  Otherwise, save the current \spacefactor, insert '', set \dblqu@te.	%  to L, and reset the original \spacefactor.	%	\else2	   \let\xxx=\spacefactor		% Save the \spacefactor>	   ''\global\let\dblqu@te=L%		% Insert '' and reset \dblqu@te/	   \spacefactor\xxx			% Reset the \spacefactorm#	\fi					% End of \if L\dblqu@te...n	}					% End of " macro }						% End of groupl\gdef\enddoublequotes{%i'	\catcode`\"=12				%Set " back to other"	}%P%%%%%%%%%%%%%%%%%%%%%%%%%%%%%%%%%%%%%%%%%%%%%%%%%%%%%%%%%%%%%%%%%%%%%%%%%%%%%%%%2%                           Header & Footer MacrosP%%%%%%%%%%%%%%%%%%%%%%%%%%%%%%%%%%%%%%%%%%%%%%%%%%%%%%%%%%%%%%%%%%%%%%%%%%%%%%%%%lL%  These macros implement the headers and footers for the newsletter format.K%  The macros accept three parameters: text that is to appear flush-left onlN%  the line, text that should be centered, and text that should be flush-rightK%  on the line.  Parameters can be omitted by specifying empty braces ({}).t%$<%  The following macros are defined for headers and footers:%&%	\evenpageheader{LEFT}{CENTER}{RIGHT}%%	\oddpageheader{LEFT}{CENTER}{RIGHT}s&%	\evenpagefooter{LEFT}{CENTER}{RIGHT}%%	\oddpagefooter{LEFT}{CENTER}{RIGHT}t%J%  If the headers/footers are the same for even & odd pages, the following0%  macros can be used instead of the four above:%\"%	\pageheader{LEFT}{CENTER}{RIGHT}"%	\pagefooter{LEFT}{CENTER}{RIGHT}%}(%  Additional header/footer definitions:%%?%	\pageheaderlinetrue		- A line should extend below header text5%	\pageheaderlinefalse		- Header does NOT have a line ?%	\pagefooterlinetrue		- A line should extend above footer text5%	\pagefooterlinefalse		- Footer does NOT have a line 7%	\headfootrule=Xpt		- Thickness of header/footer lines7%	\pageheaderskip=Xpt		- \vskip between header and liner7%	\pagefooterskip=Xpt		- \vskip between footer and linel4%	\headfont=\fontname		- Font to use for header text4%	\footfont=\fontname		- Font to use for footer text%b%  Example:l%- %	\pageheader{}{My Newsletter}{}&%	\pagefooter{October 1989}{}{\pageno}% J\newif\ifpageheaderline  \pageheaderlinefalse	% By default, no header lineJ\newif\ifpagefooterline  \pagefooterlinefalse	% By default, no footer lineN\newdimen\headfootrule   \headfootrule=0.50pt	% Height of header & footer ruleL\newdimen\pageheaderskip \pageheaderskip=4pt	% Space between header and ruleL\newdimen\pagefooterskip \pagefooterskip=4pt	% Space between rule and footerL\let\headfont=\twelverm \let\footfont=\twelverm	% Assign fonts for head/foot\def\@pageheader#1#2#3{%$	\ifpageheaderline			% If headerlineG	\vbox{\hbox to\normalhsize{{\headfont\rlap{#1}\hss{#2}\hss\llap{#3}}}%hD	\vskip\pageheaderskip\hrule height\headfootrule}% Do hbox and hrule	\else*	{\headfont\rlap{#1}\hss{#2}\hss\llap{#3}}	\fi	}\def\@pagefooter#1#2#3{%	\ifpagefooterline6	\vbox{\hrule height\headfootrule\vskip\pagefooterskip@	\hbox to\normalhsize{\footfont\rlap{#1}\hss{#2}\hss\llap{#3}}}%	\else*	{\footfont\rlap{#1}\hss{#2}\hss\llap{#3}}	\fi	}%o9% Define default headers and footers - null lines of texta%i3\def\@oddhead{\nullline}  \def\@evenhead{\nullline}o3\def\@oddfoot{\nullline}  \def\@evenfoot{\nullline}eB\def\@newhead{\headline{\ifodd\pageno\@oddhead\else\@evenhead\fi}}B\def\@newfoot{\footline{\ifodd\pageno\@oddfoot\else\@evenfoot\fi}}J\def\oddpageheader#1#2#3{\@newhead\def\@oddhead{\@pageheader{#1}{#2}{#3}}}L\def\evenpageheader#1#2#3{\@newhead\def\@evenhead{\@pageheader{#1}{#2}{#3}}}J\def\oddpagefooter#1#2#3{\@newfoot\def\@oddfoot{\@pagefooter{#1}{#2}{#3}}}L\def\evenpagefooter#1#2#3{\@newfoot\def\@evenfoot{\@pagefooter{#1}{#2}{#3}}}%hP%  If no difference between even and odd pages, just define both to be the same.%nL\def\pageheader#1#2#3{\evenpageheader{#1}{#2}{#3}\oddpageheader{#1}{#2}{#3}}L\def\pagefooter#1#2#3{\evenpagefooter{#1}{#2}{#3}\oddpagefooter{#1}{#2}{#3}}P%%%%%%%%%%%%%%%%%%%%%%%%%%%%%%%%%%%%%%%%%%%%%%%%%%%%%%%%%%%%%%%%%%%%%%%%%%%%%%%%%o9%  Command for "poor man's bold":  \pmb   (use sparingly)e%b3\def\pmb#1{\setbox0=\hbox{#1}%			% Copy box to box03    \leavevmode\hbox{%				% Make an hbox that holdslC    \kern-.025em\copy0\kern-\wd0%		% Move left 1/4 em and copy box0oA    \kern.05em\copy0\kern-\wd0%			% Move right 1/4 em and copy it F    \kern-.025em\raise.0433em\box0 }}		% Raise a little and copy again%e9%  Define dots for ending sentences (4 dots instead of 3)h%}1\def\eldots{\mathinner{\ldotp\ldotp\ldotp\ldotp}}i9\def\edots{\relax\ifmmode\eldots\else$\m@th\eldots\,$\fi}x?\def\ellip{\hskip.2em\ifmmode\ldots\else$\ldots$\fi\hskip.25em}n%hM%  Define macros \ital and \slant to switch to italic (\it) and slanted (\sl)iI%  respectively.  These macros automatically insert the italic correctioniB%  unless the next character is a period or a comma.  Based on the?%  \predict macro presented in _TeX for the Impatient_, p. 233.o%i*%  These macros use \toks0 as a temporary.%nM%  The \futurelet\it@next in \ital and \slant defines \it@next to be whateverbG%  the character following the parameter is.  \d@slant checks to see ifeI%  \it@next is a comma or period; if it is neither, the italic correction%%  (\/) is included.% A\def\ital#1{\toks0={#1}\let\slf@nt=\it\futurelet\it@next\d@slant}hB\def\slant#1{\toks0={#1}\let\slf@nt=\sl\futurelet\it@next\d@slant}!\def\d@slant{{\slf@nt\the\toks0}%	+	\ifx\it@next,%			% If \it@next not a comma /	\else\ifx\it@next.%		% ... and is not a periodt,	\else\/%			% ... insert the correction (\/)	\fi\fi%				% ...n,	\let\it@next=\relax%		% "Undefine" \it@next	}5\def\book#1{\ital{#1}}				%For ease, define \book too}%	<%  Important - make "@" a valid alphanumeric character again%h7\catcode`\@=12				% Follow TeX's lead on variable nameso*\tenpoint					% Default point size is 10pt