%----------------------------------------------------------------------------
% ----- File:        internals.tex 
% ----- Author:      Rainer Menzner (rmz@neuroinformatik.ruhr-uni-bochum.de)
% ----- Date:        1999-05-05
% ----- Description: This file is part of the t1lib-documentation.
% ----- Copyright:   t1lib is copyrighted (c) Rainer Menzner, 1996-1999. 
%                    As of version 0.5, t1lib is distributed under the
%                    GNU General Public Library License. The
%                    conditions can be found in the files LICENSE and
%                    LGPL, which should reside in the toplevel
%                    directory of the distribution.  Please note that 
%                    there are parts of t1lib that are subject to
%                    other licenses:
%                    The parseAFM-package is copyrighted by Adobe Systems
%                    Inc.
%                    The type1 rasterizer is copyrighted by IBM and the
%                    X11-consortium.
% ----- Warranties:  Of course, there's NO WARRANTY OF ANY KIND :-)
% ----- Credits:     I want to thank IBM and the X11-consortium for making
%                    their rasterizer freely available.
%                    Also thanks to Piet Tutelaers for his ps2pk, from
%                    which I took the rasterizer sources in a format
%                    independent from X11.
%                    Thanks to all people who make free software living!
%----------------------------------------------------------------------------

\newpage
\section{Internals (incomplete)}
\label{internals}%
\vskip1cm
\hrule
\vskip0.5cm
\begin{center}
\sffamily\large
{\Huge\bfseries Note!}\\
This section is still very incomplete and some facts are not true
anymore. This should be kept in mind. Currently I have no time to
write this section. But I try to keep figure \ref{figure:t1data}
consistent to the current releases. This may lead to inconsistencies
between the text and the figure. 
\end{center}
\hrule
\vskip1cm
In this section, some information on internals of \tonelib\ is given. There is
no need for an average user to read this section although having understood
what is going on internally might be helpful if problems occur. 

The basic idea of this section is to describe the data structures and to give
information on when they are initialized, allocated and referenced. Figure
\ref{figure:t1data} shows an image of the data-structures for the special case that
the font with ID 0 has already been loaded and several size-instances have
already been created. 
%-- Figure: The data structures of t1lib
\begin{figure}
\begin{center}
\includegraphics*[angle=90]{t1_data.eps}
\end{center}
\hrule\vskip3mm\small
\caption{\label{figure:t1data}The internal data structures of \tonelib. The
underlying substructures are shown only for the first font
{\tt FontID=0}.} 
\end{figure}
As the figure indicates, the complete area may be split into three
different sub-areas, thereby pointing out their logical functions.

\subsection{Level 0: Global Data}
\label{globaldata}%
This area contains information needed for the overall organization of the
\tonelib. Its contents and its size are thus determined at the time
\tonelib\ is initialized. This is done based on the contents of the
configuration file and the fontdatabase file. The entries in detail are:
\begin{itemize}
\item {\tt Filename-Searchpaths}: This entry essentially does not depend on
  any other data. It consists of 4 \verb+\0+-terminated strings that are read
  from the configuration file. They are referenced internally by the global
  symbols
  \verb+PFAB_ptr+, \verb+AFM_ptr+, \verb+ENC_ptr+ and \verb+FDB_ptr+
  respectively. All these are declared as \verb+unsigned char *+. These
  strings are used by \tonelib\ to locate the respective file types. If no
  configuration file exists or some path declaration is missing, the
  corresponding searchpath is set to ``\verb+.+'', causing \tonelib\ to only
  search the current working directory. 
\item \verb+no_fonts_ini+: This value is assigned after examining the
  fontdatabase file. It is meant to store the number of fonts initially
  declared in the fontdatabase file. In other words, it is assigned the
  integer number located on the first line of the fontdatabase file.
\item \verb+no_fonts+: The number of actually allocated fonts. Initially, this
  quantity is identical to \verb+no_fonts_ini+. But if one creates a new
  logical font by calling \verb+T1_CopyFont()+ this counter is incremented to
  keep track of allocated fonts. \verb+no_fonts+ thus represents most large
  \verb+FontID+ minus 1 that makes sense to specify to any function of
  \tonelib. 
\item \verb+no_fonts_limit+: The number of fonts for which memory is currently
  allocated. This also is initially set to \verb+no_fonts_ini+ and is
  automatically enlarged to a multiple of the initial value if a call to
  \verb+T1_CopyFont()+ requires additional memory for logical fonts (see
  \ref{logicalfonts}). 
\item \verb+bitmap_pad+: This variable contains the number of bits to which
  scanlines of bitmaps and antialiased bitmaps are padded. It is set during
  initialization, either to a default value or to the value the application
  specified before starting initialization using
  \verb+T1_SetBimapPad()+. Allowed values are currently `8', `16' and `32'. 
\item \verb+endian+: During initialization the hardware is checked for
  representation of data in memory. If Big Endian is used, \verb+endian+ is
  set to \verb+1+ and otherwise it is set to \verb+0+. \verb+endian+ is needed
  at several times when an application or \tonelib\ itself must know the
  byte order of words and long words.
\item \verb+pFontArray+: This a pointer to an array of structures whose type
  is referred to as \\
  \verb+FONTPRIVATE+ in \tonelib. The contents of these
  structures will be described below. After \tonelib\ has been
  initialized, memory is allocated for exactly \verb+no_fonts_ini+
  structures. This memory pool may be enlarged later if the one wants to make
  use of logical fonts, for example. The data in these structures initially is
  not specified. It is written with meaningful values when a font is loaded
  into memory. The index to access this array-elements is the well known font
  identification number (\verb+FontID+).
\item \verb+pFontFileNameIDArray+: A pointer to a memory area where the
  font file names corresponding to the \verb+FontID+s are stored. During
  initialization, \tonelib\ looks for font files with extension \verb+.pfa+
  and \verb+.pfb+. The basename of the file found is stored in this area and
  if the font is to be loaded later, its font file name is looked up here.
\end{itemize}
We should now discuss the entries of the structures of type
\verb+FONTPRIVATE+. The term \verb+FONTPRIVATE+ indicates that every font
needs its own structure area. As mentioned earlier, this area is initialized
when the corresponding font is loaded.
\begin{itemize}
\item \verb+pAFMData+: A pointer to a memory area where Adobe Font Metric data
  of the font is stored. The memory area itself is build by the
  \verb+parse_afm+-package which is supplied by Adobe System and included in
  \tonelib. This happens while a font is loaded. In case there is no AFM file
  for the font in question, this pointer is given the value \verb+NULL+. 
\item \verb+pType1Data+: A pointer to the data area where the Type 1
  information is stored. The known PostScript Type 1 objects
  Charstrings-dictionary, Subroutines, Othersubroutines and
  Fontinfo-dictionary are located here. The memory is filled with data during
  parsing the font file when the font is loaded.
\item \verb+pFontEnc+: A pointer to an optional external encoding
  vector. During initialization, this pointer is set to \verb+NULL+, thus
  indicating that by default the font's internal encoding should be used.
  If a font is reencoded using a previously loaded encoding vector from an
  encoding file, this pointer simply is assigned the address of a valid
  encoding array somewhere in memory.
\item \verb+vm_base+: The base address of the virtual memory required by the
  font. Unlike the original rasterizer, which allocated virtual memory in
  chunks of a fixed size, t1lib uses another principle. Since it is \`a
  priori not obvious how many virtual memory a font consumes, \tonelib\ tries
  to load a font repeatedly and increases the amount of virtual memory during
  every trial. In order not to waste memory, the memory is reallocated to the
  needed size when the font is completely loaded. Finally, the starting
  address of the virtual memory is needed when a font is to be unloaded and
  the memory it consumes is to be given back to system. 
\item \verb+pFontSizeDeps+: A pointer to the area where the size dependent
  data is to be stored. This data essentially consists of generated glyphs
  plus some administrative item (see \ref{sizedependentfontdata}).
\item \verb+FontMatrix+: A matrix of four \verb+double+-values specifying the
  font matrix. If the FontInfo-dictionary of the font file defines a
  FontMatrix, it is copied to this location. If not, a default matrix is
  used which does no transformation and scales to $1/1000$~bp.
\item \verb+FontTransform+: A matrix that will be concatenated with the
  FontMatrix to produce the final transformation of the characters. It is this
  matrix that is modified if a font is to be slanted or extended.
\item \verb+slant+: A slant factor for the current font. Note that this 
  value is initially 0, even for italic font. Only artificially slanting a
  font leads to values different from 0.
\item \verb+extend+: The horizontal extension factor for the current font. Its
  default value is 1 and the font is thus rendered at its natural width.
\item \verb+physical+: This is a switch that marks a font either being
  ``physical'' or ``logical''. A physical font by definition is a font for
  which a Type 1 font file is available and for which thus Level 1
  (size-independent) data is present (see Fig.\ 5.1). In contrast, the term 
  ``logical font'' refers to a structure of type \verb+FONTPRIVATE+ whose
  entry \verb+pType1Data+ points to Level 1 data of another (physical)
  font. This \verb+FONTPRIVATE+-structure is created by calling
  \verb+T1_CopyFont()+ with the identification number of an existing physical
  font as argument (see \ref{logicalfonts}).
\item \verb+refcount+: This counter keeps track on how much logical fonts
  refer to the physical font that is represented by the current structure of
  type \verb+FONTPRIVATE+. In this since, \verb+refcount+ is only meaningful for
  physical fonts. It is necessary to keep track of the reference of logical
  fonts because if this font would be removed from memory by calling
  \verb+T1_DeleteFont()+, the Level 1 font data memory area would be given back
  to the system but the logical fonts referring to that font would still
  expect to find Type 1 or Font Metric data at this address. By checking
  \verb+refcount+, \verb+T1_DeleteFont()+ can check for logical fonts referring
  to the font in question and prevent from removing this font from memory.
  
  In structures describing logical fonts, \verb+refcount+ is used to
  store the information which physical font this logical font is
  referring to. This information is also needed by
  \verb+T1_DeleteFont()+ since when removing logical fonts, the
  reference counter of the corresponding physical font has to be
  decremented.
\item \verb+space_position+: This variable stores the encoding index of the
  ``space''-character of the current font. If the space character does not
  appear in the current font's encoding, \verb+space_position+ is assigned
  -1. It follows that \verb+space_position+
  is assigned when (1) a font loaded and (2) every time a
  font is reencoded. Why is it convenient to store the position of the space
  character in the encoding vector? The properties of the space character are
  set apart from the other characters' properties not only by the fact that it
  does not produce any colored pixels but also by that it may shrink and
  stretch in \tonelib. As a consequence a space character is treated by simply
  inserting a horizontal escapement of the width of the space
  character---corrected by the quantity \verb+space_off+ that a user may
  specify (see \ref{generatingbitmaps}). This involves always checking every
  character for being the space character and since the encoding principle is
  used in \tonelib, every check needs a call to \verb+strcmp()+. This overhead
  is avoided if the position of space is stored.
\end{itemize} 
\subsection{Level 1: Size-Independent Font Data}
\label{sizeindependentfontdata}%
Size-independent data may be split into three categories as indicated in
figure \ref{figure:t1data}. The external encoding is optional and is generated
by loading an encoding file as described in \ref{encoding}. It is simply an
array of 256 pointers to \verb+unsigned char+ and an ensemble of 256
\verb+\0+-terminated strings. Each pointer references one of the 256 strings
in order. The strings are the characters' names to be defined in a
\tonelib-encoding file.

The internal Type 1 data structures hold all data specified in a type font
file. I do not want to describe these data structures here, because this could
fill a book. Adobe has made the description of the Type 1 font format
available to the public. 

The Adobe Font Metrics area is entirely created by the
\verb+parse_afm+-package. Adobe has made this available by means of the file
\verb+parseAFM.shar+ which is a shell-archive and included in \tonelib\ in the
subdirectory \verb+parse_afm+.

\subsection{Level 2: Size-Dependent Font Data}
\label{sizedependentfontdata}%

$\ldots$

%%% Local Variables: 
%%% mode: latex
%%% TeX-master: "t1lib_doc"
%%% End: 
