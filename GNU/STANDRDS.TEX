\input texinfo @c -*-texinfo-*-
@c %**start of header
@setfilename standards.info
@settitle GNU Coding Standards
@c %**end of header

@setchapternewpage off

@ifinfo
Copyright (C) 1992, 1993 Free Software Foundation
Permission is granted to make and distribute verbatim copies of
this manual provided the copyright notice and this permission notice
are preserved on all copies.

@ignore
Permission is granted to process this file through TeX and print the
results, provided the printed document carries copying permission
notice identical to this one except for the removal of this paragraph
(this paragraph not being relevant to the printed manual).
@end ignore

Permission is granted to copy and distribute modified versions of this
manual under the conditions for verbatim copying, provided that the entire
resulting derived work is distributed under the terms of a permission
notice identical to this one.

Permission is granted to copy and distribute translations of this manual
into another language, under the above conditions for modified versions,
except that this permission notice may be stated in a translation approved
by the Free Software Foundation.
@end ifinfo

@titlepage
@title GNU Coding Standards
@author Richard Stallman
@author last updated 19 July 1993
@c Note date also appears below.
@page

@vskip 0pt plus 1filll
Copyright @copyright{} 1992, 1993 Free Software Foundation

Permission is granted to make and distribute verbatim copies of
this manual provided the copyright notice and this permission notice
are preserved on all copies.

Permission is granted to copy and distribute modified versions of this
manual under the conditions for verbatim copying, provided that the entire
resulting derived work is distributed under the terms of a permission
notice identical to this one.

Permission is granted to copy and distribute translations of this manual
into another language, under the above conditions for modified versions,
except that this permission notice may be stated in a translation approved
by Free Software Foundation.
@end titlepage

@ifinfo
@node Top, Reading Non-Free Code, (dir), (dir)
@top Version

Last updated 19 July 1993.
@c Note date also appears above.
@end ifinfo

@menu
* Reading Non-Free Code::	Referring to Proprietary Programs
* Contributions::		Accepting Contributions
* Change Logs::			Recording Changes
* Compatibility::		Compatibility with Other Implementations
* Makefile Conventions::	Makefile Conventions
* Configuration::		How Configuration Should Work
* Source Language::		Using Languages Other Than C
* Formatting::			Formatting Your Source Code
* Comments::			Commenting Your Work
* Syntactic Conventions::	Clean Use of C Constructs
* Names::			Naming Variables and Functions
* Using Extensions::		Using Non-standard Features
* Semantics::			Program Behavior for All Programs
* Errors::			Formatting Error Messages
* Libraries::			Library Behavior
* Portability::			Portability As It Applies to GNU
* User Interfaces::		Standards for Command Line Interfaces
* Documentation::		Documenting Programs
* Releases::			Making Releases
@end menu

@node Reading Non-Free Code
@chapter Referring to Proprietary Programs

Don't in any circumstances refer to Unix source code for or during
your work on GNU!  (Or to any other proprietary programs.)

If you have a vague recollection of the internals of a Unix program,
this does not absolutely mean you can't write an imitation of it, but
do try to organize the imitation internally along different lines,
because this is likely to make the details of the Unix version
irrelevant and dissimilar to your results.

For example, Unix utilities were generally optimized to minimize
memory use; if you go for speed instead, your program will be very
different.  You could keep the entire input file in core and scan it
there instead of using stdio.  Use a smarter algorithm discovered more
recently than the Unix program.  Eliminate use of temporary files.  Do
it in one pass instead of two (we did this in the assembler).

Or, on the contrary, emphasize simplicity instead of speed.  For some
applications, the speed of today's computers makes simpler algorithms
adequate.

Or go for generality.  For example, Unix programs often have static
tables or fixed-size strings, which make for arbitrary limits; use
dynamic allocation instead.  Make sure your program handles NULs and
other funny characters in the input files.  Add a programming language
for extensibility and write part of the program in that language.

Or turn some parts of the program into independently usable libraries.
Or use a simple garbage collector instead of tracking precisely when
to free memory, or use a new GNU facility such as obstacks.


@node Contributions
@chapter Accepting Contributions

If someone else sends you a piece of code to add to the program you are
working on, we need legal papers to use it---the same sort of legal
papers we will need to get from you.  @emph{Each} significant
contributor to a program must sign some sort of legal papers in order
for us to have clear title to the program.  The main author alone is not
enough.

So, before adding in any contributions from other people, tell us
so we can arrange to get the papers.  Then wait until we tell you
that we have received the signed papers, before you actually use the
contribution.

This applies both before you release the program and afterward.  If
you receive diffs to fix a bug, and they make significant change, we
need legal papers for it.

You don't need papers for changes of a few lines here or there, since
they are not significant for copyright purposes.  Also, you don't need
papers if all you get from the suggestion is some ideas, not actual code
which you use.  For example, if you write a different solution to the
problem, you don't need to get papers.

I know this is frustrating; it's frustrating for us as well.  But if
you don't wait, you are going out on a limb---for example, what if the
contributor's employer won't sign a disclaimer?  You might have to take
that code out again!

The very worst thing is if you forget to tell us about the other
contributor.  We could be very embarrassed in court some day as a
result.

@node Change Logs
@chapter Change Logs

Keep a change log for each directory, describing the changes made to
source files in that directory.  The purpose of this is so that people
investigating bugs in the future will know about the changes that
might have introduced the bug.  Often a new bug can be found by
looking at what was recently changed.  More importantly, change logs
can help eliminate conceptual inconsistencies between different parts
of a program; they can give you a history of how the conflicting
concepts arose.

Use the Emacs command @kbd{M-x add-change} to start a new entry in the
change log.  An entry should have an asterisk, the name of the changed
file, and then in parentheses the name of the changed functions,
variables or whatever, followed by a colon.  Then describe the changes
you made to that function or variable.

Separate unrelated entries with blank lines.  When two entries
represent parts of the same change, so that they work together, then
don't put blank lines between them.  Then you can omit the file name
and the asterisk when successive entries are in the same file.

Here are some examples:

@example
* register.el (insert-register): Return nil.
(jump-to-register): Likewise.

* sort.el (sort-subr): Return nil.

* tex-mode.el (tex-bibtex-file, tex-file, tex-region):
Restart the tex shell if process is gone or stopped.
(tex-shell-running): New function.

* expr.c (store_one_arg): Round size up for move_block_to_reg.
(expand_call): Round up when emitting USE insns.
* stmt.c (assign_parms): Round size up for move_block_from_reg.
@end example

There's no need to describe here the full purpose of the changes or how
they work together.  It is better to put this explanation in comments in
the code.  That's why just ``New function'' is enough; there is a
comment with the function in the source to explain what it does.

However, sometimes it is useful to write one line to describe the
overall purpose of a large batch of changes.

You can think of the change log as a conceptual ``undo list'' which
explains how earlier versions were different from the current version.
People can see the current version; they don't need the change log
to tell them what is in it.  What they want from a change log is a
clear explanation of how the earlier version differed.

When you change the calling sequence of a function in a simple
fashion, and you change all the callers of the function, there is no
need to make individual entries for all the callers.  Just write in
the entry for the function being called, ``All callers changed.''

When you change just comments or doc strings, it is enough to write an
entry for the file, without mentioning the functions.  Write just,
``Doc fix.''  There's no need to keep a change log for documentation
files.  This is because documentation is not susceptible to bugs that
are hard to fix.  Documentation does not consist of parts that must
interact in a precisely engineered fashion; to correct an error, you
need not know the history of the erroneous passage.


@node Compatibility
@chapter Compatibility with Other Implementations

With certain exceptions, utility programs and libraries for GNU should
be upward compatible with those in Berkeley Unix, and upward compatible
with @sc{ANSI} C if @sc{ANSI} C specifies their behavior, and upward
compatible with @sc{POSIX} if @sc{POSIX} specifies their behavior.

When these standards conflict, it is useful to offer compatibility
modes for each of them.

@sc{ANSI} C and @sc{POSIX} prohibit many kinds of extensions.  Feel
free to make the extensions anyway, and include a @samp{--ansi} or
@samp{--compatible} option to turn them off.  However, if the extension
has a significant chance of breaking any real programs or scripts,
then it is not really upward compatible.  Try to redesign its
interface.

When a feature is used only by users (not by programs or command
files), and it is done poorly in Unix, feel free to replace it
completely with something totally different and better.  (For example,
vi is replaced with Emacs.)  But it is nice to offer a compatible
feature as well.  (There is a free vi clone, so we offer it.)

Additional useful features not in Berkeley Unix are welcome.
Additional programs with no counterpart in Unix may be useful,
but our first priority is usually to duplicate what Unix already
has.

@comment The makefile standards are in a separate file that is also
@comment included by make.texinfo.  Done by roland@gnu.ai.mit.edu on 1/6/93.
@include make-stds.texi

@node Configuration
@chapter How Configuration Should Work

Each GNU distribution should come with a shell script named
@code{configure}.  This script is given arguments which describe the
kind of machine and system you want to compile the program for.

The @code{configure} script must record the configuration options so
that they affect compilation.

One way to do this is to make a link from a standard name such as
@file{config.h} to the proper configuration file for the chosen system.
If you use this technique, the distribution should @emph{not} contain a
file named @file{config.h}.  This is so that people won't be able to
build the program without configuring it first.

Another thing that @code{configure} can do is to edit the Makefile.  If
you do this, the distribution should @emph{not} contain a file named
@file{Makefile}.  Instead, include a file @file{Makefile.in} which
contains the input used for editing.  Once again, this is so that people
won't be able to build the program without configuring it first.

If @code{configure} does write the @file{Makefile}, then @file{Makefile}
should have a target named @file{Makefile} which causes @code{configure}
to be rerun, setting up the same configuration that was set up last
time.  The files that @code{configure} reads should be listed as
dependencies of @file{Makefile}.

All the files which are output from the @code{configure} script should
have comments at the beginning explaining that they were generated
automatically using @code{configure}.  This is so that users won't think
of trying to edit them by hand.

The @code{configure} script should write a file named @file{config.status}
which describes which configuration options were specified when the
program was last configured.  This file should be a shell script which,
if run, will recreate the same configuration.

The @code{configure} script should accept an option of the form
@samp{--srcdir=@var{dirname}} to specify the directory where sources are found
(if it is not the current directory).  This makes it possible to build
the program in a separate directory, so that the actual source directory
is not modified.

If the user does not specify @samp{--srcdir}, then @code{configure} should
check both @file{.} and @file{..} to see if it can find the sources.  If
it finds the sources in one of these places, it should use them from
there.  Otherwise, it should report that it cannot find the sources, and
should exit with nonzero status.

Usually the easy way to support @samp{--srcdir} is by editing a
definition of @code{VPATH} into the Makefile.  Some rules may need to
refer explicitly to the specified source directory.  To make this
possible, @code{configure} can add to the Makefile a variable named
@code{srcdir} whose value is precisely the specified directory.

The @code{configure} script should also take an argument which specifies the
type of system to build the program for.  This argument should look like
this:

@example
@var{cpu}-@var{company}-@var{system}
@end example

For example, a Sun 3 might be @samp{m68k-sun-sunos4.1}.

The @code{configure} script needs to be able to decode all plausible
alternatives for how to describe a machine.  Thus, @samp{sun3-sunos4.1}
would be a valid alias.  So would @samp{sun3-bsd4.2}, since SunOS is
basically @sc{BSD} and no other @sc{BSD} system is used on a Sun.  For many
programs, @samp{vax-dec-ultrix} would be an alias for
@samp{vax-dec-bsd}, simply because the differences between Ultrix and
@sc{BSD} are rarely noticeable, but a few programs might need to distinguish
them.

There is a shell script called @file{config.sub} that you can use
as a subroutine to validate system types and canonicalize aliases.

Other options are permitted to specify in more detail the software
or hardware are present on the machine:

@table @samp
@item --with-@var{package}
The package @var{package} will be installed, so configure this package
to work with @var{package}.

Possible values of @var{package} include @samp{x}, @samp{gnu-as} (or
@samp{gas}), @samp{gnu-ld}, @samp{gnu-libc}, and @samp{gdb}.

@item --nfp
The target machine has no floating point processor.

@item --gas
The target machine assembler is GAS, the GNU assembler.
This is obsolete; use @samp{--with-gnu-as} instead.

@item --x
The target machine has the X Window System installed.
This is obsolete; use @samp{--with-x} instead.
@end table

All @code{configure} scripts should accept all of these ``detail''
options, whether or not they make any difference to the particular
package at hand.  In particular, they should accept any option that
starts with @samp{--with-}.  This is so users will be able to configure
an entire GNU source tree at once with a single set of options.

Packages that perform part of compilation may support cross-compilation.
In such a case, the host and target machines for the program may be
different.  The @code{configure} script should normally treat the
specified type of system as both the host and the target, thus producing
a program which works for the same type of machine that it runs on.

The way to build a cross-compiler, cross-assembler, or what have you, is
to specify the option @samp{--host=@var{hosttype}} when running
@code{configure}.  This specifies the host system without changing the
type of target system.  The syntax for @var{hosttype} is the same as
described above.

Programs for which cross-operation is not meaningful need not accept the
@samp{--host} option, because configuring an entire operating system for
cross-operation is not a meaningful thing.

Some programs have ways of configuring themselves automatically.  If
your program is set up to do this, your @code{configure} script can simply
ignore most of its arguments.


@node Source Language
@chapter Using Languages Other Than C

Using a language other than C is like using a non-standard feature: it
will cause trouble for users.  Even if GCC supports the other language,
users may find it inconvenient to have to install the compiler for that
other language in order to build your program.  So please write in C.

There are three exceptions for this rule:

@itemize @bullet
@item
It is okay to use a special language if the same program contains an
interpreter for that language.

Thus, it is not a problem that GNU Emacs contains code written in Emacs
Lisp, because it comes with a Lisp interpreter.

@item
It is okay to use another language in a tool specifically intended for
use with that language.

This is okay because the only people who want to build the tool will be
those who have installed the other language anyway.

@item
If an application is not of extremely widespread interest, then perhaps
it's not important if the application is inconvenient to install.
@end itemize

@node Formatting
@chapter Formatting Your Source Code

It is important to put the open-brace that starts the body of a C
function in column zero, and avoid putting any other open-brace or
open-parenthesis or open-bracket in column zero.  Several tools look
for open-braces in column zero to find the beginnings of C functions.
These tools will not work on code not formatted that way.

It is also important for function definitions to start the name of the
function in column zero.  This helps people to search for function
definitions, and may also help certain tools recognize them.  Thus,
the proper format is this:

@example
static char *
concat (s1, s2)        /* Name starts in column zero here */
     char *s1, *s2;
@{                     /* Open brace in column zero here */
  @dots{}
@}
@end example

@noindent
or, if you want to use @sc{ANSI} C, format the definition like this:

@example
static char *
concat (char *s1, char *s2)
@{
  @dots{}
@}
@end example

In @sc{ANSI} C, if the arguments don't fit nicely on one line,
split it like this:

@example
int
lots_of_args (int an_integer, long a_long, short a_short,
              double a_double, float a_float)
@dots{}
@end example

For the body of the function, we prefer code formatted like this:

@example
if (x < foo (y, z))
  haha = bar[4] + 5;
else
  @{
    while (z)
      @{
        haha += foo (z, z);
        z--;
      @}
    return ++x + bar ();
  @}
@end example

We find it easier to read a program when it has spaces before the
open-parentheses and after the commas.  Especially after the commas.

When you split an expression into multiple lines, split it
before an operator, not after one.  Here is the right way:

@example
if (foo_this_is_long && bar > win (x, y, z)
    && remaining_condition)
@end example

Try to avoid having two operators of different precedence at the same
level of indentation.  For example, don't write this:

@example
mode = (inmode[j] == VOIDmode
        || GET_MODE_SIZE (outmode[j]) > GET_MODE_SIZE (inmode[j])
        ? outmode[j] : inmode[j]);
@end example

Instead, use extra parentheses so that the indentation shows the nesting:

@example
mode = ((inmode[j] == VOIDmode
         || (GET_MODE_SIZE (outmode[j]) > GET_MODE_SIZE (inmode[j])))
        ? outmode[j] : inmode[j]);
@end example

Insert extra parentheses so that Emacs will indent the code properly.
For example, the following indentation looks nice if you do it by hand,
but Emacs would mess it up:

@example
v = rup->ru_utime.tv_sec*1000 + rup->ru_utime.tv_usec/1000
    + rup->ru_stime.tv_sec*1000 + rup->ru_stime.tv_usec/1000;
@end example

But adding a set of parentheses solves the problem:

@example
v = (rup->ru_utime.tv_sec*1000 + rup->ru_utime.tv_usec/1000
     + rup->ru_stime.tv_sec*1000 + rup->ru_stime.tv_usec/1000);
@end example

Format do-while statements like this:

@example
do
  @{
    a = foo (a);
  @}
while (a > 0);
@end example

Please use formfeed characters (control-L) to divide the program into
pages at logical places (but not within a function).  It does not matter
just how long the pages are, since they do not have to fit on a printed
page.  The formfeeds should appear alone on lines by themselves.


@node Comments
@chapter Commenting Your Work

Every program should start with a comment saying briefly what it is for.
Example: @samp{fmt - filter for simple filling of text}.

Please put a comment on each function saying what the function does,
what sorts of arguments it gets, and what the possible values of
arguments mean and are used for.  It is not necessary to duplicate in
words the meaning of the C argument declarations, if a C type is being
used in its customary fashion.  If there is anything nonstandard about
its use (such as an argument of type @code{char *} which is really the
address of the second character of a string, not the first), or any
possible values that would not work the way one would expect (such as,
that strings containing newlines are not guaranteed to work), be sure
to say so.

Also explain the significance of the return value, if there is one.

Please put two spaces after the end of a sentence in your comments, so
that the Emacs sentence commands will work.  Also, please write
complete sentences and capitalize the first word.  If a lower-case
identifer comes at the beginning of a sentence, don't capitalize it!
Changing the spelling makes it a different identifier.  If you don't
like starting a sentence with a lower case letter, write the sentence
differently (e.g., ``The identifier lower-case is @dots{}'').

The comment on a function is much clearer if you use the argument
names to speak about the argument values.  The variable name itself
should be lower case, but write it in upper case when you are speaking
about the value rather than the variable itself.  Thus, ``the inode
number NODE_NUM'' rather than ``an inode''.

There is usually no purpose in restating the name of the function in
the comment before it, because the reader can see that for himself.
There might be an exception when the comment is so long that the function
itself would be off the bottom of the screen.

There should be a comment on each static variable as well, like this:

@example
/* Nonzero means truncate lines in the display;
   zero means continue them.  */
int truncate_lines;
@end example

Every @samp{#endif} should have a comment, except in the case of short
conditionals (just a few lines) that are not nested.  The comment should
state the condition of the conditional that is ending, @emph{including
its sense}.  @samp{#else} should have a comment describing the condition
@emph{and sense} of the code that follows.  For example:

@example
#ifdef foo
  @dots{}
#else /* not foo */
  @dots{}
#endif /* not foo */
@end example

@noindent
but, by contrast, write the comments this way for a @samp{#ifndef}:

@example
#ifndef foo
  @dots{}
#else /* foo */
  @dots{}
#endif /* foo */
@end example


@node Syntactic Conventions
@chapter Clean Use of C Constructs

Please explicitly declare all arguments to functions.
Don't omit them just because they are @code{int}s.

Declarations of external functions and functions to appear later in the
source file should all go in one place near the beginning of the file
(somewhere before the first function definition in the file), or else
should go in a header file.  Don't put @code{extern} declarations inside
functions.

It used to be common practice to use the same local variables (with
names like @code{tem}) over and over for different values within one
function.  Instead of doing this, it is better declare a separate local
variable for each distinct purpose, and give it a name which is
meaningful.  This not only makes programs easier to understand, it also
facilitates optimization by good compilers.  You can also move the
declaration of each local variable into the smallest scope that includes
all its uses.  This makes the program even cleaner.

Don't use local variables or parameters that shadow global identifiers.

Don't declare multiple variables in one declaration that spans lines.
Start a new declaration on each line, instead.  For example, instead
of this:

@example
int    foo,
       bar;
@end example

@noindent
write either this:

@example
int foo, bar;
@end example

@noindent
or this:

@example
int foo;
int bar;
@end example

@noindent
(If they are global variables, each should have a comment preceding it
anyway.)

When you have an @code{if}-@code{else} statement nested in another
@code{if} statement, always put braces around the @code{if}-@code{else}.
Thus, never write like this:

@example
if (foo)
  if (bar)
    win ();
  else
    lose ();
@end example

@noindent
always like this:

@example
if (foo)
  @{
    if (bar)
      win ();
    else
      lose ();
  @}
@end example

If you have an @code{if} statement nested inside of an @code{else}
statement, either write @code{else if} on one line, like this,

@example
if (foo)
  @dots{}
else if (bar)
  @dots{}
@end example

@noindent
with its @code{then}-part indented like the preceding @code{then}-part,
or write the nested @code{if} within braces like this:

@example
if (foo)
  @dots{}
else
  @{
    if (bar)
      @dots{}
  @}
@end example

Don't declare both a structure tag and variables or typedefs in the
same declaration.  Instead, declare the structure tag separately
and then use it to declare the variables or typedefs.

Try to avoid assignments inside @code{if}-conditions.  For example,
don't write this:

@example
if ((foo = (char *) malloc (sizeof *foo)) == 0)
  fatal ("virtual memory exhausted");
@end example

@noindent
instead, write this:

@example
foo = (char *) malloc (sizeof *foo);
if (foo == 0)
  fatal ("virtual memory exhausted");
@end example

Don't make the program ugly to placate @code{lint}.  Please don't insert any
casts to @code{void}.  Zero without a cast is perfectly fine as a null
pointer constant.

@node  Names
@chapter Naming Variables and Functions

Please use underscores to separate words in a name, so that the Emacs
word commands can be useful within them.  Stick to lower case; reserve
upper case for macros and @code{enum} constants, and for name-prefixes
that follow a uniform convention.

For example, you should use names like @code{ignore_space_change_flag};
don't use names like @code{iCantReadThis}.

Variables that indicate whether command-line options have been
specified should be named after the meaning of the option, not after
the option-letter.  A comment should state both the exact meaning of
the option and its letter.  For example,

@example
/* Ignore changes in horizontal whitespace (-b).  */
int ignore_space_change_flag;
@end example

When you want to define names with constant integer values, use
@code{enum} rather than @samp{#define}.  GDB knows about enumeration
constants.

Use file names of 14 characters or less, to avoid creating gratuitous
problems on System V.


@node Using Extensions
@chapter Using Non-standard Features

Many GNU facilities that already exist support a number of convenient
extensions over the comparable Unix facilities.  Whether to use these
extensions in implementing your program is a difficult question.

On the one hand, using the extensions can make a cleaner program.
On the other hand, people will not be able to build the program
unless the other GNU tools are available.  This might cause the
program to work on fewer kinds of machines.

With some extensions, it might be easy to provide both alternatives.
For example, you can define functions with a ``keyword'' @code{INLINE}
and define that as a macro to expand into either @code{inline} or
nothing, depending on the compiler.

In general, perhaps it is best not to use the extensions if you can
straightforwardly do without them, but to use the extensions if they
are a big improvement.

An exception to this rule are the large, established programs (such as
Emacs) which run on a great variety of systems.  Such programs would
be broken by use of GNU extensions.

Another exception is for programs that are used as part of
compilation: anything that must be compiled with other compilers in
order to bootstrap the GNU compilation facilities.  If these require
the GNU compiler, then no one can compile them without having them
installed already.  That would be no good.

Since most computer systems do not yet implement @sc{ANSI} C, using the
@sc{ANSI} C features is effectively using a GNU extension, so the
same considerations apply.  (Except for @sc{ANSI} features that we
discourage, such as trigraphs---don't ever use them.)

@node Semantics
@chapter Program Behavior for All Programs

Avoid arbitrary limits on the length or number of @emph{any} data
structure, including filenames, lines, files, and symbols, by allocating
all data structures dynamically.  In most Unix utilities, ``long lines
are silently truncated''.  This is not acceptable in a GNU utility.

Utilities reading files should not drop NUL characters, or any other
nonprinting characters @emph{including those with codes above 0177}.  The
only sensible exceptions would be utilities specifically intended for
interface to certain types of printers that can't handle those characters.

Check every system call for an error return, unless you know you wish to
ignore errors.  Include the system error text (from @code{perror} or
equivalent) in @emph{every} error message resulting from a failing
system call, as well as the name of the file if any and the name of the
utility.  Just ``cannot open foo.c'' or ``stat failed'' is not
sufficient.

Check every call to @code{malloc} or @code{realloc} to see if it
returned zero.  Check @code{realloc} even if you are making the block
smaller; in a system that rounds block sizes to a power of 2,
@code{realloc} may get a different block if you ask for less space.

In Unix, @code{realloc} can destroy the storage block if it returns
zero.  GNU @code{realloc} does not have this bug: if it fails, the
original block is unchanged.  Feel free to assume the bug is fixed.  If
you wish to run your program on Unix, and wish to avoid lossage in this
case, you can use the GNU @code{malloc}.

You must expect @code{free} to alter the contents of the block that was
freed.  Anything you want to fetch from the block, you must fetch before
calling @code{free}.

Use @code{getopt_long} to decode arguments, unless the argument syntax
makes this unreasonable.

When static storage is to be written in during program execution, use
explicit C code to initialize it.  Reserve C initialized declarations
for data that will not be changed.

Try to avoid low-level interfaces to obscure Unix data structures (such
as file directories, utmp, or the layout of kernel memory), since these
are less likely to work compatibly.  If you need to find all the files
in a directory, use @code{readdir} or some other high-level interface.
These will be supported compatibly by GNU.

By default, the GNU system will provide the signal handling functions of
@sc{BSD} and of @sc{POSIX}.  So GNU software should be written to use
these.

In error checks that detect ``impossible'' conditions, just abort.
There is usually no point in printing any message.  These checks
indicate the existence of bugs.  Whoever wants to fix the bugs will have
to read the source code and run a debugger.  So explain the problem with
comments in the source.  The relevant data will be in variables, which
are easy to examine with the debugger, so there is no point moving them
elsewhere.


@node Errors
@chapter Formatting Error Messages

Error messages from compilers should look like this:

@example
@var{source-file-name}:@var{lineno}: @var{message}
@end example

Error messages from other noninteractive programs should look like this:

@example
@var{program}:@var{source-file-name}:@var{lineno}: @var{message}
@end example

@noindent
when there is an appropriate source file, or like this:

@example
@var{program}: @var{message}
@end example

@noindent
when there is no relevant source file.

In an interactive program (one that is reading commands from a
terminal), it is better not to include the program name in an error
message.  The place to indicate which program is running is in the
prompt or with the screen layout.  (When the same program runs with
input from a source other than a terminal, it is not interactive and
would do best to print error messages using the noninteractive style.)

The string @var{message} should not begin with a capital letter when
it follows a program name and/or filename.  Also, it should not end
with a period.

Error messages from interactive programs, and other messages such as
usage messages, should start with a capital letter.  But they should not
end with a period.


@node Libraries
@chapter Library Behavior

Try to make library functions reentrant.  If they need to do dynamic
storage allocation, at least try to avoid any nonreentrancy aside from
that of @code{malloc} itself.

Here are certain name conventions for libraries, to avoid name
conflicts.

Choose a name prefix for the library, more than two characters long.
All external function and variable names should start with this
prefix.  In addition, there should only be one of these in any given
library member.  This usually means putting each one in a separate
source file.

An exception can be made when two external symbols are always used
together, so that no reasonable program could use one without the
other; then they can both go in the same file.

External symbols that are not documented entry points for the user
should have names beginning with @samp{_}.  They should also contain
the chosen name prefix for the library, to prevent collisions with
other libraries.  These can go in the same files with user entry
points if you like.

Static functions and variables can be used as you like and need not
fit any naming convention.


@node Portability
@chapter Portability As It Applies to GNU

Much of what is called ``portability'' in the Unix world refers to
porting to different Unix versions.  This is a secondary consideration
for GNU software, because its primary purpose is to run on top of one
and only one kernel, the GNU kernel, compiled with one and only one C
compiler, the GNU C compiler.  The amount and kinds of variation among
GNU systems on different cpu's will be like the variation among Berkeley
4.3 systems on different cpu's.

All users today run GNU software on non-GNU systems.  So supporting a
variety of non-GNU systems is desirable; simply not paramount.
The easiest way to achieve portability to a reasonable range of systems
is to use Autoconf.  It's unlikely that your program needs to know more
information about the host machine than Autoconf can provide, simply
because most of the programs that need such knowledge have already been
written.

It is difficult to be sure exactly what facilities the GNU kernel
will provide, since it isn't finished yet.  Therefore, assume you can
use anything in 4.3; just avoid using the format of semi-internal data
bases (e.g., directories) when there is a higher-level alternative
(@code{readdir}).

You can freely assume any reasonably standard facilities in the C
language, libraries or kernel, because we will find it necessary to
support these facilities in the full GNU system, whether or not we
have already done so.  The fact that there may exist kernels or C
compilers that lack these facilities is irrelevant as long as the GNU
kernel and C compiler support them.

It remains necessary to worry about differences among cpu types, such
as the difference in byte ordering and alignment restrictions.  It's
unlikely that 16-bit machines will ever be supported by GNU, so there
is no point in spending any time to consider the possibility that an
int will be less than 32 bits.

You can assume that all pointers have the same format, regardless
of the type they point to, and that this is really an integer.
There are some weird machines where this isn't true, but they aren't
important; don't waste time catering to them.  Besides, eventually
we will put function prototypes into all GNU programs, and that will
probably make your program work even on weird machines.

Since some important machines (including the 68000) are big-endian,
it is important not to assume that the address of an @code{int} object
is also the address of its least-significant byte.  Thus, don't
make the following mistake:

@example
int c;
@dots{}
while ((c = getchar()) != EOF)
        write(file_descriptor, &c, 1);
@end example

You can assume that it is reasonable to use a meg of memory.  Don't
strain to reduce memory usage unless it can get to that level.  If
your program creates complicated data structures, just make them in
core and give a fatal error if malloc returns zero.

If a program works by lines and could be applied to arbitrary
user-supplied input files, it should keep only a line in memory, because
this is not very hard and users will want to be able to operate on input
files that are bigger than will fit in core all at once.


@node User Interfaces
@chapter Standards for Command Line Interfaces

Please don't make the behavior of a utility depend on the name used
to invoke it.  It is useful sometimes to make a link to a utility
with a different name, and that should not change what it does.

Instead, use a run time option or a compilation switch or both
to select among the alternate behaviors.

Likewise, please don't make the behavior of the program depend on the
type of output device it is used with.  Device independence is an
important principle of the system's design; do not compromise it
merely to save someone from typing an option now and then.

If you think one behavior is most useful when the output is to a
terminal, and another is most useful when the output is a file or a
pipe, then it is usually best to make the default behavior the one that
is useful with output to a terminal, and have an option for the other
behavior.

Compatibility requires certain programs to depend on the type of output
device.  It would be disastrous if @code{ls} or @code{sh} did not do so
in the way all users expect.  In some of these cases, we supplement the
program with a preferred alternate version that does not depend on the
output device type.  For example, we provide a @code{dir} program much
like @code{ls} except that its default output format is always
multi-column format.

It is a good idea to follow the @sc{POSIX} guidelines for the
command-line options of a program.  The easiest way to do this is to use
@code{getopt} to parse them.  Note that the GNU version of @code{getopt}
will normally permit options anywhere among the arguments unless the
special argument @samp{--} is used.  This is not what @sc{POSIX}
specifies; it is a GNU extension.

Please define long-named options that are equivalent to the
single-letter Unix-style options.  We hope to make GNU more user
friendly this way.  This is easy to do with the GNU function
@code{getopt_long}.

One of the advantages of long-named options is that they can be
consistent from program to program.  For example, users should be able
to expect the ``verbose'' option of any GNU program which has one, to be
spelled precisely @samp{--verbose}.  To achieve this uniformity, look at
the table of common long-option names when you choose the option names
for your program.  The table appears below.

If you use names not already in the table, please send
@samp{gnu@@prep.ai.mit.edu} a list of them, with their meanings, so we
can update the table.

It is usually a good idea for file names given as ordinary arguments
to be input files only; any output files would be specified using
options (preferably @samp{-o}).  Even if you allow an output file name
as an ordinary argument for compatibility, try to provide a suitable
option as well.  This will lead to more consistency among GNU
utilities, so that there are fewer idiosyncracies for users to
remember.

Programs should support an option @samp{--version} which prints the
program's version number on standard output and exits successfully, and
an option @samp{--help} which prints option usage information on
standard output and exits successfully.  These options should inhibit
the normal function of the command; they should do nothing except print
the requested information.

@table @samp
@item after-date
@samp{-N} in @code{tar}.
@item all
@samp{-a} in @code{du}, @code{ls}, @code{nm}, @code{stty}, @code{uname},
and @code{unexpand}.
@item all-text
@samp{-a} in @code{diff}.
@item almost-all
@samp{-A} in @code{ls}.
@item append
@samp{-a} in @code{etags}, @code{tee}, @code{time}; @samp{-r} in
@code{tar}.
@item archive
@samp{-a} in @code{cp}.
@item ascii
@samp{-a} in @code{diff}.
@item assume-new
@samp{-W} in Make.
@item assume-old
@samp{-o} in Make.
@item backward-search
@samp{-B} in etags.
@item batch
Used in GDB.
@item baud
Used in GDB.
@item before
@samp{-b} in @code{tac}.
@item binary
@samp{-b} in @code{cpio} and @code{diff}.
@item block-size
Used in @code{cpio} and @code{tar}.
@item blocks
@samp{-b} in @code{head} and @code{tail}.
@item brief
Used in various programs to make output shorter.
@item bytes
@samp{-c} in @code{head}, @code{split}, and @code{tail}.
@item c++
@samp{-C} in @code{etags}.
@item catenate
@samp{-A} in @code{tar}.
@item cd
Used in various programs to specify the directory to use.
@item changes
@samp{-c} in @code{chgrp} and @code{chown}.
@item classify
@samp{-F} in @code{ls}.
@item command
@samp{-c} in @code{su}; @samp{-x} in GDB.
@item compare
@samp{-d} in @code{tar}.
@item compress
@samp{-z} in @code{tar}.
@item concatenate
@samp{-A} in @code{tar}.
@item confirmation
@samp{-w} in @code{tar}.
@item context
Used in @code{diff}.
@item core
Used in GDB.
@item count
@samp{-q} in @code{who}.
@item count-links
@samp{-l} in @code{du}.
@item create
Used in @code{tar} and @code{cpio}.
@item cxref
@samp{-x} in @code{etags}.
@item date
@samp{-d} in @code{touch}.
@item debug
@samp{-d} in @code{make}; @samp{-t} in Bison.
@item defines
@samp{-d} in Bison and @code{etags}.
@item delete
@samp{-D} in @code{tar}.
@item dereference
@samp{-L} in @code{chgrp}, @code{chown}, @code{cpio}, @code{du},
@code{ls}, and @code{tar}.
@item dereference-args
@samp{-D} in @code{du}.
@item dictionary-order
@samp{-d} in @code{look}.
@item diff
@samp{-d} in @code{tar}.
@item directory
Specify the directory to use, in various programs.  In @code{ls}, it
means to show directories themselves rather than their contents.  In
@code{rm} and @code{ln}, it means to not treat links to directories
specially.
@item discard-all
@samp{-x} in @code{strip}.
@item discard-locals
@samp{-X} in @code{strip}.
@item dry-run
@samp{-n} in Make.
@item ed
@samp{-e} in @code{diff}.
@item entire-new-file
@samp{-N} in @code{diff}.
@item environment-overrides
@samp{-e} in Make.
@item eof
@samp{-e} in @code{xargs}.
@item epoch
Used in GDB.
@item error-limit
Used in Makeinfo.
@item escape
@samp{-b} in @code{ls}.
@item exclude-from
@samp{-X} in @code{tar}.
@item exec
Used in GDB.
@item exit
@samp{-x} in @code{xargs}.
@item expand-tabs
@samp{-t} in @code{diff}.
@item expression
@samp{-e} in @code{sed}.
@item extern-only
@samp{-g} in @code{nm}.
@item extract
@samp{-i} in @code{cpio}; @samp{-x} in @code{tar}.
@item faces
@samp{-f} in @code{finger}.
@item fast
@samp{-f} in @code{su}.
@item file
@samp{-f} in @code{info}, Make, @code{mt}, and @code{tar}.  @samp{-n} in
@code{sed}; @samp{-r} in @code{touch}.
@item file-prefix
@samp{-b} in Bison.
@item file-type
@samp{-F} in @code{ls}.
@item files-from
@samp{-T} in @code{tar}.
@item fill-column
Used in Makeinfo.
@item fixed-output-files
@samp{-y} in Bison.
@item follow
@samp{-f} in @code{tail}.
@item footnote-style
Used in Makeinfo.
@item force
@samp{-f} in @code{cp}, @code{ln}, @code{mv}, and @code{rm}.
@item format
Used in @code{ls} and @code{time}.
@item forward-search
@samp{-F} in @code{etags}.
@item fullname
Used in GDB.
@item get
@samp{-x} in @code{tar}.
@item graphic
@samp{-i} in @code{ul}.
@item group
@samp{-g} in @code{install}.
@item header
@samp{-h} in @code{objdump}.
@item heading
@samp{-H} in @code{who}.
@item help
Used to ask for brief usage information.
@item hide-control-chars
@samp{-q} in @code{ls}.
@item idle
@samp{-u} in @code{who}.
@item ifdef
@samp{-D} in @code{diff}.
@item ignore
@samp{-I} in @code{ls}.
@item ignore-all-space
@samp{-w} in @code{diff}.
@item ignore-backups
@samp{-B} in @code{ls}.
@item ignore-blank-lines
@samp{-B} in @code{diff}.
@item ignore-case
@samp{-f} in @code{look}; @samp{-i} in @code{diff}.
@item ignore-errors
@samp{-i} in Make.
@item ignore-indentation
@samp{-S} in @code{etags}.
@item ignore-init-file
@samp{-f} in Oleo.
@item ignore-interrupts
@samp{-i} in @code{tee}.
@item ignore-matching-lines
@samp{-I} in @code{diff}.
@item ignore-space-change
@samp{-b} in @code{diff}.
@item ignore-zeros
@samp{-i} in @code{tar}.
@item include
@samp{-i} in @code{etags}.
@item include-dir
@samp{-I} in Make.
@item incremental
@samp{-G} in @code{tar}.
@item info
@samp{-i}, @samp{-l}, and @samp{-m} in Finger.
@item initial
@samp{-i} in @code{expand}.
@item initial-tab
@samp{-T} in @code{diff}.
@item inode
@samp{-i} in @code{ls}.
@item interactive
@samp{-i} in @code{cp}, @code{ln}, @code{mv}, @code{rm}; @samp{-p} in
@code{xargs}; @samp{-w} in @code{tar}.
@item jobs
@samp{-j} in Make.
@item just-print
@samp{-n} in Make.
@item keep-going
@samp{-k} in Make.
@item kilobytes
@samp{-k} in @code{du} and @code{ls}.
@item line-bytes
@samp{-C} in @code{split}.
@item lines
Used in @code{split}, @code{head}, and @code{tail}.
@item link
@samp{-l} in @code{cpio}.
@item list
@samp{-t} in @code{cpio}.
@item list
@samp{-t} in @code{tar}.
@item literal
@samp{-N} in @code{ls}.
@item load-average
@samp{-l} in Make.
@item login
Used in @code{su}.
@item machine
@samp{-m} in @code{uname}.
@item make-directories
@samp{-d} in @code{cpio}.
@item makefile
@samp{-f} in Make.
@item mapped
Used in GDB.
@item max-args
@samp{-n} in @code{xargs}.
@item max-chars
@samp{-n} in @code{xargs}.
@item max-lines
@samp{-l} in @code{xargs}.
@item max-load
@samp{-l} in Make.
@item max-procs
@samp{-P} in @code{xargs}.
@item mesg
@samp{-T} in @code{who}.
@item message
@samp{-T} in @code{who}.
@item minimal
@samp{-d} in @code{diff}.
@item mode
@samp{-m} in @code{install}, @code{mkdir}, and @code{mkfifo}.
@item modification-time
@samp{-m} in @code{tar}.
@item multi-volume
@samp{-M} in @code{tar}.
@item name-prefix
@samp{-a} in Bison.
@item new-file
@samp{-W} in Make.
@item no-builtin-rules
@samp{-r} in Make.
@item no-create
@samp{-c} in @code{touch}.
@item no-defines
@samp{-D} in @code{etags}.
@item no-dereference
@samp{-d} in @code{cp}.
@item no-keep-going
@samp{-S} in Make.
@item no-lines
@samp{-l} in Bison.
@item no-prof
@samp{-e} in @code{gprof}.
@item no-sort
@samp{-p} in @code{nm}.
@item no-split
Used in Makeinfo.
@item no-static
@samp{-a} in @code{gprof}.
@item no-time
@samp{-E} in @code{gprof}.
@item no-validate
Used in Makeinfo.
@item no-warn
Used in various programs to inhibit warnings.
@item node
@samp{-n} in @code{info}.
@item nodename
@samp{-n} in @code{uname}.
@item nonmatching
@samp{-f} in @code{cpio}.
@item nstuff
@samp{-n} in @code{objdump}.
@item null
@samp{-0} in @code{xargs}.
@item number
@samp{-n} in @code{cat}.
@item number-nonblank
@samp{-b} in @code{cat}.
@item numeric-sort
@samp{-n} in @code{nm}.
@item numeric-uid-gid
@samp{-n} in @code{cpio} and @code{ls}.
@item nx
Used in GDB.
@item old-archive
@samp{-o} in @code{tar}.
@item old-file
@samp{-o} in Make.
@item one-file-system
@samp{-l} in @code{tar}, @code{cp}, and @code{du}.
@item only-prof
@samp{-f} in @code{gprof}.
@item only-time
@samp{-F} in @code{gprof}.
@item output
In various programs, specify the output file name.
@item override
@samp{-o} in @code{rm}.
@item owner
@samp{-o} in @code{install}.
@item paginate
@samp{-l} in @code{diff}.
@item paragraph-indent
Used in Makeinfo.
@item parents
@samp{-p} in @code{mkdir} and @code{rmdir}.
@item pass-all
@samp{-p} in @code{ul}.
@item pass-through
@samp{-p} in @code{cpio}.
@item port
@samp{-P} in @code{finger}.
@item portability
@samp{-c} in @code{cpio} and @code{tar}.
@item preserve
Used in @code{tar} and @code{cp}.
@item preserve-environment
@samp{-p} in @code{su}.
@item preserve-modification-time
@samp{-m} in @code{cpio}.
@item preserve-order
@samp{-s} in @code{tar}.
@item preserve-permissions
@samp{-p} in @code{tar}.
@item print
@samp{-l} in @code{diff}.
@item print-chars
@samp{-L} in @code{cmp}.
@item print-data-base
@samp{-p} in Make.
@item print-directory
@samp{-w} in Make.
@item print-file-name
@samp{-o} in @code{nm}.
@item print-symdefs
@samp{-s} in @code{nm}.
@item question
@samp{-q} in Make.
@item quiet
Used in many programs to inhibit the usual output.  @strong{Note:} every
program accepting @samp{--quiet} should accept @samp{--silent} as a
synonym.
@item quote-name
@samp{-Q} in @code{ls}.
@item rcs
@samp{-n} in @code{diff}.
@item read-full-blocks
@samp{-B} in @code{tar}.
@item readnow
Used in GDB.
@item recon
@samp{-n} in Make.
@item record-number
@samp{-R} in @code{tar}.
@item recursive
Used in @code{chgrp}, @code{chown}, @code{cp}, @code{ls}, @code{diff},
and @code{rm}.
@item reference-limit
Used in Makeinfo.
@item regex
@samp{-r} in @code{tac}.
@item release
@samp{-r} in @code{uname}.
@item relocation
@samp{-r} in @code{objdump}.
@item rename
@samp{-r} in @code{cpio}.
@item replace
@samp{-i} in @code{xargs}.
@item report-identical-files
@samp{-s} in @code{diff}.
@item reset-access-time
@samp{-a} in @code{cpio}.
@item reverse
@samp{-r} in @code{ls} and @code{nm}.
@item reversed-ed
@samp{-f} in @code{diff}.
@item same-order
@samp{-s} in @code{tar}.
@item same-permissions
@samp{-p} in @code{tar}.
@item save
@samp{-g} in @code{stty}.
@item se
Used in GDB.
@item separate-dirs
@samp{-S} in @code{du}.
@item separator
@samp{-s} in @code{tac}.
@item shell
@samp{-s} in @code{su}.
@item show-all
@samp{-A} in @code{cat}.
@item show-c-function
@samp{-p} in @code{diff}.
@item show-ends
@samp{-E} in @code{cat}.
@item show-function-line
@samp{-F} in @code{diff}.
@item show-tabs
@samp{-T} in @code{cat}.
@item silent
Used in many programs to inhibit the usual output.
@strong{Note:} every program accepting
@samp{--silent} should accept @samp{--quiet} as a synonym.
@item size
@samp{-s} in @code{ls}.
@item sort
Used in @code{ls}.
@item sparse
@samp{-S} in @code{tar}.
@item speed-large-files
@samp{-H} in @code{diff}.
@item squeeze-blank
@samp{-s} in @code{cat}.
@item starting-file
Used in @code{tar} and @code{diff} to specify which file within
a directory to start processing with.
@item stop
@samp{-S} in Make.
@item strip
@samp{-s} in @code{install}.
@item strip-all
@samp{-s} in @code{strip}.
@item strip-debug
@samp{-S} in @code{strip}.
@item suffix
@samp{-S} in @code{cp}, @code{ln}, @code{mv}.
@item sum
@samp{-s} in @code{gprof}.
@item summarize
@samp{-s} in @code{du}.
@item symbolic
@samp{-s} in @code{ln}.
@item symbols
Used in GDB and @code{objdump}.
@item sysname
@samp{-s} in @code{uname}.
@item tabs
@samp{-t} in @code{expand} and @code{unexpand}.
@item tabsize
@samp{-T} in @code{ls}.
@item terminal
@samp{-T} in @code{tput} and @code{ul}.
@item text
@samp{-a} in @code{diff}.
@item time
Used in @code{ls} and @code{touch}.
@item to-stdout
@samp{-O} in @code{tar}.
@item total
@samp{-c} in @code{du}.
@item touch
@samp{-t} in Make and @code{ranlib}.
@item tty
Used in GDB.
@item typedefs
@samp{-t} in @code{etags}.
@item typedefs-and-c++
@samp{-T} in @code{etags}.
@item uncompress
@samp{-z} in @code{tar}.
@item unconditional
@samp{-u} in @code{cpio}.
@item undefined-only
@samp{-u} in @code{nm}.
@item update
@samp{-u} in @code{cp}, @samp{etags}, @samp{mv}, @samp{tar}.
@item verbose
Print more information about progress.  Many programs support this.
@item verify
@samp{-W} in @code{tar}.
@item version
Print the version number.
@item version-control
@samp{-V} in @code{cp}, @code{ln}, @code{mv}.
@item vgrind
@samp{-v} in @code{etags}.
@item volume
@samp{-V} in @code{tar}.
@item what-if
@samp{-W} in Make.
@item width
@samp{-w} in @code{ls}.
@item writable
@samp{-T} in @code{who}.
@item zeros
@samp{-z} in @code{gprof}.
@end table

@node Documentation
@chapter Documenting Programs

Please use Texinfo for documenting GNU programs.  See the Texinfo
manual, either the hardcopy or the version in the GNU Emacs Info
subsystem (@kbd{C-h i}).  See existing GNU Texinfo files (e.g., those
under the @file{man/} directory in the GNU Emacs distribution) for
examples.

The title page of the manual should state the version of the program
which the manual applies to.  The Top node of the manual should also
contain this information.  If the manual is changing more frequently
than or independent of the program, also state a version number for
the manual in both of these places.

The manual should document all command-line arguments and all
commands.  It should give examples of their use.  But don't organize
the manual as a list of features.  Instead, organize it by the
concepts a user will have before reaching that point in the manual.
Address the goals that a user will have in mind, and explain how to
accomplish them.  Don't use Unix man pages as a model for how to
write GNU documentation; they are a bad example to follow.

The manual should have a node named @samp{@var{program} Invocation} or
@samp{Invoking @var{program}}, where @var{program} stands for the name
of the program being described, as you would type it in the shell to run
the program.  This node (together with its subnodes, if any) should
describe the program's command line arguments and how to run it (the
sort of information people would look in a man page for).  Start with an
@samp{@@example} containing a template for all the options and arguments
that the program uses.

Alternatively, put a menu item in some menu whose item name fits one of
the above patterns.  This identifies the node which that item points to
as the node for this purpose, regardless of the node's actual name.

There will be automatic features for specifying a program name and
quickly reading just this part of its manual.

If one manual describes several programs, it should have such a node for
each program described.

In addition to its manual, the package should have a file named
@file{NEWS} which contains a list of user-visible changes worth
mentioning.  In each new release, add items to the front of the file and
identify the version they pertain to.  Don't discard old items; leave
them in the file after the newer items.  This way, a user upgrading from
any previous version can see what is new.

If the @file{NEWS} file gets very long, move some of the older items
into a file named @file{ONEWS} and put a note at the end referring the
user to that file.

It is ok to supply a man page for the program as well as a Texinfo
manual if you wish to.  But keep in mind that supporting a man page
requires continual effort, each time the program is changed.  Any time
you spend on the man page is time taken away from more useful things you
could contribute.

Thus, even if a user volunteers to donate a man page, you may find this
gift costly to accept.  Unless you have time on your hands, it may be
better to refuse the man page unless the same volunteer agrees to take
full responsibility for maintaining it---so that you can wash your hands
of it entirely.  If the volunteer ceases to do the job, then don't feel
obliged to pick it up yourself; it may be better to withdraw the man
page until another volunteer offers to carry on with it.

Alternatively, if you expect the discrepancies to be small enough that
the man page remains useful, put a prominent note near the beginning of
the man page explaining that you don't maintain it and that the Texinfo
manual is more authoritative, and describing how to access the Texinfo
documentation.

@node Releases
@chapter Making Releases

Package the distribution of Foo version 69.96 in a gzipped tar file
named @file{foo-69.96.tar.gz}.  It should unpack into a subdirectory
named @file{foo-69.96}.

Building and installing the program should never modify any of the files
contained in the distribution.  This means that all the files that form
part of the program in any way must be classified into @dfn{source
files} and @dfn{non-source files}.  Source files are written by humans
and never changed automatically; non-source files are produced from
source files by programs under the control of the Makefile.

Naturally, all the source files must be in the distribution.  It is okay
to include non-source files in the distribution, provided they are
up-to-date and machine-independent, so that building the distribution
normally will never modify them.  We commonly included non-source files
produced by Bison, Lex, @TeX{}, and Makeinfo; this helps avoid
unnecessary dependencies between our distributions, so that users can
install whichever packages they want to install.

Non-source files that might actually be modified by building and
installing the program should @strong{never} be included in the
distribution.  So if you do distribute non-source files, always make
sure they are up to date when you make a new distribution.

Make sure that the directory into which the distribution unpacks (as
well as any subdirectories) are all world-writable (octal mode 777).
This is so that old versions of @code{tar} which preserve the
ownership and permissions of the files from the tar archive will be
able to extract all the files even if the user is unprivileged. 

Make sure that no file name in the distribution is more than 14
characters long.  Likewise, no file created by building the program
should have a name longer than 14 characters.  The reason for this is
that some systems adhere to a foolish interpretation of the POSIX
standard, and refuse to open a longer name, rather than truncating as
they did in the past.

Don't include any symbolic links in the distribution itself.  If the tar
file contains symbolic links, then people cannot even unpack it on
systems that don't support symbolic links.  Also, don't use multiple
names for one file in different directories, because certain file
systems cannot handle this and that prevents unpacking the
distribution.

Try to make sure that all the file names will be unique on MS-DOG.  A
name on MS-DOG consists of up to 8 characters, optionally followed by a
period and up to three characters.  MS-DOG will truncate extra
characters both before and after the period.  Thus,
@file{foobarhacker.c} and @file{foobarhacker.o} are not ambiguous; they
are truncated to @file{foobarha.c} and @file{foobarha.o}, which are
distinct.

Include in your distribution a copy of the @file{texinfo.tex} you used
to test print any @file{*.texinfo} files.

Likewise, if your program uses small GNU software packages like regex,
getopt, obstack, or termcap, include them in the distribution file.
Leaving them out would make the distribution file a little smaller at
the expense of possible inconvenience to a user who doesn't know what
other files to get.

@contents

@bye
