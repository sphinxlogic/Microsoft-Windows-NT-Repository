@comment This file is included by both standards.texi and make.texinfo.
@comment It was broken out of standards.texi on 1/6/93 by roland.

@node Makefile Conventions
@chapter Makefile Conventions
@comment standards.texi does not print an index, but make.texinfo does.
@cindex makefile, conventions for
@cindex conventions for makefiles
@cindex standards for makefiles

This chapter describes conventions for writing the Makefiles for GNU programs.

@menu
* Makefile Basics::
* Utilities in Makefiles::
* Standard Targets::
* Command Variables::
* Directory Variables::
@end menu

@node Makefile Basics
@section General Conventions for Makefiles

Every Makefile should contain this line:

@example
SHELL = /bin/sh
@end example

@noindent
to avoid trouble on systems where the @code{SHELL} variable might be
inherited from the environment.  (This is never a problem with GNU
@code{make}.)

Don't assume that @file{.} is in the path for command execution.  When
you need to run programs that are a part of your package during the
make, please make sure that it uses @file{./} if the program is built as
part of the make or @file{$(srcdir)/} if the file is an unchanging part
of the source code.  Without one of these prefixes, the current search
path is used.  

The distinction between @file{./} and @file{$(srcdir)/} is important
when using the @samp{--srcdir} option to @file{configure}.  A rule of
the form:

@smallexample
foo.1 : foo.man sedscript
        sed -e sedscript foo.man > foo.1
@end smallexample

@noindent
will fail when the current directory is not the source directory,
because @file{foo.man} and @file{sedscript} are not in the current
directory.

When using GNU @code{make}, relying on @samp{VPATH} to find the source
file will work in the case where there is a single dependency file,
since the @file{make} automatic variable @samp{$<} will represent the
source file wherever it is.  (Many versions of @code{make} set @samp{$<}
only in implicit rules.)  A makefile target like

@smallexample
foo.o : bar.c
        $(CC) -I. -I$(srcdir) $(CFLAGS) -c bar.c -o foo.o
@end smallexample

@noindent
should instead be written as

@smallexample
foo.o : bar.c
        $(CC) $(CFLAGS) $< -o $@@
@end smallexample

@noindent
in order to allow @samp{VPATH} to work correctly.  When the target has
multiple dependencies, using an explicit @samp{$(srcdir)} is the easiest
way to make the rule work well.  For example, the target above for
@file{foo.1} is best written as:

@smallexample
foo.1 : foo.man sedscript
        sed -s $(srcdir)/sedscript $(srcdir)/foo.man > foo.1
@end smallexample

@node Utilities in Makefiles
@section Utilities in Makefiles

Write the Makefile commands (and any shell scripts, such as
@code{configure}) to run in @code{sh}, not in @code{csh}.  Don't use any
special features of @code{ksh} or @code{bash}.

The @code{configure} script and the Makefile rules for building and
installation should not use any utilities directly except these:

@example
cat cmp cp echo egrep expr grep
ln mkdir mv pwd rm rmdir sed test touch
@end example

Stick to the generally supported options for these programs.  For
example, don't use @samp{mkdir -p}, convenient as it may be, because
most systems don't support it.

The Makefile rules for building and installation can also use compilers
and related programs, but should do so via @code{make} variables so that the
user can substitute alternatives.  Here are some of the programs we
mean:

@example
ar bison cc flex install ld lex
make makeinfo ranlib texi2dvi yacc
@end example

When you use @code{ranlib}, you should test whether it exists, and run
it only if it exists, so that the distribution will work on systems that
don't have @code{ranlib}.

If you use symbolic links, you should implement a fallback for systems
that don't have symbolic links.

It is ok to use other utilities in Makefile portions (or scripts)
intended only for particular systems where you know those utilities to
exist.

@node Standard Targets
@section Standard Targets for Users

All GNU programs should have the following targets in their Makefiles:

@table @samp
@item all
Compile the entire program.  This should be the default target.  This
target need not rebuild any documentation files; Info files should
normally be included in the distribution, and DVI files should be made
only when explicitly asked for.

@item install
Compile the program and copy the executables, libraries, and so on to
the file names where they should reside for actual use.  If there is a
simple test to verify that a program is properly installed, this target
should run that test.

The commands should create all the directories in which files are to be
installed, if they don't already exist.  This includes the directories
specified as the values of the variables @code{prefix} and
@code{exec_prefix}, as well as all subdirectories that are needed.
One way to do this is by means of an @code{installdirs} target
as described below.

Use @samp{-} before any command for installing a man page, so that
@code{make} will ignore any errors.  This is in case there are systems
that don't have the Unix man page documentation system installed.

The way to install Info files is to copy them into @file{$(infodir)}
with @code{$(INSTALL_DATA)} (@pxref{Command Variables}), and then run
the @code{install-info} program if it is present.  @code{install-info}
is a script that edits the Info @file{dir} file to add or update the
menu entry for the given Info file; it will be part of the Texinfo package.
Here is a sample rule to install an Info file:

@smallexample
$(infodir)/foo.info: foo.info
# There may be a newer info file in . than in srcdir.
# Run install-info only if it exists.
# Use `if' instead of just prepending `-' to the
# line so we notice real errors from install-info.
        -if test -f foo.info; then d=.; else d=$(srcdir); fi; \
        $(INSTALL_DATA) $$d/foo.info $@@; \
        if install-info --version >/dev/null 2>&1; then \
          install-info --infodir=$(infodir) $$d/foo.info; \
        else true; fi
@end smallexample

@item uninstall
Delete all the installed files that the @samp{install} target would
create (but not the noninstalled files such as @samp{make all} would
create).

@comment The gratuitous blank line here is to make the table look better
@comment in the printed Make manual.  Please leave it in.
@item clean

Delete all files from the current directory that are normally created by
building the program.  Don't delete the files that record the
configuration.  Also preserve files that could be made by building, but
normally aren't because the distribution comes with them.

Delete @file{.dvi} files here if they are not part of the distribution.

@item distclean
Delete all files from the current directory that are created by
configuring or building the program.  If you have unpacked the source
and built the program without creating any other files, @samp{make
distclean} should leave only the files that were in the distribution.

@item mostlyclean
Like @samp{clean}, but may refrain from deleting a few files that people
normally don't want to recompile.  For example, the @samp{mostlyclean}
target for GCC does not delete @file{libgcc.a}, because recompiling it
is rarely necessary and takes a lot of time.

@item realclean
Delete everything from the current directory that can be reconstructed
with this Makefile.  This typically includes everything deleted by
@code{distclean}, plus more: C source files produced by Bison, tags tables,
Info files, and so on.

One exception, however: @samp{make realclean} should not delete
@file{configure} even if @file{configure} can be remade using a rule in
the Makefile.  More generally, @samp{make realclean} should not delete
anything that needs to exist in order to run @file{configure}
and then begin to build the program.

@item TAGS
Update a tags table for this program.

@item info
Generate any Info files needed.  The best way to write the rules is as
follows:

@smallexample
info: foo.info

foo.info: foo.texi chap1.texi chap2.texi
        $(MAKEINFO) $(srcdir)/foo.texi
@end smallexample

@noindent
You must define the variable @code{MAKEINFO} in the Makefile.  It should
run the @code{makeinfo} program, which is part of the Texinfo
distribution.

@item dvi
Generate DVI files for all TeXinfo documentation.  
For example:

@smallexample
dvi: foo.dvi

foo.dvi: foo.texi chap1.texi chap2.texi
        $(TEXI2DVI) $(srcdir)/foo.texi
@end smallexample

@noindent
You must define the variable @code{TEXI2DVI} in the Makefile.  It should
run the program @code{texi2dvi}, which is part of the Texinfo
distribution.  Alternatively, write just the dependencies, and allow GNU
Make to provide the command.

@item dist
Create a distribution tar file for this program.  The tar file should be
set up so that the file names in the tar file start with a subdirectory
name which is the name of the package it is a distribution for.  This
name can include the version number.

For example, the distribution tar file of GCC version 1.40 unpacks into
a subdirectory named @file{gcc-1.40}.

The easiest way to do this is to create a subdirectory appropriately
named, use @code{ln} or @code{cp} to install the proper files in it, and
then @code{tar} that subdirectory.

The @code{dist} target should explicitly depend on all non-source files
that are in the distribution, to make sure they are up to date in the
distribution.  
@xref{Releases, , Making Releases, standards, GNU Coding Standards}.

@item check
Perform self-tests (if any).  The user must build the program before
running the tests, but need not install the program; you should write
the self-tests so that they work when the program is built but not
installed.
@end table

The following targets are suggested as conventional names, for programs
in which they are useful.

@table @code
@item installcheck
Perform installation tests (if any).  The user must build and install
the program before running the tests.  You should not assume that
@file{$(bindir)} is in the search path.  

@item installdirs
It's useful to add a target named @samp{installdirs} to create the
directories where files are installed, and their parent directories.
There is a script called @file{mkinstalldirs} which is convenient for
this; find it in the Texinfo package.@c It's in /gd/gnu/lib/mkinstalldirs.
You can use a rule like this:

@comment This has been carefully formatted to look decent in the Make manual.
@comment Please be sure not to make it extend any further to the right.--roland
@smallexample
# Make sure all installation directories (e.g. $(bindir))
# actually exist by making them if necessary.
installdirs: mkinstalldirs
        $(srcdir)/mkinstalldirs $(bindir) $(datadir) \
                                $(libdir) $(infodir) \
                                $(mandir)
@end smallexample
@end table

@node Command Variables
@section Variables for Specifying Commands

Makefiles should provide variables for overriding certain commands, options,
and so on.

In particular, you should run most utility programs via variables.
Thus, if you use Bison, have a variable named @code{BISON} whose default
value is set with @samp{BISON = bison}, and refer to it with
@code{$(BISON)} whenever you need to use Bison.

File management utilities such as @code{ln}, @code{rm}, @code{mv}, and
so on, need not be referred to through variables in this way, since users
don't need to replace them with other programs.

Each program-name variable should come with an options variable that is
used to supply options to the program.  Append @samp{FLAGS} to the
program-name variable name to get the options variable name---for
example, @code{BISONFLAGS}.  (The name @code{CFLAGS} is an exception to
this rule, but we keep it because it is standard.)  Use @code{CPPFLAGS}
in any compilation command that runs the preprocessor, and use
@code{LDFLAGS} in any compilation command that does linking as well as
in any direct use of @code{ld}.

If there are C compiler options that @emph{must} be used for proper
compilation of certain files, do not include them in @code{CFLAGS}.
Users expect to be able to specify @code{CFLAGS} freely themselves.
Instead, arrange to pass the necessary options to the C compiler
independently of @code{CFLAGS}, by writing them explicitly in the
compilation commands or by defining an implicit rule, like this:

@smallexample
CFLAGS = -g
ALL_CFLAGS = -I. $(CFLAGS)
.c.o:
        $(CC) -c $(CPPFLAGS) $(ALL_CFLAGS) $<
@end smallexample

Do include the @samp{-g} option in @code{CFLAGS}, because that is not
@emph{required} for proper compilation.  You can consider it a default
that is only recommended.  If the package is set up so that it is
compiled with GCC by default, then you might as well include @samp{-O}
in the default value of @code{CFLAGS} as well.

Put @code{CFLAGS} last in the compilation command, after other variables
containing compiler options, so the user can use @code{CFLAGS} to
override the others.

Every Makefile should define the variable @code{INSTALL}, which is the
basic command for installing a file into the system.

Every Makefile should also define the variables @code{INSTALL_PROGRAM}
and @code{INSTALL_DATA}.  (The default for each of these should be
@code{$(INSTALL)}.)  Then it should use those variables as the commands
for actual installation, for executables and nonexecutables
respectively.  Use these variables as follows:

@example
$(INSTALL_PROGRAM) foo $(bindir)/foo
$(INSTALL_DATA) libfoo.a $(libdir)/libfoo.a
@end example

@noindent
Always use a file name, not a directory name, as the second argument of
the installation commands.  Use a separate command for each file to be
installed.

@node Directory Variables
@section Variables for Installation Directories

Installation directories should always be named by variables, so it is
easy to install in a nonstandard place.  The standard names for these
variables are:

@table @samp
@item prefix
A prefix used in constructing the default values of the variables listed
below.  The default value of @code{prefix} should be @file{/usr/local}
(at least for now).

@item exec_prefix
A prefix used in constructing the default values of the some of the
variables listed below.  The default value of @code{exec_prefix} should
be @code{$(prefix)}.

Generally, @code{$(exec_prefix)} is used for directories that contain
machine-specific files (such as executables and subroutine libraries),
while @code{$(prefix)} is used directly for other directories.

@item bindir
The directory for installing executable programs that users can run.
This should normally be @file{/usr/local/bin}, but write it as
@file{$(exec_prefix)/bin}.

@item libdir
The directory for installing executable files to be run by the program
rather than by users.  Object files and libraries of object code should
also go in this directory.  The idea is that this directory is used for
files that pertain to a specific machine architecture, but need not be
in the path for commands.  The value of @code{libdir} should normally be
@file{/usr/local/lib}, but write it as @file{$(exec_prefix)/lib}.

@item datadir
The directory for installing read-only data files which the programs
refer to while they run.  This directory is used for files which are
independent of the type of machine being used.  This should normally be
@file{/usr/local/lib}, but write it as @file{$(prefix)/lib}.

@item statedir
The directory for installing data files which the programs modify while
they run.  These files should be independent of the type of machine
being used, and it should be possible to share them among machines at a
network installation.  This should normally be @file{/usr/local/lib},
but write it as @file{$(prefix)/lib}.

@item includedir
@c rewritten to avoid overfull hbox --roland
The directory for installing header files to be included by user
programs with the C @samp{#include} preprocessor directive.  This
should normally be @file{/usr/local/include}, but write it as
@file{$(prefix)/include}.

Most compilers other than GCC do not look for header files in
@file{/usr/local/include}.  So installing the header files this way is
only useful with GCC.  Sometimes this is not a problem because some
libraries are only really intended to work with GCC.  But some libraries
are intended to work with other compilers.  They should install their
header files in two places, one specified by @code{includedir} and one
specified by @code{oldincludedir}.

@item oldincludedir
The directory for installing @samp{#include} header files for use with
compilers other than GCC.  This should normally be @file{/usr/include}.

The Makefile commands should check whether the value of
@code{oldincludedir} is empty.  If it is, they should not try to use
it; they should cancel the second installation of the header files.

A package should not replace an existing header in this directory unless
the header came from the same package.  Thus, if your Foo package
provides a header file @file{foo.h}, then it should install the header
file in the @code{oldincludedir} directory if either (1) there is no
@file{foo.h} there or (2) the @file{foo.h} that exists came from the Foo
package.

To tell whether @file{foo.h} came from the Foo package, put a magic
string in the file---part of a comment---and grep for that string.

@item mandir
The directory for installing the man pages (if any) for this package.
It should include the suffix for the proper section of the
manual---usually @samp{1} for a utility.  It will normally be
@file{/usr/local/man/man1}, but you should write it as
@file{$(prefix)/man/man1}. 

@item man1dir
The directory for installing section 1 man pages.
@item man2dir
The directory for installing section 2 man pages.
@item @dots{}
Use these names instead of @samp{mandir} if the package needs to install man
pages in more than one section of the manual.

@strong{Don't make the primary documentation for any GNU software be a
man page.  Write a manual in Texinfo instead.  Man pages are just for
the sake of people running GNU software on Unix, which is a secondary
application only.}

@item manext
The file name extension for the installed man page.  This should contain
a period followed by the appropriate digit; it should normally be @samp{.1}.

@item man1ext
The file name extension for installed section 1 man pages.
@item man2ext
The file name extension for installed section 2 man pages.
@item @dots{}
Use these names instead of @samp{manext} if the package needs to install man
pages in more than one section of the manual.

@item infodir
The directory for installing the Info files for this package.  By
default, it should be @file{/usr/local/info}, but it should be written
as @file{$(prefix)/info}.

@item srcdir
The directory for the sources being compiled.  The value of this
variable is normally inserted by the @code{configure} shell script.
@end table

For example:

@smallexample
@c I have changed some of the comments here slightly to fix an overfull
@c hbox, so the make manual can format correctly. --roland
# Common prefix for installation directories.
# NOTE: This directory must exist when you start the install.
prefix = /usr/local
exec_prefix = $(prefix)
# Where to put the executable for the command `gcc'.
bindir = $(exec_prefix)/bin
# Where to put the directories used by the compiler.
libdir = $(exec_prefix)/lib
# Where to put the Info files.
infodir = $(prefix)/info
@end smallexample

If your program installs a large number of files into one of the
standard user-specified directories, it might be useful to group them
into a subdirectory particular to that program.  If you do this, you
should write the @code{install} rule to create these subdirectories.

Do not expect the user to include the subdirectory name in the value of
any of the variables listed above.  The idea of having a uniform set of
variable names for installation directories is to enable the user to
specify the exact same values for several different GNU packages.  In
order for this to be useful, all the packages must be designed so that
they will work sensibly when the user does so.

